\documentclass{mitsuba}

\title{Icemap}
\author{Matthew Skala}

%%%%%%%%%%%%%%%%%%%%%%%%%%%%%%%%%%%%%%%%%%%%%%%%%%%%%%%%%%%%%%%%%%%%%%%%
%%%%%%%%%%%%%%%%%%%%%%%%%%%%%%%%%%%%%%%%%%%%%%%%%%%%%%%%%%%%%%%%%%%%%%%%
%%%%%%%%%%%%%%%%%%%%%%%%%%%%%%%%%%%%%%%%%%%%%%%%%%%%%%%%%%%%%%%%%%%%%%%%

\begin{document}

\maketitle

\tableofcontents

%%%%%%%%%%%%%%%%%%%%%%%%%%%%%%%%%%%%%%%%%%%%%%%%%%%%%%%%%%%%%%%%%%%%%%%%
%%%%%%%%%%%%%%%%%%%%%%%%%%%%%%%%%%%%%%%%%%%%%%%%%%%%%%%%%%%%%%%%%%%%%%%%
%%%%%%%%%%%%%%%%%%%%%%%%%%%%%%%%%%%%%%%%%%%%%%%%%%%%%%%%%%%%%%%%%%%%%%%%

\chapter{Introduction}

This manual describes Icemap, which is a C code generator for static maps. 
That is a programming tool used in building computer software.  It solves a
specific and somewhat obscure problem.  Icemap is part of the Tsukurimashou
Project, which focuses on Japanese-language fonts; a need for something like
it happened to come up several times in different parts of the project, so
it made sense to create the tool.  It is hoped that it may also be useful to
others.

The use case for Icemap is as follows.  Let $F$ be a function or partial
function specified as tabular data.  It takes an input and produces an
output, and although it might in principle be possible to calculate the
output by doing arithmetic on the input, it is more convenient to just list
the values.  Such a function might look something like this.

{\hspace*{\fill}\begin{tabular}{c|c}
  in & out \\ \hline
  3 & ``fizz'' \\
  5 & ``buzz'' \\
  6 & ``fizz'' \\
  9 & ``fizz'' \\
  10 & ``buzz'' \\
  12 & ``fizz'' \\
  15 & ``fizz-buzz''
\end{tabular}\hspace*{\fill}\par}

Suppose it is desired to embed this function in a C program.  Other code
will be frequently invoking the function on different inputs and using the
result in some way.  We want it to be efficient, both in space and time.  We
don't expect the function to change often, maybe never at all; at the very
least, we don't expect the user to be changing the function at run time.  We
will call this kind of function a \emph{static map}. 

It makes sense for the function to be hardcoded:  written into C-language
source code that will be processed by the compiler and made into an integral
part of the program.  Icemap is a tool for generating such source code.

The obvious thing would be for a human being to write the C code
implementing the static map.  Fine.  If we are doing it only once, that may
work well.  When the table is small, there will be very little human work
involved; possibly just cutting and pasting the table values from wherever
they come from into a C file and adding some punctuation to put it into
valid C syntax.  We can imagine a number of different algorithms and data
structures for storing and looking up the function values.  In the fizz-buzz
example, where inputs are integers up to 15 and outputs are strings, the
simplest thing might be to build a \texttt{char *array[16]}, and do simple
array indexing.  If we wanted to save space we could do smarter things, like
offsetting the indices, or even doing the modulo operations to
\emph{compute} instead of \emph{looking up} the function values.  This
all works as long as we are working on a reasonable-sized function,
willing to spend human effort on it, and we never change the function.

But consider:
\begin{itemize}
  \item larger function tables, with perhaps a few thousand rows;
  \item lazy humans; and
  \item functions that do change from time to time, though not often.
\end{itemize}

There are a number of such tables in the Tsukurimashou Project.  Most of
them derive from Unicode Consortium data products.  IDSgrep uses a built-in
table of character widths for word-wrapping mixed Japanese-English
dictionary entries in the a terminal window; this table derives from the
Unicode \texttt{EastAsianWidth.txt} file.  FontAnvil has a number of
built-in tables for translating among character encoding formats; as well as
various static maps used in parsing script files and text-based font
formats.

FontAnvil's tables are in legacy code descended from an earlier package
called PfaEdit.  They were apparently handwritten by the original author of
PfaEdit.  Exactly where he got the data is not clear (certainly not well
documented).  Some of the code looks like it may have been generated by some
kind of automated generator, but if so, that original generator has been
lost.  Some of the tables include clever optimizations human-designed for
the specific tables and depending on the structure of the function in
question.  For instance, the table from Shift-JIS to Unicode is implemented
by doing some bit shifting to convert Shift-JIS to JIS, then looking up in
either the JIS table or some other table depending on something, and then
doing multiplication and modulo to squeeze the two-dimensional JIS array
into a more memory-efficient layout.  Other tables use other optimizations,
such as cascading lookups through multiple layers of arrays that point at
each other.  The lookup processes include some built-in undocumented choices
on how to handle ambiguous cases.  Updating the data for a new Unicode
version, or changing those choices, is effectively impossible.

In the history of PfaEdit's evolution into FontForge and then FontAnvil,
there have been at least two rounds of attempts to reverse-engineer the
encoding tables and create code generators to help maintenance.  Those code
generators sit abandoned in the FontForge source tree with fragmentary
documentation.  Some of FontForge's other static mapping needs (in
particular, the lookup table from Unicode numbers to character names) have
been spun off into separately-maintained libraries, \emph{twice}, with an
internal political dispute over which of the two spun-off libraries to
actually use, and some copyright licensing implications.

Meanwhile, in IDSgrep, the character width table descends from
\texttt{EastAsianWidth.txt}.  The data is stored as a transition table for a
finite state machine, with a fragment of C code that traverses the table for
a character input until it can determine the width.  There is a separate
program that reads the Unicode Consortium's data file and uses binary
decision diagram techniques to generate an optimized transition table.  This
system at least is documented, and in the event of changes to the
Unicode data or differing preferences on how to resolve ambiguities, it can
be maintained.  However, it is highly specific to the particular function it
calculates.  It would require significant rewriting to work with anything
else.  It also makes use of certain secrets of black magic about which man
was not meant to know.  IDSgrep also embeds the Unicode blocks list, in
a handwritten C array, with documentation but no automated support for
changing it.

Icemap is intended to replace all the static maps in the Tsukurimashou
project, including FontAnvil's encodings, parser tables, and character
names, IDSgrep's width and Unicode block tables, and any future static maps
needed by Tsukurimashou C programs, with code from a single automated
generator.  Automation should go all the way back to the original data
sources.  It should be possible to just download a new
\texttt{EastAsianWidth.txt} or other original third-party data file, run
\texttt{make} with an appropriate target, and have the build system update
all the code without further human attention.  Optimizations like the
cascading lookups of PfaEdit or the minimized finite state machine of
IDSgrep should be available transparently, in all maps where they are
useful, without requiring human labour to implement them every time.

The function of Icemap can be seen as comparable to that of \texttt{lex},
\texttt{yacc}, or \texttt{gperf}.  It is intended to run as part of the
build process for some other software package, but even parties who are
compiling and modifying the other software will not necessarily need to use
Icemap directly; it is upstream in the overall build process and will not
run during an ordinary build.

Suppose Ulrich is a third-party publisher of
static map data (such as character code tables) and Alice is the original
author of Fizzbuzz, a software package which needs to embed this data.  Then:

\begin{itemize}
  \item Alice writes a control file for Icemap describing how to find
    key-value pairs in Ulrich's distributed file format, and runs Icemap with
    the control file and Ulrich's data file, to generate C code implementing
    the static map.  She puts the generated C code into the Fizzbuzz tarball
    and distributes it, like the rest of the Fizzbuzz C source code. 
    The Icemap control file also goes into the tarball for use by people
    like Dave, below, but will not be used in ordinary compilation.
  \item Bob is building a copy of Fizzbuzz from source.  \emph{He
    does not need to use Icemap.}  He just compiles the C code like any
    other C program.
  \item Carol is using a packaged binary, possibly built by Bob.  She does
    not touch Icemap nor the C code at all.
  \item Dave wants to update the static map with new data from Ulrich.  He
    needs to run Icemap with the new data file, but (assuming Ulrich's file
    format has not changed), he can still use Alice's control file, which
    tells Icemap how to read the data.
  \item Ellen is modifying Fizzbuzz in a more elaborate way:  for instance
    to use static map data from Valerie instead of Ulrich, supplied in a
    different file format.  Ellen needs to modify or rewrite the control
    file and re-run Icemap.
\end{itemize}

Things notably absent from these stories:  nobody needs to reverse-engineer
the lookup code, write new lookup code, nor manually convert Ulrich's file
format into C.

%%%%%%%%%%%%%%%%%%%%%%%%%%%%%%%%%%%%%%%%%%%%%%%%%%%%%%%%%%%%%%%%%%%%%%%%

\chapter{Icemap language concepts}

%%%%%%%%%%%%%%%%%%%%%%%%%%%%%%%%%%%%%%%%%%%%%%%%%%%%%%%%%%%%%%%%%%%%%%%%

\chapter{Examples}

%%%%%%%%%%%%%%%%%%%%%%%%%%%%%%%%%%%%%%%%%%%%%%%%%%%%%%%%%%%%%%%%%%%%%%%%

\chapter{Reference}

%%%%%%%%%%%%%%%%%%%%%%%%%%%%%%%%%%%%%%%%%%%%%%%%%%%%%%%%%%%%%%%%%%%%%%%%
%%%%%%%%%%%%%%%%%%%%%%%%%%%%%%%%%%%%%%%%%%%%%%%%%%%%%%%%%%%%%%%%%%%%%%%%
%%%%%%%%%%%%%%%%%%%%%%%%%%%%%%%%%%%%%%%%%%%%%%%%%%%%%%%%%%%%%%%%%%%%%%%%

\end{document}
