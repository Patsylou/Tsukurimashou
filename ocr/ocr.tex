\documentclass{article}

\usepackage{fontspec}
\usepackage[margin=1.25in,top=0.85in]{geometry}
\usepackage{wrapfig}
\usepackage{xltxtra}

\usepackage{hyperref}

\defaultfontfeatures{WordSpace={1,0,0},PunctuationSpace={3}}

\setlength{\parindent}{0pt}
\setlength{\parskip}{\baselineskip}
\raggedright

\title{OCR-A and OCR-B fonts\\version 0.2}
\author{Matthew Skala}
\date{September 27, 2012}

\begin{document}
\setmainfont{OCRB.otf}
\setmonofont{OCRB.otf}

\maketitle

\section{Introduction}

In the mid-2000s I had a reason to need the fonts "OCR A" and "OCR B," and I
found it much more difficult than it should have been to obtain copies and
get them working with my software.  There are commercial versions on the
market; but these fonts, being defined in and required by public standards
documents, really ought to be available in the free world.  There were free
versions available but only in the semi-obsolete MetaFont format, designed
to produce bitmap output for use with TeX\@.  I ended up getting a bunch of
conversion and tracing software and semi-manually converting the fonts to
TrueType format.  The results were just barely good enough for my own
purposes; but I posted them on my Web site anyway just in case they might be
of use to others.

It turns out that in the years since then, those fonts have been among the
most-requested resources on my Web site.  Dozens if not hundreds of people
have downloaded them and used them - and suffered the consequences of the
crummy conversion job, and complained about the bad encoding, and so on.  In
the time since then I've also learned a lot more about how fonts work, and
have designed a few of my own.  I'm no longer thrilled to have my name on
the old, poor-quality converted OCR fonts; and with my improved knowledge,
I'm now in a position to write better ones.

This package contains those better fonts.  More work still needs to be done,
but it's already a significant improvement on the older packages.

Each font is provided in three ready-made forms:  PostScript (.pfb and .afm
files), TrueType (.ttf files), and OpenType (.otf files).  Almost any
computer typesetting or word processing system should be able to read at
least one of these formats.  OpenType is probably best if you have a choice.

Please note that by definition these fonts have limited glyph sets - and
if you actually want to use these fonts for OCR applications (their original
purpose) then you should be very careful about which glyphs you use, because
both typefaces have been extended by others (not me) to contain nonstandard
glyphs beyond the official ones.  Similarly, this package contains some
nonstandard styles for OCR B (italic, reverse video, and outline) which
may be visually appealing but are probably not appropriate for actual
OCR use.

Please note that although the Metafont definitions for both OCR A and OCR B
purport to support "optical size," they actually just scale the outlines
linearly for the different sizes, so when converting to a scaleable format
there's no point treating the different sizes separately.

My email address is \url{mailto:mskala@ansuz.sooke.bc.ca}~.  As of this
version, this package has become part of the Tsukurimashou Project at
\url{http://tsukurimashou.osdn.jp/}~.  That is a bilingual page,
English and Japanese; you can select the other one in the upper right corner
if your browser's language preferences are misconfigured.  All bug reports
and feature and support requests for these OCR fonts should be filed in the
Tsukurimashou Project's ticket tracker, with the component set to "Parasite
font packages."

\setmainfont{OCRA.otf}
\setmonofont{OCRA.otf}
\section{OCR A}

Text in this section of this document is set in OCR A; the other sections
are OCR B.

OCR A is the standard font for the human-readable ISBN printed above the bar
code on most books.  It has an old-fashioned computerish feel, evoking the
mythos of "big iron" data processing for which it was originally designed.

The version in this package originates with ANSI Standard X3.17-1977,
approved January 20, 1977.  Tor Lillqvist of the Technical Research Centre
of Finland created a Metafont definition and added some semi-official
characters for writing Nordic languages, based on an appendix of the
standard.  There is a "bang path" email address (no longer routable on the
vast majority of Internet email systems) for Lillqvist in the source code
comments, but little other information about this stage of the font's
development is available to me.  Richard B. Wales of UCLA picked up the
project in 1988 and his copyright assertion and notes are in the code
comments as well.  It is not clear to me how much of the original content of
the font actually belongs to Wales.  In his copyright notice he states that
the font may be used freely, but cannot be distributed for profit (see the
notices at the start of msk-ocra.mp for details).
I released an earlier version in 2006
but that is now obsolete; the present version is newly derived in 2011 from
the Wales version.  I make no copyright claim on it myself.

The Metafont package I worked from is available at
\url{http://www.ctan.org/tex-archive/fonts/ocr-a}~.

This font includes alternate versions for some characters.  The alternates
are available in the OpenType version via the "stylistic set" and "all
alternates" features.  In each case I have let the default versions of the
characters be the ones that were default in the Wales version; in some cases
those are actually the more recently revised versions of the glyphs.  See
the source code or experiment with the OpenType features for more
information.  The alternates should also be available through the TrueType
and PostScript versions, but I can't comment on exactly how to access them.

\setmainfont{OCRB.otf}
\setmonofont{OCRB.otf}
\section{OCR B}

OCR B is the standard font for the human-readable number printed along the
bottom edge of a UPC/EAN bar code.  That means that a standard book, if it's
really using the correct fonts, needs both OCR A and OCR B in its bar code
block.  Many UPC codes do not actually use OCR B, however; often they fudge
it with Courier, Helvetica, or even Arial.  Unlike the ISBN (which is
human-readable but meant to also be machine-readable), the number
along the bottom of the bar code is only for human beings, and a computer
would only scan the bars themselves.

The version in this package descends from a set of Metafont definitions by
Norbert Schwarz of Ruhr-Universitaet Bochum, bearing dates ranging from 1986
to 2010.  He originally distributed it under a "non-commercial use only"
restriction but has since released it for unrestricted use and distribution. 
See the README file in Norbert Schwarz's package, available from the
URL below, for more details.

I frequently receive email asking me to ``confirm'' or ``clarify'' the
copyright status of the OCR B fonts.  Usually, the authors of such messages
give no indication that they have so much as read this documentation file. 
I make no copyright claim on these fonts myself.  I do not believe anyone
else makes a copyright claim that would conflict with the use of these fonts
in a commercial project, but I cannot speak for any party other than myself. 
I cannot give any kind of binding legal opinion on the status of these
fonts.  This documentation file already explains their origin clearly, and
contains pointers to other relevant documents.  If you require authoritative
legal advice about your proposed use of these fonts, then you must retain a
lawyer.  I will no longer answer email queries on this subject.

The Metafont definitions include a number of variants for things like
"sharp ends" and "reverse video." It's not clear how valuable those
alternate fonts are; they aren't suitable for actual OCR use, and there are
some problems in the outlines (for instance, with poor overlapping of
sharp-ended strokes) that make them less than optimal for human use too. 
Making the alternate versions work with MetaType1 is going to require a fair
bit of effort working around bugs in MetaType1 and Fontforge, as well as
correcting the problems in the originals; I plan to do that eventually, but
it's not done in this version.

Just as with the OCR A font, there are alternate glyphs for a
few characters.  In the OpenType version, these are available through the
"stylistic set" and "all alternates" features.

The current version of the Schwarz package is available from CTAN at this
address:  \url{http://www.ctan.org/tex-archive/fonts/ocr-b}

There is also a package by Zdenˇ\hspace{-0.1in}ek Wagner that is similar
in general nature to this one:
\url{http://www.ctan.org/tex-archive/fonts/ocr-b-outline}

Like my earlier package, Wagner's was derived by tracing the outlines from
the Metafont originals in a semi-automated way.  The fonts in my current
package are probably cleaner.

\section{Compiling the fonts}

Note that the binary font files are included in the package.  Most users
will have no reason to recompile them, and can safely ignore this section.

As of this version, this package builds using a stripped-down version of
MetaType1 with some bugs fixed, inherited from the Tsukurimashou Project. 
The relevant code is bundled with this package, and (where applicable)
relicensed to public domain.  MetaType1 is no longer a dependency; Perl,
Metapost (which should be included in a standard TeX distribution) and t1asm
(part of the t1utils package) are now dependencies.  FontForge is a
dependency.  Recompiling this document will also require XeLaTeX (which
should be included in a standard TeX distribution); and there are additional
considerations relevant to the test suite, for which see the "Testing the
fonts" section below.  Having "expect" is recommended but not required.

This package includes a standard GNU Autotools build system; if you have the
prerequisites, it should work by running "./configure" followed by "make". 
If you turn off the default feature that hides them, the compilation process
will produce a large number of error messages; most of these are associated
with bugs in Fontforge's spline geometry code, and are unavoidable.

The build system supports a "make install" target; however, you might not
want to use it, because it will install all the different styles and
formats of the fonts and quite possibly install them in places other than
where you expect.  Most users will probably only want one format and a
limited selection of styles, and would be better served by manually copying
the files they want after building.  Note that all the finished fonts
intended for installation and use have filenames starting with capitalized
"OCR"; there are intermediate Postscript files created during the build
under names that start with lowercase "msk-ocr," but those are lacking
important metadata and should not be installed and used directly.

Most of the Metapost source code files in this package
have had their names changed by prefixing "msk-"; that is to prevent a
collision with the filenames used by the original TeX packages.  If, like
me, you try to compile these fonts on a system that also has the original
TeX packages installed, there would otherwise be a danger of getting the
original MetaFont files mixed with these Metapost files in a way that would
cause it to fail.  I couldn't figure out how to force Metapost to really
use a specified pathname instead of going through TeX's filename search; it
appears to strip off all specified path information.

\section{Testing the fonts}

The build system supports a "make check" target, which will run FontForge's
"fontlint" program on all the installable fonts.  This is a very demanding
test.  It will report a failure on anything that FontForge's developers (and
even though I have sometimes been credited as one of these, I do not take
responsibility for this point) think is against the rules or even vaguely
questionable.  Most fontlint validation errors are harmless in actual
practice; so if you run "make check" and see nothing but red, Don't Panic.

At the very least, "make check" will almost certainly fail if your fonts
were built with a version of FontForge that did not support my proposed
optional argument to AddExtrema(); and as of this writing the only version
of FontForge that supports that is the one in my Github fork at
\url{https://github.com/mskala/fontforge}~.  The issue, for those
interested, is that FontForge's "add extrema" operation has several
different operating modes, including one that adds all possible extrema and
one that only adds them in cases where it's considered "safe" to do so.  The
scripting language by default can only invoke the "safe" version; but
fontlint demands all extrema whether safe or not, so without a patched
version that makes the other modes available to the scripting language, it's
not possible for scripts to generate fonts that can pass fontlint.

Please do not report validation errors as bugs if you are not using the
version of FontForge from my Github fork.

\end{document}
