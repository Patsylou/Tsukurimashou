\documentclass[14pt]{extarticle}

%
% User Manual for Tsukurimashou
% Copyright (C) 2011, 2012, 2013  Matthew Skala
%
% This program is free software: you can redistribute it and/or modify
% it under the terms of the GNU General Public License as published by
% the Free Software Foundation, version 3.
%
% As a special exception, if you create a document which uses this font, and
% embed this font or unaltered portions of this font into the document, this
% font does not by itself cause the resulting document to be covered by the
% GNU General Public License. This exception does not however invalidate any
% other reasons why the document might be covered by the GNU General Public
% License. If you modify this font, you may extend this exception to your
% version of the font, but you are not obligated to do so. If you do not
% wish to do so, delete this exception statement from your version.
%
% This program is distributed in the hope that it will be useful,
% but WITHOUT ANY WARRANTY; without even the implied warranty of
% MERCHANTABILITY or FITNESS FOR A PARTICULAR PURPOSE.  See the
% GNU General Public License for more details.
%
% You should have received a copy of the GNU General Public License
% along with this program.  If not, see <http://www.gnu.org/licenses/>.
%
% Matthew Skala
% http://ansuz.sooke.bc.ca/
% mskala@ansuz.sooke.bc.ca
%

\input{version.tex}

\usepackage{fontspec}
\usepackage[margin=1.25in,top=0.85in]{geometry}
\usepackage{tikz}
\usepackage{tocloft}
\usepackage{wrapfig}
\usepackage{xcolor}
\usepackage{xltxtra}

% colors
\definecolor{darkgreen}{rgb}{0,0.35,0}
\definecolor{purplish}{rgb}{0.4,0,0.6}

\usepackage[letterpaper,breaklinks,bookmarks,plainpages=false,
   colorlinks,citecolor=darkgreen,linkcolor=purplish]{hyperref}

\defaultfontfeatures{Mapping=tex-text,Path=../otf/}

\usetikzlibrary{arrows,shapes.misc}

\setlength{\parindent}{0pt}
\setlength{\parskip}{\baselineskip}

%%%%%%%%%%%%%%%%%%%%%%%%%%%%%%%%%%%%%%%%%%%%%%%%%%%%%%%%%%%%%%%%%%%%%%%%

\newfontface\maru{TsukurimashouMaruPS}
\newfontface\mincho{TsukurimashouMinchoPS}
\newfontface\anbiruteki{TsukurimashouAnbirutekiPS}
\newfontface\tenshinokami{TsukurimashouTenshinoKamiPS}
\newfontface\bokukko{TsukurimashouBokukkoPS}

\expandafter\ifx\csname haveJieubsidaDodumPS\endcsname\relax\else
  \newfontface\dodum[RawFeature={+ccmp,+ljmo,+vjmo,+liga}]{JieubsidaDodumPS}
\fi

\expandafter\ifx\csname haveTsuItaSokuPS\endcsname\relax
  \newfontfamily\kaku[ItalicFont={TsukurimashouAnbirutekiPS}]{TsukurimashouKakuPS}
  \kaku
  \setmainfont[ItalicFont={TsukurimashouAnbirutekiPS}]{TsukurimashouKakuPS}
\else
  \newfontfamily\kaku[ItalicFont={TsuItaSokuPS}]{TsukurimashouKakuPS}
  \kaku
  \setmainfont[ItalicFont={TsuItaSokuPS}]{TsukurimashouKakuPS}
\fi

\setmonofont[WordSpace={1,0,0},PunctuationSpace=3]{TsukurimashouMincho}

\renewcommand{\labelitemi}{{\fontspec[RawFeature=+ornm]{TsukurimashouKaku}C}}
\newcommand{\fracfont}[1]{{\fontspec[RawFeature=+afrc]{TsukurimashouKaku}#1}}
\newcommand{\enclfont}[1]{{\fontspec[RawFeature=+ss02]{TsukurimashouKakuPS}#1}}

% section and subsection names in this document are bilingual; define
% special commands to format them nicely.
\newcommand{\blsection}[2]{%
\kaku\clearpage\Large\phantomsection%
\addcontentsline{toc}{section}{#1 #2}%
#1\\ #2\par\addpenalty{-300}\normalsize}
\newcommand{\blsubsection}[2]{%
\kaku\large#1\qquad\phantomsection%
\addcontentsline{toc}{subsection}{#1 #2}%
#2\par\addpenalty{-300}\normalsize}

%%%%%%%%%%%%%%%%%%%%%%%%%%%%%%%%%%%%%%%%%%%%%%%%%%%%%%%%%%%%%%%%%%%%%%%%

\begin{document}
\pagestyle{plain}\thispagestyle{empty}

%%%%%%%%%%%%%%%%%%%%%%%%%%%%%%%%%%%%%%%%%%%%%%%%%%%%%%%%%%%%%%%%%%%%%%%%
%%%%%%%%%%%%%%%%%%%%%%%%%%%%%%%%%%%%%%%%%%%%%%%%%%%%%%%%%%%%%%%%%%%%%%%%

\kaku
\begin{center}\LARGE

\vspace*{\fill}

\phantomsection\belowpdfbookmark{とびら Title Page}{bkm:title}

{\Huge 作りましょう~\TsukurimashouVWide}\\
{\huge パラメタ方式フォントファミリ\\
ユーザマニュアル}

\vspace*{0.75in}

{\Huge Tsukurimashou~\TsukurimashouVersion}\\
{\huge Parametric Font Family\\
User Manual}

\vspace*{1.5in}

Matthew Skala\\
mskala@ansuz.sooke.bc.ca\\
\TsukurimashouRDWide\qquad\TsukurimashouReleaseDate

\vspace*{\fill}

\end{center}
\clearpage

%%%%%%%%%%%%%%%%%%%%%%%%%%%%%%%%%%%%%%%%%%%%%%%%%%%%%%%%%%%%%%%%%%%%%%%%
%%%%%%%%%%%%%%%%%%%%%%%%%%%%%%%%%%%%%%%%%%%%%%%%%%%%%%%%%%%%%%%%%%%%%%%%

\vspace*{\fill}

\phantomsection\belowpdfbookmark{コピーライト Copyright}{bkm:copyright}

This project's English-language home page is at\\
\hspace*{1em}\url{http://tsukurimashou.sourceforge.jp/index.php.en}.\\
このプロジェクトは、日本語のページが\\
\hspace*{1em}\url{http://tsukurimashou.sourceforge.jp/index.php.ja}です。

\vspace*{1in}

User manual for Tsukurimashou\\
Copyright © 2011, 2012, 2013\quad Matthew Skala

This program is free software: you can redistribute it and/or modify
it under the terms of the GNU General Public License as published by
the Free Software Foundation, version 3.

As a special exception, if you create a document which uses this font, and
embed this font or unaltered portions of this font into the document, this
font does not by itself cause the resulting document to be covered by the
GNU General Public License. This exception does not however invalidate any
other reasons why the document might be covered by the GNU General Public
License. If you modify this font, you may extend this exception to your
version of the font, but you are not obligated to do so. If you do not
wish to do so, delete this exception statement from your version.

This program is distributed in the hope that it will be useful,
but WITHOUT ANY WARRANTY; without even the implied warranty of
MERCHANTABILITY or FITNESS FOR A PARTICULAR PURPOSE.  See the
GNU General Public License for more details.

You should have received a copy of the GNU General Public License
along with this program.  If not, see \url{http://www.gnu.org/licenses/}.

\clearpage

%%%%%%%%%%%%%%%%%%%%%%%%%%%%%%%%%%%%%%%%%%%%%%%%%%%%%%%%%%%%%%%%%%%%%%%%
%%%%%%%%%%%%%%%%%%%%%%%%%%%%%%%%%%%%%%%%%%%%%%%%%%%%%%%%%%%%%%%%%%%%%%%%

\phantomsection\belowpdfbookmark{目次 Contents}{bkm:contents}
\renewcommand\contentsname{目次 Contents}
\renewcommand{\cftbeforesubsecskip}{0pt}

\tableofcontents

\clearpage

%%%%%%%%%%%%%%%%%%%%%%%%%%%%%%%%%%%%%%%%%%%%%%%%%%%%%%%%%%%%%%%%%%%%%%%%
%%%%%%%%%%%%%%%%%%%%%%%%%%%%%%%%%%%%%%%%%%%%%%%%%%%%%%%%%%%%%%%%%%%%%%%%

\blsection{イントロ}{Introduction}

日本ごのユーザマニュアルをまちうけば、ごめんなさい。~%
みらいは、日本ごのユーザマニュア
ルをかきます。

I want to learn Japanese.  That's a large project, likely to involve years
of daily study to memorize a few tens of thousands of words.  It's clearly
within the range of human instrumentality, because many millions of Japanese
people have done it; but most of them started in infancy, and it's widely
believed that people's brains change during childhood in such a way that
it's much harder to learn languages if you start as an adult.  I also would
like to be fully literate within somewhat less than the 15 years or so that
it takes a native learner to gain that skill.

What about designing a Japanese-language typeface family?  Typeface design
is a difficult activity, requiring hours of work by an expert to design each
glyph.  A typeface for English requires maybe 200 glyphs or so; one for
Japanese requires thousands.  That makes the English-language typeface about
a year's full-time work, and the Japanese-language typeface enough work (20
to 25 man-years) that in practice it's rare for such a project to be
completed by just one person working alone, at all.  There is also the minor
detail that I said a ``typeface family''---one of those typically consists
of five or six individual faces, so the time estimate increases to between
100 and 150 years.

However, I've already decided to spend the time on the language-learning
project.  And I'm sure that if I design a glyph for a character, I'm going
to remember that character a lot better than if I just see it a few times on
a flash card.  And as a highly proficient user of computer automation, I
have access to a number of efficiency-increasing techniques that most font
designers don't.  So the deal is, I think if I study the language
\emph{and also design a typeface family for it,} I might be able to
finish both big projects with less, or not much more, effort than just
completing the one big project of learning the language alone.  And then I'd
also end up with a custom-made Japanese-language typeface family, which
would be a neat thing to have.  And so, here I am, building one.

Because this is a parametrized design built with Metafont, users can
generate a potentially unlimited number of typeface designs from the source
code; but I've provided parameters for several ready-made faces, each of
which can be drawn in monospace or proportional forms.  The mainline
Tsukurimashou fonts are intended for Japanese, but version 0.5 also
introduced a series of fonts called Jieubsida, designed for Korean
hangul.  There is also some experimental code intended to eventually become
a set of blackletter fonts, but it is disabled by default and not currently
usable.

The main goal of this project is for me to learn the kanji by designing
glyphs for them, and so far it does seems to be helping my learning process,
though the nature of computer programming is certainly having some unusual
disruptive effects.  For instance, in some cases I appear to have
unintentionally memorized the Unicode code points of kanji without learning
their meanings or pronunciations.  I may end up learning to speak Japanese
just like a robot; but by the time I'm fluent, the Japanese demographic
collapse will be in full swing and they'll have replaced much of their
population with humanoid interfaces anyway, so maybe I'll fit right in.

Please understand that the finished product is not the point so much as the
process of creation, hence the name.  Furthermore, although the fonts cover
the MES-1 subset of Unicode, and thus can in principle be used for almost
all popular languages that use the Latin script, and I know that at this
stage most users are probably more likely to use the fonts for English than
for anything else, nonetheless these fonts are intended for eventual primary
use with the Japanese language.  Some decisions on the Latin characters were
driven by that consideration---in particular, the simplistic serif design
and weighting in the Latin characters of Tsukurimashou Mincho, and the
limited customization of things like diacritical mark positioning in Latin
characters not used by English.  I biased some marks (notably ogonek and
ha\v{c}ek) toward the styles appropriate for Czech and Polish as a nod to my
own ancestors, but I cannot read those languages myself, and I cannot claim
that these fonts will really look right for them.  I'm not interested in
spending a lot of time tweaking the Latin to be perfect because it's not
really the point. I already know how to read and write English.

Some other notes:

\begin{itemize}
  \item Since my learning the kanji is a big part of the goal of this
  project, ``labour-saving'' approaches that would relieve me of having to
  look at all the kanji individually myself (for instance, by feeding a
  pre-existing database of kanji shapes into my existing general font
  technology) are not appropriate to the original intention of the project. 
  Nonetheless, as of Fall 2012 I have become involved in a proposed research
  project to more or less exactly feed a pre-existing database of kanji
  shapes into my existing general font technology.  Most likely,
  Tsukurimashou itself will continue to be manually maintained by me alone
  in support of my own language learning, and some future version of this
  document will also contain information about the database-driven system,
  which will also be free, will involve other maintainers besides me, and
  will share technology with Tsukurimashou.

  \item Proportional spacing and kerning still require some work.  Be aware
  that future versions will change the spacing of some characters, so if you
  are one of those people to whom any changes in line breaking are anathema,
  you should not expect to be able to upgrade the proportionally spaced
  versions of these fonts in archived documents.  The monospace
  versions have more, but still not complete, long-term stability.

  \item I would like to include at least some support for vertical script,
  but it is not a high priority.  One obstacle is that I don't have access
  to competent vertical typesetting software, whether the font could support
  it or not.

  \item Tsukurimashou is designed primarily for typesetting Japanese,
  secondarily for English.  I have no immediate plans to support
  other Han-script languages (such as any dialect of Chinese) nor put a
  lot of effort into tweaking the fine details of characters only intended
  for use in occasional foreign words.

  \item Support of Korean is limited because of my limited knowledge
  of that language; and learning it is not a high priority for me.  At
  this point the Jieubsida fonts only support Korean hangul, not hanja
  (which are the Korean equivalent of kanji, but just different enough
  that copying over the Tsukurimashou kanji would not be good enough).

  \item I reserve the right to add features that I think are fun, even if
  they are not useful.

  \item Tsukurimashou is designed as a vector font, assuming an output
  device with sufficient resolution to reproduce it.  In practice, that
  probably means a high-quality laser printer.  I have not spent time
  optimizing it for screens or low-resolution printers, and the hinting is
  automated.

  \item If it turns out to be too much work after all, I might abandon the
  whole project.

  \item Both building and use of Tsukurimashou require working around many
  bugs in third-party packages, some of which were mentioned here in earlier
  versions of this document.  The list has now grown so long it needs its
  own subsection, which starts on page~\pageref{sub:bugs} of this document.
\end{itemize}

The Tsukurimashou fonts are distributed under the GNU General Public
License, version 3, with an added paragraph clarifying that they may be
embedded in documents.  See the files COPYING and COPYING.GPL3, and
note the following addition:

\begin{quotation}
As a special exception, if you create a document which uses this font, and
embed this font or unaltered portions of this font into the document, this
font does not by itself cause the resulting document to be covered by the
GNU General Public License. This exception does not however invalidate any
other reasons why the document might be covered by the GNU General Public
License. If you modify this font, you may extend this exception to your
version of the font, but you are not obligated to do so. If you do not wish
to do so, delete this exception statement from your version.
\end{quotation}

The license means (and this paragraph is a general summary, not overriding
the binding terms of the license) that you may use the fonts at no charge;
you may modify them; you may distribute them with or without modifications;
but if you distribute them in binary form, you must make the source code
available.  Furthermore (this is where font-embedding becomes relevant)
embedding the font, for instance in a PDF file, does not in itself
trigger the source-distribution requirement.

My plan is that at some point in the future, when the fonts are in a more
useful and complete form, I will make precompiled binaries available through
commercial online channels.  That will serve several purposes: it will allow
me to make some money from my work, and it will also probably encourage some
people to use the fonts who wouldn't otherwise.  One of the bizarre aspects
of human behaviour is that some people will buy a product they would not
accept for free.  Okay, whatever; in such a case I'm happy to take the money
for it.  Having a pay option will also give anybody who wants to support my
efforts, an easy way to do that.  For now, though, I am distributing
Tsukurimashou only as this source package, with precompiled versions
included for the Kaku and Mincho styles.  The Jieubsida fonts, which don't
need the in-progress kanji glyphs, may also be available in precompiled form
as a separate package.  If you want other styles, you'll have to compile
them yourself or get them from someone who has done so.  This limitation is
deliberate: with the fonts in their current partial form, I'd rather limit
their circulation to hobbyists.

I maintain several other free font projects, which as of version 0.7 have
largely been merged into the Tsukurimashou packaging and build system
because they share code.  These ``parasite'' packages appear as
subdirectories of the Tsukurimashou source distribution, and may also be
available as separate packages.  See the discussion in
Subsection~\ref{sub:parasites} for more information about parasite packages.

This documentation file gives some notes on the build system and on how
to use the OpenType features built into the fonts.  Other documentation
files included in the package demonstrate what the fonts look like and
list the current kanji coverage.  Better documentation (and some day,
Japanese-language documentation) will probably appear in a later
version; at the moment, I'm just more interested in designing fonts
than in writing about them.  Of course, all the typesetting in this
manual is done with fonts from this package.

The name ``Tsukurimashou'' could be translated as ``Let's make something!''

From time to time, people ask how they can help with the project.  I'm
hesitant to accept contributions to the coding, because of the pedagogical
goal: I need to do it myself in order to learn by doing it.  I also don't
have much need for monetary donations.  If you are in a position to actually
offer me full-time employment appropriate to my skills and experience, I
might like to hear from you, but as far as the Tsukurimashou project is
concerned, the one thing that would really help a lot would be publicity. 
Share the link on social networks; write about it on your Web log (or invite
me to write a guest posting); or even just do a ``review'' or a ``rating''
on the Sourceforge.JP project page.  Tsukurimashou is also registered on
Github, Ohloh, and CIA.vc; if you use one of those systems, you're
encouraged to ``follow'' or ``subscribe'' to it as appropriate, both to keep
yourself updated and to raise the project's profile.

There is some possibility that I may actually be able to get funding in the
future to work on kanji fonts and dictionaries full-time for a while. 
Exactly what that would mean for Tsukurimashou is unknown, because the
project under discussion would have goals similar to but not exactly the
same as the current Tsukurimashou/IDSgrep, and it's too soon to announce
anything.

The home pages for this project, where you can download the latest releases,
browse the source-control repository, and so on, are:\\
\hspace*{1em}\url{http://tsukurimashou.sourceforge.jp/index.php.en}%
\quad (English)\\
\hspace*{1em}\url{http://tsukurimashou.sourceforge.jp/index.php.ja}%
\quad(日本語)

よろしくおねがいします。

Matthew Skala\\mskala@ansuz.sooke.bc.ca\\\TsukurimashouReleaseDate

%%%%%%%%%%%%%%%%%%%%%%%%%%%%%%%%%%%%%%%%%%%%%%%%%%%%%%%%%%%%%%%%%%%%%%%%

\blsubsection{\TsukurimashouVWide
のニュース}{What's new in \TsukurimashouVersion?}

Version 0.8 covers 1502 kanji, including all those taught through Grade Four
in the Japanese school system.  This release includes relatively few major
changes to infrastructure.  The new kanji are the main new content in this
version.  These (100 scheduled by the roadmap, roughly another hundred
spin-offs as a result of building kanji that are parts of others, and so on)
turned out to be disproportionately hard.  It seems like Grade Four is when
the Japanese school system takes the gloves off, so to speak, and expects
children to learn kanji that are much less well-behaved than before in terms
of being made of simple parts in simple ways.  Many of the new kanji in this
release contain weird stroke structures that do \emph{not} occur widely in
other glyphs.  As a result, they required a fair bit more new code, and this
new code is likely to be less reusable, than the kanji in earlier releases. 
I'm hoping that some of the kanji in the next few versions will be easier to
handle, but that remains to be seen.

Here are some other things of note in the new version:

\begin{itemize}
\item Kleknev, a build system profiler, added as a parasite package.  This
actually originated in some last-minute issues that came up while packaging
the previous version, 0.7.  It remains somewhat undocumented and
experimental, but there is at least a man page.

\item The expect script used for invoking \XeLaTeX\ and a few other things
now waits for child processes to avoid creating zombies.  That it didn't,
before, was a bug discovered during Kleknev development.

\item IDSgrep 0.4, released a few days before this version of Tsukurimashou,
now incorporates experimental bit vector indices.  This is actually a pretty
big deal; it is cutting-edge computer science research from which I hope to
gain many prestigious academic publications.  From a user's point of view,
though, the practical consequence is simply that IDSgrep should now run a
lot faster.  See the IDSgrep documentation for more about the bit vector
indices.

\item Shortly after the release of Tsukurimashou~0.8, I will be giving a
presentation about the project at TUG~2013, the \TeX\ Users Group annual
meeting, in Tokyo.  That should raise the profile of the project a bit.

\item Since May 2013, I am unemployed and looking for work.  That may have
consequences eventually.  The fact that I am still single really worries me
more, but both are important.
\end{itemize}

%%%%%%%%%%%%%%%%%%%%%%%%%%%%%%%%%%%%%%%%%%%%%%%%%%%%%%%%%%%%%%%%%%%%%%%%

\blsubsection{外のプロジェクト}{Other similar projects}

Maybe you shouldn't use this package!  It is designed for specific purposes
that are relevant to its designer, and although I certainly hope others will
find it useful, my goals may or may not be in line with yours.  Also,
although I sometimes describe Tsukurimashou as the first parameterized
METAFONT family with Japanese-language coverage, that claim requires careful
qualification because many projects with similar aims have existed in some
form for a long time.  Here are some others, going back a few decades, that
you might want to check out.

This is not intended to be a complete list; in particular, I'm leaving out
many sources of CJK fonts that are not METAFONT-related, and many academic
papers that are not associated with publically-available fonts.  There is
also no doubt a great deal of research and development locked up inside
commercial organizations, or published in the Chinese language and thus
inaccessible to me.

\begin{itemize}

\item ``LCCD, A Language for Chinese Character Design,'' Tung Yun Mei,
Stanford technical report STAN-CS-80-824.  He built a language similar in
nature to METAFONT (collaborating with Knuth and sharing ideas with the
earliest versions of METAFONT) and constructed 112 kanji for use in TAOCP. 
It's interesting how little has changed since this early work.

\item ``A Chinese Meta-Font,'' John Hobby and Gu Guoan, paper in TUGboat
5--2, 1984.  Proof of concept and discussion of some of the graphic design
issues for parameterized CJK fonts.  They built 140 radicals and 128
characters, using infrastructure very similar to Knuth's techniques for
Latin fonts, and the traditional Chinese stroke-based analysis of
characters.  High-quality parameterized designs.  No apparent plan to
actually turn these into a usable full-coverage family; it seems to have
been meant as research into the techniques only. 
\url{http://www.tug.org/TUGboat/tb05-2/tb10hobby.pdf}  John Hobby has posted
the source code described in the paper on his Web site at
\url{http://ect.bell-labs.com/who/hobby/hobbygu.tar.gz}, but as he says in
the enclosing Web page, the files are written in the now-obsolete
METAFONT79 language and ``they are of limited use because they are not
compatible with today's METAFONT.''

\item p\TeX, ASCII Corporation, 1987 onward: not a font project, but a \TeX\
engine modified to handle 16-bit character codes and using existing fonts
from other systems.  Very popular in Japan; to some extent it still is,
though other projects of similar nature (mostly not listed here) have
gained a lot of market share in recent years.

\item The Quixote Oriental Fonts Project, spearheaded by Dan Hosek,
announced in a paper at the TUG 1989 Conference.  Intended to be a
parameterized METAFONT-native family for Chinese, Japanese, and Korean.
Hosek apparently had some source code in hardcopy form displayed at the
conference, but I've not been able to find the code nor any subsequent
discussion of the project.
\url{http://www.tug.org/TUGboat/tb10-4/tb26hosek.pdf}

\item Poor Man's Chinese and Poor Man's Japanese, 1990, Tom Ridgeway:
technology for displaying 24×24 bitmap fonts through METAFONT.  This
was not curve tracing, nor smooth scaling, but a way to actually display
the dot matrix, jaggies and all.  Still available in CTAN package
``\texttt{poorman},'' but considered obsolete.

\item Jem\TeX, 1991, François Jalbert: included a program called
\texttt{jis2mf} which would auto-trace 24×24 bitmaps to produce
non-parameterized METAFONT code.  Many sources from the 1990s
(for instance, a regular Usenet posting
aiming to list all then-available \texttt{.mf}-format fonts) describe
the availability of ``Metafont for 61 Japanese fonts,'' which is the output
from this program.

\item The CJK package, described in TUGboat at least as early as 1997, still
available though no longer popular, Werner Lemberg.  Not a font project but
a system for typesetting CJK text in \LaTeX\ under the standard 8-bit
engines, getting around the encoding issues by splitting each font into many
smaller virtual fonts (similar to Tsukurimashou's ``page'' system).  Fonts
for this, at least at the outset, were usually bitmap fonts imported from
other systems (one popular one was 48×48); later, as free vector fonts
became available, those started to be used, some of them via auto-conversion
from formats like TrueType to non-parameterized METAFONT.

\item HanGlyph, 1997 and 2003: a language for describing Chinese
characters, and support for rendering them in MetaPost and \LaTeX.
This is intended to address the Chinese equivalent of the ``gaiji''
problem: how to typeset rare characters that are not included in
standard fonts or encodings.  The user can describe the missing
character and a small font containing just that character will be
automatically created and used.  In principle, HanGlyph's technology
could be used to create a full-coverage font, but as of 2012 it
doesn't appear anyone has done that.  Availability and licensing terms
are unclear; no released code or fonts seem to be available, but there
have been papers published about it. \url{http://www.hanglyph.com/}

\item IPA Mincho and IPA Gothic fonts, 2003 onward.  Note ``IPA'' in this
case stands for ``Information-technology Promotion Agency,'' not the
``International Phonetic Alphabet,'' and these fonts do not cover that IPA.
Free high-quality fonts for Japanese, TrueType format, not parameterized. 
\url{http://ossipedia.ipa.go.jp/ipafont/index.html}

\item \XeTeX, SIL International, 2004 onward: \TeX\ engine extended to
handle Unicode and modern font technologies.  Used to compile this document,
and one of the main compatibility targets for Tsukurimashou.
\url{http://scripts.sil.org/xetex}

\item Hóng Zì project by Javier R.\ Laguna.  Aimed to be a parameterized
METAFONT family for Chinese.  The last release, which was in 2006,
contained 125 characters.  No infrastructure for addressing issues
like METAFONT's 256-glyph limit, or radicals changing shape depending
on their context.  Probably abandoned.  However, it did make several
releases of code that you can still download and compile.
\url{http://hongzi.sourceforge.net/}

\item The KanjiVG Project, coordinated by Ulrich Apel, current in 2011. 
Still under development, but already has basically complete coverage, and is
deployed in several important applications.  This is not a font family, but
a database of kanji (primarily from a Japanese point of view) broken down
into strokes and radicals, with some curve points and a lot of abstract
information about how the strokes correspond to the traditional radical
classification (so that you can automatically recognize, for instance, what
示 and 礻 have in common), stroke order, and so on.  This is a valuable
resource for dictionaries and handwriting recognition systems.  Some kind of
supervised semi-automatic processing could probably turn it into a font, but
keeping the style consistent (because the database has multiple sources),
and adding the serifs and other visual information needed for styles like
Mincho, would require some significant work. 
\url{http://kanjivg.tagaini.net/} See it in action in Ben Bullock's
handwritten kanji recognizer at \url{http://kanji.sljfaq.org/draw.html}.

\item Character Description Language, from Wenlin Institute, Inc.  Current
in 2011.  This is a commercial product.  It is apparently (though the
Web page could be clearer on the exact nature of what they're selling)
a database of character descriptions similar to KanjiVG though with
wider coverage, plus a binary-only rendering library, the combination
available for license at some unspecified price.  It says it's capable
of generating MetaPost as one of its several output formats.  Not
clear to what extent there is parameterization, but presumably that
would be in the converter rather than in the resulting MetaPost.
\url{http://www.wenlin.com/cdl/}

\item Type Project Adjustable Fonts, announced in 2012.  Commercial type
foundry offers to provide Japanese-language corporate fonts that are
``adjustable'' for weight and width.  It is not clear that they actually
have a full set of kanji; from the description on the Web site it appears
that they have a ready-made set of parameterized kana and then they will
create whatever specific kanji the client wants to pay for.  The
parameterization technology is evidently smarter than the purest form of
linear interpolation (because they have a JavaScript demo highlighting the
difference) but it still seems to be within the range of what could be
accomplished with, for instance, Adobe Multiple Master. 
\url{http://typeproject.com/projects/adjustable}

\end{itemize}

%%%%%%%%%%%%%%%%%%%%%%%%%%%%%%%%%%%%%%%%%%%%%%%%%%%%%%%%%%%%%%%%%%%%%%%%

\blsubsection{デベロップメントロードマップ}{Development roadmap}

This version contains all the kyouiku kanji (the ones taught in Japanese
elementary school) through Grade 4.  The current plan is
to release a minor version after each grade level of kyouiku kanji and one
halfway through each grade level, which will take us up to version 0.12 at
the end of Grade 6.  Version 1.0 will probably be a separate version
released shortly after 0.12, with a general clean-up and renovation, but
it's possible I might skip directly from 0.11 to 1.0.

There are 1006 kyouiku kanji, though the fonts already contain more than
that number of kanji glyphs because my general practice is to add other
glyphs that are convenient to add whenever they come up, regardless of their
level.  For instance, when I added the ``gate'' radical 門 it was easy to
add many other kanji that consist of that wrapped around an easy
pre-existing kanji, even though some of those are not in common use and one,
閠, isn't even a real kanji at all, having been created by an error in the
standards process.  But having just over a thousand in the main-line roadmap
makes the end of the kyouiku kanji a good milestone for the first major
version number.

That may be a few years from now.  Progress past that point will be somewhat
dependent on how I feel about the project by then and what my personal
career situation is.  My hope is that at that point or before, I'll have the
chance to present this work at one or more conferences and that it will have
attracted some attention.  Of course, if I can figure out a way to get paid
to do it that would be nice, but attention is more important.

Although this is subject to change and cancellation, my current thought is
that the next major versions would be 2.0 with the jouyou kanji (taught in
high school, a total of 1130 additional glyphs), and 3.0 with the jinmeiyou
kanji (the ``name-only'' kanji, 983 additional glyphs).  I don't know how
I'd break those up into minor versions, but presumably I'd aim for a similar
spacing of about 100 new characters per release.  At the 3.0 point, with a
little over 3000 kanji, the fonts should be basically complete in the sense
of being usable to write the full Japanese language as most reasonably
competent native readers know it.  Many more kanji exist; I don't know how
far I'll want to take this project toward covering them all.  For reference,
looking at some other fonts I have handy, IPA Mincho contains 6682 kanji
(probably aiming to cover the JIS~208 standard), and Sazanami Gothic
contains 12202 (probably aiming to cover JIS~212).  Those might be
reasonable milestones for 4.0 and 5.0.

Figure~\ref{fig:glyph-counts} is a chart of the progress to date.  Note the
horizontal axis is labelled by version but scaled by time.  Even-numbered
versions tend to take more time because I tend to do the easier characters
in each grade level first.  All the glyph counts in these charts are for the
Tsukurimashou (Japanese) fonts alone.  The Jieubsida (Korean) fonts contain
the 11172-glyph block of precomposed syllables, which because they are
algorithmically generated cannot be well compared to the more
manually-created kanji and other glyphs.  The Jieubsida fonts also contain a
few hundred non-precomposed glyphs, beyond the core they share with
Tsukurimashou.

% glyph counts by release:

% version   date      jdate     kanji    glyphs     LOC
%   0.1    20110219   2455612       0      1307   14944
%   0.2    20110406   2455658     198      1646   21157
%   0.3    20110510   2455692     348      1797   24799
%   0.4    20111016   2455851     573      2021   30148
%   0.5    20111216   2455911     776      2260   36479
%   0.6    20120618   2456096    1110      2586   46635
%   0.7    20130307   2456358    1291      2770   48879
%   0.8    20130826   2456530    1502      2981   60177

\begin{figure}
\tikzset{horizontal axis/.style={xscale=0.11,xshift=1.5cm}}
\tikzset{vertical axis/.style={yscale=0.2}}
\centering
\begin{tikzpicture}
  \begin{scope}[vertical axis]
    \draw[arrows=-triangle 60] (0,-1) -- (0,32);
    \foreach \y/\ylbl in
      {0/0,5/500,10/1000,15/1500,20/2000,25/2500,30/3000} {
      \draw (-0.2,\y) -- (0.2,\y);
      \node[anchor=east] at (-0.2,\y) {\ylbl};
    }
    \node[rotate=90] at (-2,16) {glyphs};
  \end{scope}
  \begin{scope}[horizontal axis]
    \draw[arrows=-triangle 60] (-2,0) -- (95.0,0);
    \foreach \x/\xlbl in
      {1.2/0.1,5.8/0.2,9.2/0.3,25.1/0.4,31.1/0.5,49.6/0.6,76.0/0.7,93.0/0.8} {
      \draw (\x,-0.2) -- (\x,0.2);
      \node[anchor=north] at (\x,-0.2) {\scriptsize\xlbl};
    }
    \node at (37.5,-1) {version and time};
  \end{scope}
  \begin{scope}[horizontal axis,vertical axis]
    \node (v1k) at (1.2,0) {漢};
    \node (v2k) at (5.8,1.98) {漢};
    \node (v3k) at (9.2,3.48) {漢};
    \node (v4k) at (25.1,5.73) {漢};
    \node (v5k) at (31.1,7.76) {漢};
    \node (v6k) at (49.6,11.10) {漢};
    \node (v7k) at (75.7,12.91) {漢};
    \node (v8k) at (93.0,15.02) {漢};
    \node (v1t) at (1.2,13.07) {\labelitemi};
    \node (v2t) at (5.8,16.46) {\labelitemi};
    \node (v3t) at (9.2,17.97) {\labelitemi};
    \node (v4t) at (25.1,20.21) {\labelitemi};
    \node (v5t) at (31.1,22.60) {\labelitemi};
    \node (v6t) at (49.6,25.86) {\labelitemi};
    \node (v7t) at (75.7,27.70) {\labelitemi};
    \node (v8t) at (93.0,29.81) {\labelitemi};
    \draw (v1k) -- (v2k) -- (v3k) -- (v4k) 
      -- (v5k) -- (v6k) -- (v7k) node[sloped,above,midway] {kanji};
    \draw (v7k) -- (v8k);
    \draw (v1t) -- (v2t) -- (v3t) -- (v4t)
      -- (v5t) -- (v6t) -- (v7t) node[sloped,above,midway] {total};
    \draw (v7t) -- (v8t);
  \end{scope}
\end{tikzpicture}
\caption{Growth of glyph counts}
\label{fig:glyph-counts}
\end{figure}

Figure~\ref{fig:loc-per-glyph} gives a different view of development
progress: the number of lines of code (total lines in mp/*.mp, including
comments and blanks but not including code in other languages and locations;
also excluding jieub-*.mp, but still including a few other files from
Jieubsida) plotted against the total number of glyphs in the main
Tsukurimashou family.

\begin{figure}
\tikzset{horizontal axis/.style={xscale=0.3}}
\tikzset{vertical axis/.style={yscale=0.18}}
\centering
\begin{tikzpicture}
  \begin{scope}[vertical axis]
    \draw[arrows=-triangle 60] (0,0) -- (0,62);
    \foreach \y/\ylbl in
      {0/0,10/10000,20/20000,30/30000,40/40000,50/50000,60/60000} {
      \draw (-0.2,\y) -- (0.2,\y);
      \node[anchor=east] at (-0.2,\y) {\ylbl};
    }
    \node[rotate=90] at (-2.2,25) {lines of code};
  \end{scope}
  \begin{scope}[horizontal axis]
    \draw[arrows=-triangle 60] (0,0) -- (32,0);
    \foreach \x/\xlbl in
      {0/0,5/500,10/1000,15/1500,20/2000,25/2500,30/3000} {
      \draw (\x,-0.2) -- (\x,0.2);
      \node[anchor=north] at (\x,-0.2) {\xlbl};
    }
    \node at (15,-1.3) {glyphs};
  \end{scope}
  \begin{scope}[horizontal axis,vertical axis]
    \node (v1) at (13.07,14.944) {\labelitemi};
    \node (v2) at (16.46,21.157) {\labelitemi};
    \node (v3) at (17.97,24.799) {\labelitemi};
    \node (v4) at (20.21,30.148) {\labelitemi};
    \node (v5) at (22.60,36.479) {\labelitemi};
    \node (v6) at (25.86,46.635) {\labelitemi};
    \node (v7) at (27.70,48.879) {\labelitemi};
    \node (v8) at (29.81,60.177) {\labelitemi};
    \draw (v1) -- (v2) -- (v3) -- (v4) -- (v5) -- (v6) -- (v7) -- (v8);
  \end{scope}
\end{tikzpicture}
\caption{Lines of code per glyph}
\label{fig:loc-per-glyph}
\end{figure}

\clearpage

%%%%%%%%%%%%%%%%%%%%%%%%%%%%%%%%%%%%%%%%%%%%%%%%%%%%%%%%%%%%%%%%%%%%%%%%

\blsubsection{外のソフトのバグ}{Relevant bugs in other software}
\label{sub:bugs}

Building Tsukurimashou (and, especially, its documentation) requires the use
of some fairly advanced features of third-party software.  Some of those
features are not often used; as a result, I've become an unintentional
beta-tester, and in some cases a maintainer, for the third-party packages. 
I've previously noted bugs when they come up, in code comments and the
relevant parts of this document, but as of the current version, such bugs
have become so numerous that it makes sense to also have a central list.

\begin{itemize}
  \item Metapost (maybe even METAFONT) issue propagated to METATYPE1:
    the equation solver has a non-renewable resource of ``independent
    variable instance serial numbers'' which are consumed as code executes. 
    Basically, one is used up permanently every time an assignment statement
    executes.  Very old versions of Metapost either did not have these, or
    allowed the counter to wrap around, and so the solver would produce
    incorrect results in long-running programs.  At some time before
    version 0.641, the solver
    mistakes were fixed by the limited serial-number scheme, but
    serial numbers would run out and cause a fatal error when they reached
    $\textrm{2}^\textrm{\small 25}$.  In version 1.501, the limit was
    increased to $\textrm{2}^\textrm{\small 31}$.  Any sufficiently
    long-running Metapost program will eventually die as the limit is
    exceeded.  Some experimental versions of the Blackletter Lolita
    curve-fitter would exceed the $\textrm{2}^\textrm{\small 25}$ limit; the
    current experimental version is less computation-intensive, probably
    wouldn't exceed $\textrm{2}^\textrm{\small 25}$, and certainly fits
    comfortably in $\textrm{2}^\textrm{\small 31}$, but it seems like the
    limit should not exist at all.  Debugging is hindered by some currently
    in progress redesign work on Metapost's data structures, such that (as
    of October 2011) the cutting-edge development version leaks memory fast
    and crashes for that reason long before the serial numbers can run out
    anyway.  Memory leaks acknowledged and planned to be fixed by Metapost
    maintainer Taco Hoekwater, but it may take a while.  I've posted a link
    target to track this issue at \url{http://ansuz.sooke.bc.ca/entry/213}.

  \item METATYPE1 sometimes runs glyph names through the METAFONT
    tokenizer.  At this point, I don't know how essential that is to
    the operation of METATYPE1 or whether it can be changed.  It has
    subtle effects that can cause problems.  One issue shows up in
    glyph names that contain a decimal digit followed by a dot, as in
    ``uni1100.bug''; then what gets written into the Postscript output
    is ``uni1100bug'' because that is equivalent but more canonical in
    METAFONT syntax.  A more serious issue shows up with the glyph name
    ``uni1100.l1'' from Tsukurimashou 0.5; in the new METATYPE1 version
    0.55, that gets tokenized in a context where the token ``l'' is a
    ``spark,'' and so the whole compilation fails.  The workaround
    for that was to change ``l'' to ``lj'' (for ``lead jamo'').  A further
    workaround, introduced in Tsukurimashou 0.6, was to modify the bundled
    METATYPE1 code to save the glyph name as a string and put that in the
    Postscript file instead of the tokenized version.  It still also parses
    the names as tokens, however, so there is also code to insert a bunch of
    underscores in the internally-used name before doing that, to reduce the
    chance of name collisions from this parsing.

  \item METATYPE1 pen\_stroke\_edge macro: as of METATYPE1 version
    0.44, for reasons unknown to me if left to its own devices it will
    sometimes attempt to evaluate the ``turning angle'' of a zero vector,
    and then blow up.  This seems to happen most often when stroking a
    vector in a direction of approximately 290 degrees.  As of Tsukurimashou
    0.6, we are bundling a macro derived from METATYPE1 version 0.55, which
    seems to have fixed this bug; the workaround in previous versions of
    Tsukurimashou has been removed.

  \item Not really a bug, because they warn about it in the documentation
    and it is a reasonable consequence of design decisions made for good
    reasons, but: in some cases the METATYPE1 pen\_stroke\_edge macro's
    output does not well approximate the theoretical ideal shape that would
    be obtained by stroking the specified pen along the specified path.  A
    perfect result is not possible because the theoretical perfect curve is
    not, in general, a cubic spline curve, and there are tricky topological
    considerations in play too.  Some approximation is necessary, and the
    one chosen by METATYPE1 basically uses one control point in the output
    envelope for each control point in the input path.  That may or may not
    be enough control points to produce a visually nice result.  It seems to
    especially often be a problem when there is a tight curve and not
    actually a sharp corner in the input path, so curvature is great but not
    infinite, or when there is an inflection point in the input path.  These
    cases are best simply avoided, but Tsukurimashou's bundled version of
    METATYPE1 now contains added code that attempts to detect tight curves
    and inflection points and add extra control points automatically to
    reduce the likelihood of visual problems.  The detection rule the code
    uses was chosen by trial and error educated by dimensional analysis, not
    by solid theoretical results, and it will probably be incorrect in some
    cases.  The extra points and resulting topological weirdness necessitate
    FontForge postprocessing, and may tend to trigger bugs in FontForge, but
    FontForge postprocessing was already a dangerous necessity in this
    project for other reasons anyway.  Thanks to Shriramana Sharma and the
    Metapost mailing list for discussion encouraging me to work on this
    issue.

  \item METATYPE1 infrastructure in general: sometimes generates paths that
    some software layer (possibly METATYPE1's own code) flags as
    ``degenerate,'' triggering a fatal error.  Workaround is to filter
    things, before rendering, through the regenerate macro in intro.mp,
    which removes any very short path segments.  Possibly related: the Fill
    macro will sometimes abort in response to some conditions on ``turning
    number'' that do not appear to actually be harmful.  Workaround is to
    use our dangerousFill in intro.mp instead, which is just a copy of Fill
    with the error checking removed.  Since the switch to bundled code based
    on METATYPE1 version 0.55, this seems to be a less significant issue.

  \item FontForge spline geometry operations, such as overlap removal and
    simplification: these have historically tended to be very numerically
    unstable, and subject to some combination of infinite loops,
    segmentation faults, unexplained floating-point exceptions, bizarre
    error messages, incorrect output, and so on.  The most recent
    development versions seem to be relatively good; best advice is to
    switch to one of those, turn debugging on, and hope.  It is also
    sometimes possible to get past individual problems by switching between
    the different floating-point formats offered by FontForge's compile-time
    configuration: float, double, and long double.  But which of those is
    the best choice isn't always predictable, and as of October 2012 some
    FontForge developers are attempting to remove the option to choose
    between them (which should improve stability in the long term by
    improving maintainability, but may harm stability in the short term). 
    Some numerical tweaks in actual glyph outlines also exist to try to work
    around these issues.

  \item FontForge usually fails to automatically insert appropriate
    subtable breaks when it reads a feature file describing a large
    OpenType table.  Workaround:  our Perl code inserts explicit breaks;
    trial and error needed to figure out how frequent they ought to be.

  \item When FontForge saves an Adobe feature file, what it writes may bear
    only a passing resemblance to what was actually in the font, and in
    particular, cannot subsequently be loaded by FontForge (sometimes cannot
    be loaded at all, other times can be loaded but the loaded tables behave
    differently from the ones that were saved).  Workaround:  do not save
    feature files from FontForge.  As of October 2012 there has been some
    interest shown by FontForge developers in addressing this issue, but it
    will likely be a gradual process instead of a single complete fix.

  \item FontForge can segfault while writing a feature file because of
    dereferencing a pointer first and checking whether it was null
    afterward.  Fixed by the FontForge developers in December 2011.

  \item FontForge does not apply features of the ``DFLT'' language system as
    a default to Unicode ranges that have no explicit mention in the font
    file (Adobe spec says it should).  Workaround: explicitly list language
    systems to cover every Unicode range we care about.

  \item FontForge may write up to 23 horizontal stem hints per glyph when
    writing a PostScript-flavoured OTF font, resulting in a PostScript stack
    overflow on loading if the glyph also has a non-default advance width. 
    Went undetected for a long time apparently because only CJK fonts are
    likely to have so many horizontal stems, and CJK fonts are likely to
    be monospace and thus won't also have per-glyph non-default advance
    widths.  Fixed by changing the limit to 22, with a patch I wrote that
    the FontForge developers accepted in December 2011.

  \item FontForge save and then load of an SFD file has the effect of
    renaming the ``clig'' feature to ``rtla'' on only one of the three
    machines where I've tried it.  May be related to the x86-64
    architecture.  Not reliably reproducible.  Reported on mailing list
    November 2011, mentioned in a developer's commit message December 2011,
    not observed recently and probably fixed by now.

  \item FontForge rasterization to BDF via FreeType as opposed to whatever
    other code FontForge would use:  sometimes produces corrupt results,
    which has complicated indirect consequences because the Tsukurimashou
    infrastructures uses BDF files as input to the auto-kerning program.  As
    a result, when this bug bites the horizontal metrics go screwy on some
    fonts---notably, Jieubsida Batang PS ends up looking monospaced. 
    Reported to FontForge mailing list in November 2011, inspiring FreeType
    maintainer Werner Lemberg to find and fix an unrelated bug in FreeType
    (it was unable to process BDF files with high code points; see
    \url{http://savannah.nongnu.org/bugs/?34896}).  But that bug did not
    actually affect us because FontForge uses its own code for BDFs instead
    of FreeType's anyway, and Lemberg says can't help with the corrupt
    rasterization.  Workaround is to compile FontForge without FreeType
    support; from the GUI it is possible to just turn off FreeType
    rasterization on individual bitmap-creation operations, but that option
    doesn't seem to be available from the scripting language and so the
    package has to actually be built without FreeType.  As of April 2012,
    seems not to be an issue with Arch Linux's packaged versions.  However,
    as of a few minutes before the Tsukurimashou~0.8 release (ouch!), this
    is a problem again with the lastest FontForge, and the latest FontForge
    now requires FreeType, so the workaround of disabling FreeType no longer
    works.  I've reported it as FontForge GitHub issue number 685
    (\url{https://github.com/fontforge/fontforge/issues/685}), am hacking my
    local copy of FontForge to generate distribution fonts, and will hope to
    get it resolved properly in mainline FontForge as soon as possible.

  \item FontForge attempts to modify the names of all glyphs that it thinks
    are ``related'' whenever a glyph name changes for any reason. 
    Related-glyph renaming has wacky consequences when one glyph name is a
    substring of another, and the Beikaitoru glyph-naming scripts (among
    other things in Tsukurimashou) have to take it into account.  This issue
    was reported as FontForge Github issue number 523
    (\url{https://github.com/fontforge/fontforge/issues/523}) and patched in
    their mainline; I don't understand FontForge's release schedule if they
    have one, but presumably, distributed versions will someday contain the
    fix.  Until we're sure that day has passed, Tsukurimashou will continue
    to work around the issue.

  \item \XeTeX\ fails to advance glyph pointer after a successful match in a
    GSUB table, which has complicated consequences for chaining
    substitutions, most notably that ``ignore sub'' rules have no effect.
    This is actually a bug in the third-party ICU library which \XeTeX\ uses.
    Reported to \XeTeX\ mailing list in November 2011; after discussion there
    (\url{http://tug.org/pipermail/xetex/2011-November/022298.html}) I found
    an existing issue in ICU's bug tracker dating from June 2010
    (\url{http://bugs.icu-project.org/trac/ticket/7753}).  Poor
    workaround is to carefully write all substitution features to work
    regardless of whether the pointer advances; one useful technique is to
    make what would be an ``ignore sub'' rule instead substitute to a
    series of identical-looking glyphs that are never matched on input to
    subsequent rules.  As of mid-2012 there was interest expressed by
    \XeTeX\ developers in replacing ICU with HarfBuzz, which is known to
    handle this case correctly and may have other advantages too.

  \item \XeTeX\ fontspec package: in some versions has trouble with treating
    its WordSpace configuration option as a multiplier when font size
    changes, in a way that is most noticeable when using monospaced OTF
    fonts; complicated by the fact that it's hard to define just what should
    count as a ``monospace'' font.  Discussed at length on the \XeTeX\ 
    mailing list in February 2011
    (\url{http://tug.org/pipermail/xetex/2011-February/020065.html}) and
    eventually resulted in three items and planned fixes in the fontspec issue
    tracker (\url{https://github.com/wspr/fontspec/issues/97},
    \url{https://github.com/wspr/fontspec/issues/98},
    \url{https://github.com/wspr/fontspec/issues/99}).  The main bug has now
    been fixed, and it has also become less relevant to Tsukurimashou
    documentation since the introduction of the proportionally spaced fonts,
    but still bears some watching.  If they implement all my suggestions,
    then the current PunctuationSpace multipliers in Tsukurimashou
    documentation will become much too large and need to be reduced.
    Earlier versions of Tsukurimashou toyed with workarounds based on the
    everysel package or on poking into the internals.

  \item \XeTeX\ and fontspec do not apply by default some OpenType features
    that are supposed to be applied by default (in particular, ``ljmo'' and
    ``vjmo'').  Workaround: manually request them with the RawFeature
    option.  It is possible that more careful language tagging (specifically
    of Korean text) in the \TeX\ input might reduce or eliminate this issue.

  \item \LaTeX\ tocloft package:  as of May 2011, it sets an entry without a
    dot leader by actually requesting an entry with a leader, but with a
    font-dependent invalid spacing between dots.  The DVI renderer is
    supposed to reject the bad spacing and not set a leader at all.  With
    default font sizes, in most DVI renderers, that results in either the
    correct appearance (no leader) or an almost-correct appearance (a single
    dot instead of a leader).  But with the larger sizes of the extsizes
    classes, in some renderers, the resulting DVI file is so invalid as to
    cause the renderer to blank the rest of the page.  Reported to Will
    Robertson, package maintainer; he acknowledges it, and says will fix (by
    not requesting dot leader when they're not wanted) in next version.  A
    workaround involving poking into the package internals to set a smaller
    invalid spacing is implemented near the top of doc/bkstyle.tex.

  \item PGF (lower layer of TikZ) shapes.callouts library: in version 2.10
    only, ellipse callouts just don't work (causing fatal \TeX\ errors if
    attempted), apparently because a macro name was changed in the PGF
    development process and the change was not propagated to all files. 
    Workaround implemented in our build system consists of detecting version
    2.10 and making a local modified copy of the buggy file with the macro
    name corrected.  The problem was apparently known to the maintainers and
    already fixed in their source code repository before I noticed it.  It
    presumably will not be an issue in any future releases, but buggy 2.10
    is the latest release and widely used as of late 2011; it will probably
    remain in the wild for a long time.

\end{itemize}

%%%%%%%%%%%%%%%%%%%%%%%%%%%%%%%%%%%%%%%%%%%%%%%%%%%%%%%%%%%%%%%%%%%%%%%%
%%%%%%%%%%%%%%%%%%%%%%%%%%%%%%%%%%%%%%%%%%%%%%%%%%%%%%%%%%%%%%%%%%%%%%%%

\blsection{『作りましょう』の使い方}{Using Tsukurimashou}

Fonts appear in OpenType format in the otf/ subdirectory of the
package.  If you just unpack a distribution, there will be four fonts there
corresponding to the monospace and proportional versions of Tsukurimashou
Kaku and Mincho.  If you run the build process, that will create more.
For a quick start, all you need to do is install the font files in whatever
way is standard for installing OpenType fonts on your system.  It is safe to
delete everything else in the package, though you might want to keep the
PDF documentation files.

Samples of what the different styles look like are in the file demo.pdf,
which see.  Here's a brief summary:

\begin{itemize}
  \item {\kaku 作りましょう角~~Tsukurimashou Kaku (``Square Gothic''):
  sans-serif with squared stroke-ends.}
  \item {\maru 作りましょう丸~~Tsukurimashou Maru (``Round Gothic''):
  sans-serif with rounded stroke-ends.}
  \item {\anbiruteki 作りましょうアンビル的~~Tsukurimashou
  Anbiruteki (``Anvilicious''): extra-bold, rounded sans-serif.}
  \item {\tenshinokami 作りましょう天使の髪~~Tsukurimashou
  Tenshi no Kami (``Angel Hair''): very thin hairline display font.}
  \item {\bokukko 作りましょう僕っ娘~~Tsukurimashou
  Bokukko (``Tomboy''): felt marker style.}
  \item {\mincho 作りましょう明朝~~Tsukurimashou
  Mincho (``Ming Dynasty''): modern serif.}
\expandafter\ifx\csname haveTsuItaAtamaPS\endcsname\relax\else
  \item {\fontspec{TsuItaAtamaPS}ツイタ頭~~TsuIta Atama (``Head''):
    italics to go with Tsukurimashou Kaku.}
\fi
\expandafter\ifx\csname haveTsuItaSokuPS\endcsname\relax\else
  \item \emph{ツイタ足~~TsuIta Soku (``Foot''): italics to go with
    Tsukurimashou Mincho.}
\fi
\end{itemize} 

The remainder of this document is reference material describing the special
features of the fonts, as well as instructions on how to build and customize
your own font files, including the styles that aren't distributed in
precompiled form.
\clearpage

%%%%%%%%%%%%%%%%%%%%%%%%%%%%%%%%%%%%%%%%%%%%%%%%%%%%%%%%%%%%%%%%%%%%%%%%
%%%%%%%%%%%%%%%%%%%%%%%%%%%%%%%%%%%%%%%%%%%%%%%%%%%%%%%%%%%%%%%%%%%%%%%%

\blsection{OpenTypeのフィーチャー}{OpenType Features}
\label{sec:opentype-features}

OpenType contains a mechanism for defining what are called ``features'' in
a specific technical sense rather than the more general usage of that word. 
They are identified by four-letter tags, and generally correspond to extra
data added to the font that compatible renderers can interpret to provide
special typesetting effects.  The Tsukurimashou build system's configure
script accepts an ``-{}-enable-ot-features='' argument which can be given a
list of feature tags, or ``all'', each optionally prefaced by an exclamation
point to negate it.  These are processed left to right, so that a setting
like ``-{}-enable-ot-features=all,!ccmp'' will enable everything except the
``ccmp'' feature.  The default value is ``all.''

Be aware that just because you selected a feature during config does not
mean you can actually use it.  Features are only included in fonts if they
make sense for the particular font being built (thus, monospace fonts will
not contain proportional-only features) and if all necessary character codes
are available (thus, fonts with Japanese glyph coverage will not contain
shaping features specific to Korean script).  Also, build-time configuration
only controls what will be included in the font.  Given that the feature is
in the font, your renderer must then support the feature you want to use,
and you may need to adjust the renderer configuration to tell it to use the
feature.  Some renderers make that easier or harder to do than others, and
some renderers do not turn on by default certain features that I recommend
and the standards say should be turned on by default.

Note that although OpenType would permit such a thing, features in any
single Tsukurimashou font are NOT language- or script-specific; for
instance, the set of features available for Latin script is the same as the
set of features available for Japanese script in the same font, and the
behaviour of each feature does not change depending on whether the text is
tagged as ``English'' or ``Japanese.'' This even extends as far as, for
instance, the fact that you can really kern kanji in Tsukurimashou
proportional fonts.  A feature applied to a code point sequence for which it
has no meaning (such as ``lead jamo shaping'' applied to English text) will
simply have no effect.  Specific languages are nonetheless mentioned in the
font tables for the benefit of FontForge, which doesn't seem to handle
defaults very well, and any other software that behaves like FontForge.

Some of the features mentioned in this section are not actually OpenType
features in the technical sense; they are instead options that configure the
Tsukurimashou build system to control things that should or shouldn't appear
in the OpenType font file output.  They are given customized four-letter
tags and configured through the same interface as true OpenType features
because they are somewhat related and it's more convenient than having to
have some other interface just for them.

%%%%%%%%%%%%%%%%%%%%%%%%%%%%%%%%%%%%%%%%%%%%%%%%%%%%%%%%%%%%%%%%%%%%%%%%

\blsubsection{外の分すう}{Alternate Fractions (afrc)}

Tsukurimashou does not contain any special support for diagonal fractions,
but it does support vertical or ``nut'' fractions using the OpenType feature
``afrc.'' With the afrc feature turned on, any sequence of up to four
digits, a slash, and up to four more digits becomes a vertical fraction:

{\fontspec{TsukurimashouKaku}
1/2 → \fracfont{1/2}\qquad
34/56 → \fracfont{34/56}\qquad
789/123 → \fracfont{789/123}\qquad
4567/8901 → \fracfont{4567/8901}
}

This feature works with both the ``narrow'' digits and slash
(ASCII code 47--57, Unicode U+002F--U+0039) and the ``wide'' ones (Unicode
U+FF0F--U+FF19).  If the input digits are narrow, and the fraction is one
digit over one digit, then the resulting glyph will be narrow (the same
width as monospaced Latin characters).  Otherwise---with wide input or more
than one digit in the numerator or the denominator---the fraction
will be the width of a wide (ideographic) character.

{\fontspec{TsukurimashouKaku}
]1/2[ → \fracfont{]1/2[}\quad
]1/2[ → \fracfont{]1/2[}\qquad
]34/56[ → \fracfont{]34/56[}\quad
]34/56[ → \fracfont{]34/56[}\qquad
}

As of version 0.5, this feature works in both the monospace and proportional
fonts.  However, in order to make it work in proportional fonts it was
necessary to exclude the glyphs involved from the kerning table; as a
result, only precomposed fractions will benefit from kerning.

%%%%%%%%%%%%%%%%%%%%%%%%%%%%%%%%%%%%%%%%%%%%%%%%%%%%%%%%%%%%%%%%%%%%%%%%

\blsubsection{キャプがスモールになって}{Capitals to Small Caps (c2sc)}

{\fontspec[RawFeature=+c2sc]{TsukurimashouKakuPS}This feature replaces
capital letters with small caps, as might be desired for typesetting
all-caps abbreviations like TEPCO and IAEA.  It works much like the ``smcp''
feature.  Similar limitations apply.}

%%%%%%%%%%%%%%%%%%%%%%%%%%%%%%%%%%%%%%%%%%%%%%%%%%%%%%%%%%%%%%%%%%%%%%%%

\blsubsection{文脈代替字}{Contextual Alternates (calt)}

The ``calt'' feature in general replaces glyphs with other glyphs that look
better in certain contexts.  In the Tsukurimashou family, it is used only by
monospace fonts, to replace Latin capital letters with small caps when they
are followed by combining accents.  The accents (which do not shift
vertically) would otherwise usually collide with the upper part of the
letters.  Accents in proportionally-spaced fonts are positioned using the
GPOS table and will shift vertically when applied to capital letters.  Small
caps in monospace Tsukurimashou fonts are the same height as lowercase
letters (a little shorter than small caps in proportionally spaced fonts) to
make this substitution work nicely.

When the ``ccmp'' feature is enabled, as is default, precomposed accented
capitals (which are taller than small cap plus combining accent
combinations) will be used wherever possible, and the ``calt'' feature will
only affect unusual combinations not covered by precomposed glyphs.

%%%%%%%%%%%%%%%%%%%%%%%%%%%%%%%%%%%%%%%%%%%%%%%%%%%%%%%%%%%%%%%%%%%%%%%%

\blsubsection{グリフの併合と分解}{Glyph (De)Composition (ccmp)}

This feature's general function is to join and split glyphs that can be
thought of as combinations of other glyphs.  Exactly what that means depends
on the script and the particular font.  Often, the results of ``ccmp'' are
needed as input for other features.  It is recommended that this feature
should be turned on by default in fonts that have it, and that is required
by Microsoft's spec.

In all fonts, this feature will substitute precomposed accented Latin letter
glyphs for combinations of letter plus combining accent, wherever
precomposed glyphs exist.  The precomposed glyphs generally look better
because the accent in the precomposed glyph may change shape or size to work
better with its base letter; this issue is especially significant in the
case of monospace fonts and capital letters, because the generic accent
would usually be too low and is not subject to GPOS positioning.  When no
precomposed glyph exists, rendering will fall back on the unaccented letter
and a generic accent, with the capital-letter issue in monospace fonts
partially mitigated by the ``calt'' feature's substitutions, if those are
enabled.

Similarly, this feature will substitute precomposed voiced kana glyphs for
combinations of unvoiced kana plus combining dakuten or handakuten.

In the Jieubsida fonts, this feature breaks precomposed
syllables that don't have final consonants into their component jamo, and
combines sequences of jamo for single vowels and consonants into jamo
representing the sequences, where possible.  Splitting tail-less precomposed
syllables is necessary to support the Unicode behaviour of adding a tail to
a precomposed syllable that doesn't have one; the splitting is later undone
by the ``liga'' feature.  Combining single jamo into multiples does not seem
to be required by Unicode, but is vaguely described in Microsoft's spec, and
it seems like a reasonable thing to do.

%%%%%%%%%%%%%%%%%%%%%%%%%%%%%%%%%%%%%%%%%%%%%%%%%%%%%%%%%%%%%%%%%%%%%%%%

\blsubsection{合字}{Ligatures (liga)}

The Tsukurimashou fonts do not contain ligatures for the Latin script at
present.  Their basic letter forms were designed with monospace setting in
mind, where ligatures don't make sense; and given the importance of keeping
everything legible to readers whose native language is Japanese and who may
not be familiar with some of the advanced aspects of the Latin script, it
was a design decision to make the letters look good without ligatures even
in proportional fonts.

The Jieubsida fonts, however, contain an extensive ligature table for
hangul, and use of this feature is required to get full support for hangul
script.  Without it, the decomposition of precomposed tail-less syllables in
the ``ccmp'' feature will stand, leaving those syllables looking poorly
designed; and syllables written out as individual jamo will be approximated
with on-the-fly composition even when a precomposed glyph would be
available.  In order for the ligature table to operate correctly, the
``ljmo'' and ``vjmo'' features should also be turned on.  See the section on
Korean language support for more details of how this feature works.

%%%%%%%%%%%%%%%%%%%%%%%%%%%%%%%%%%%%%%%%%%%%%%%%%%%%%%%%%%%%%%%%%%%%%%%%

\blsubsection{頭子音の形}{Lead Jamo Shaping (ljmo)}

This feature only exists in the Jieubsida fonts and should be turned on by
default.  It replaces ``lead'' jamo glyphs (consonants at the starts of
syllables) with contextual variants that depend on the vowel and the
presence or absence of a tail.  For more information, see the section on
Korean support.

%%%%%%%%%%%%%%%%%%%%%%%%%%%%%%%%%%%%%%%%%%%%%%%%%%%%%%%%%%%%%%%%%%%%%%%%

\blsubsection{文字に付け方}{Mark to Base Positioning (mark)}

This feature only exists in the proportionally spaced fonts; a similar
effect is achieved in monospace fonts via the ``calt'' feature and
zero-width glyphs.  It should be turned on by default.  The feature
stores data for OpenType renderers to attach accents to letters, allowing
for combinations of letter and accent that do not have precomposed glyphs of
their own.  For instance, U+0071 U+0306 will generate ``\,q\char"0306\,,'' a
lowercase q with a breve, which does not correspond to any Unicode code
point and would not otherwise be available.

As of version 0.6, support for this feature is limited.  Not every base
letter you might want will necessarily be available; all the most popular
combining diacritical marks exist, but it's easy to imagine others you might
want that are not included; in many cases (especially when capital letters
are involved) the mark-to-base version of a combined character ends up
looking significantly different from the precomposed version; some of the
combinations just don't look very good; and intended extensions of the
scheme to, for instance, allow adding dakuten to Japanese kana don't
exist yet at all.  But many common cases should be covered.

%%%%%%%%%%%%%%%%%%%%%%%%%%%%%%%%%%%%%%%%%%%%%%%%%%%%%%%%%%%%%%%%%%%%%%%%

\blsubsection{点に付け方}{Mark to Mark Positioning (mkmk)}

This feature works in combination with the ``mark'' feature above, and it
only makes sense to use when ``mark'' is turned on; like ``mark'' it is only
available in proportional fonts.  It stores data used by OpenType renderers
to add more marks to glyphs that already have some marks; in particular,
this should allow stacking up more than one accent on the same letter to
create many glyphs that would be impossible by other means.

In version 0.6, this feature has been expanded to cover a few more of the
Latin accents than before, allowing the construction of such characters as
x\char"0308\char"030C.  It is still somewhat experimental, however.

%%%%%%%%%%%%%%%%%%%%%%%%%%%%%%%%%%%%%%%%%%%%%%%%%%%%%%%%%%%%%%%%%%%%%%%%

\blsubsection{メタデータ}{Metadata Table (name)}

This is not the usual kind of OpenType feature, but it can be turned on and
off through the same build-system interface.  By default, the font will
contain a ``name'' table listing metadata such as creatorship and licensing,
in both English and, as appropriate, one of either Japanese or Korean.  Some
of this information is obligatory, so if you turn off the ``name'' feature
you actually still get a name table, but it will be less completely
populated.

%%%%%%%%%%%%%%%%%%%%%%%%%%%%%%%%%%%%%%%%%%%%%%%%%%%%%%%%%%%%%%%%%%%%%%%%

\blsubsection{巴の花形}{Ornaments (ornm)}

Tsukurimashou provides eight tomoe ornaments that look like this:

{\fontspec[RawFeature=+ornm]{TsukurimashouKaku}A\quad B\quad C\quad D\quad
E\quad F\quad G\quad H}

These glyphs are encoded to the Unicode private-use code points
U+F1731 to U+F1738, and always available that way.  If the ``ornm'' feature is
turned on, then they will also appear as substitutions (not alternates!) for
the ASCII capital letters A through H; so that the text ``A BIG TEST'' comes
out as ``{\fontspec[RawFeature=+ornm]{TsukurimashouKakuPS}A BIG TEST}.'' It
is likely that the way this OpenType feature works will change in the
future, since it seems not to be current best practice to implement ornm by
substitution but I'm not quite sure yet what the best practice actually is.

These ornaments look the same in all the fonts; they do not change from one
style to the next.

%%%%%%%%%%%%%%%%%%%%%%%%%%%%%%%%%%%%%%%%%%%%%%%%%%%%%%%%%%%%%%%%%%%%%%%%

\blsubsection{FontForgeだけのメタデータ}%
{FontForge-specific Metadata (pfed)}

This is not the usual kind of OpenType feature, but it can be turned on and
off through the same build-system interface.  When selected, it causes the
build system to add a ``pfed'' table with a ``flog'' subtable in the
generated OpenType fonts, containing verbose information about the
installation on which the fonts were built.  This table is a
FontForge-specific extension of OpenType format; the name ``pfed'' refers to
PfaEdit, an earlier name of the software that later became FontForge.  Other
software will ignore this table; but users with FontForge can examine the
verbose metadata under the ``FONTLOG'' heading in the ``Font Info...''
dialog.  The data is also visible as a chunk of plain text near the end of
the file when the OTF file is examined with a general-purpose file viewer
such as ``less.''  The main interesting content is the command line that was
given to configure, and a dump of most of the variables known to Make. 
There's also a copy of the copyright notice and URLs for the project home
pages, giving in more verbose detail some of the same information included
in the ``name'' table.

I recommend activating this feature, especially if you will be building
fonts for distribution to others.  It may make bug reporting easier, because
it means that anyone who gets a copy of the font also gets some information
on where that font came from; if someone builds a font and then has trouble
with it, it'll be easier to help if the font contains this debugging
information.  It also improves the chances that should a font file get
separated from its context, someone stumbling upon it will be able to figure
out what to do with it.  The Net is full of inaccurately labelled fonts with
unknown authorship, often being sold by shady commercial enterprises that
have no legal right to do so, and we all ought to do what we can to stamp
that out.  Detailed human-readable metadata, in general, is a Good Thing.

But the font log contains information like software version numbers, user
account names, and installation directory names.  Some people have funny
ideas about the sensitivity of such information in relation to system
security; they may think that revealing it creates a real risk, not
otherwise present, of people breaking into their computers.  They might also
think it could be used to trace the origins of anonymously written PDF files
and that forensic investigators don't have many other ways to do that.

Such people are wrong.  However, my saying so won't cause them to change
their minds.  If I distributed software that attached this information to
generated fonts by default, then someone who didn't read the documentation
would eventually ``discover'' it and make a big fuss about it supposedly
being a security hole.  Who needs that?  The feature is therefore turned off
by default.  I recommend turning it on by passing the
``-{}-enable-ot-features=all'' option to configure.  The default is
``-{}-enable-ot-features=all,!pfed,'' which enables everything else.

Regardless of the setting chosen, the build system will place the same
information in a file called ``fontlog.txt'' in the txt/ subdirectory of the
build tree.  After doing a build you can read that file to see what would
have been put in the fonts.  The option setting just controls whether or not
the fontlog file will be added to the OTFs during the final packaging step.

%%%%%%%%%%%%%%%%%%%%%%%%%%%%%%%%%%%%%%%%%%%%%%%%%%%%%%%%%%%%%%%%%%%%%%%%

\blsubsection{スモールキャピタル}{Small Caps (smcp)}

{\fontspec[RawFeature=+smcp]{TsukurimashouKakuPS}Small caps are available
through the OpenType ``smcp'' feature.}  This feature
simply substitutes the 26 lowercase ASCII Latin letters with alternates
that are encoded into the Private Use Area at the code points formerly used
by Adobe for this purpose:  U+F761 to U+F77A.  At some point in the future,
other glyphs may be added to support small cap letters other than the 26
ASCII ones, and I may start using unencoded glyphs rather than mapping them
into the PUA.

In monospace fonts, these small caps are about the same height as most
lowercase letters, in order to allow their use as substitutes when combining
accents are applied to uppercase letters.  In proportionally spaced fonts,
small caps are a little taller than lowercase letters, but shorter than
ordinary caps.

%%%%%%%%%%%%%%%%%%%%%%%%%%%%%%%%%%%%%%%%%%%%%%%%%%%%%%%%%%%%%%%%%%%%%%%%

\blsubsection{ヘビーメタルウムラウト}{Heavy Metal Umlaut (ss01)}

With Stylistic Set 1 (OpenType feature ``ss01'') turned on, umlaut or
dieresis over the vowels ÄËÏÖÜŸäëïöüÿ and tilde over Ñ and ñ are
replaced by a ``heavy metal'' umlaut intended for spelling musical names
like {\fontspec[RawFeature=+ss01]{TsukurimashouKakuPS}``Motörhead,''
``Spıñal Tap,'' and ``Mormoñ Tabärnacle Choïr.''} The heavy metal umlaut
differs from the regular umlaut in that the dots are larger and pointier.

Glyphs to support this feature, including a ``spacing heavy metal umlaut''
character, are encoded into the private-use area at U+F1740 through U+F174E.

%%%%%%%%%%%%%%%%%%%%%%%%%%%%%%%%%%%%%%%%%%%%%%%%%%%%%%%%%%%%%%%%%%%%%%%%

\blsubsection{丸つき字}{Enclosed Letters and Numerals (ss02)}

With Stylistic Set 2 (OpenType feature ``ss02'') turned on, the enclosed
characters described by Unicode become available as contextual substitutions
for sequences of (in most cases ASCII) characters:

\begin{itemize}
\item (0) → \enclfont{(0)} through (50) → \enclfont{(50)}
\item (A) → \enclfont{(A)} through (Z) → \enclfont{(Z)}
\item (a) → \enclfont{(a)} through (z) → \enclfont{(z)}
\item (ア) → \enclfont{(ア)} through (ン) → \enclfont{(ン)}
\item ((1)) → \enclfont{((1))} through ((10)) → \enclfont{((10))}
\item \{0\} → \enclfont{\{0\}} through \{20\} → \enclfont{\{20\}}
\item \{A\} → \enclfont{\{A\}} through \{Z\} → \enclfont{\{Z\}}
\item{} [A] → \enclfont{[A]} through [Z] → \enclfont{[Z]}
\item <A> → \enclfont{<A>} through <Z> → \enclfont{<Z>}
\end{itemize}

The choice of which ranges of numbers and letters to support is mostly not
mine---I just implemented what I found in the Unicode charts.

Unicode only has code points for the unvoiced versions of the enclosed
katakana (e.g.\ \enclfont{(カ)(キ)(ク)} but not enclosed versions of
ガギグ), and it has no code point for enclosed ン.  I've added
\enclfont{(ン)}, encoded to private-use code point U+F1711, but not the
others.  If there's a demand for other enclosed characters, though, they are
pretty easy to add.  I held off on just defining dozens or hundreds more,
because it's not clear to me what the typical use of these characters
actually is, and a complete set of enclosed characters might be better seen
as a font in itself rather than a series of special characters inside the
font.

Note that for the enclosed katakana the substitution will accept either
ASCII parentheses or
wide parentheses (U+FF08 and U+FF09); the others work with ASCII
parentheses only.

%%%%%%%%%%%%%%%%%%%%%%%%%%%%%%%%%%%%%%%%%%%%%%%%%%%%%%%%%%%%%%%%%%%%%%%%

\blsubsection{母音の形}{Vowel Jamo Shaping (vjmo)}

This feature only exists in the Jieubsida fonts, and should be turned on by
default when it exists.  It replaces ``vowel'' jamo glyphs with appropriate
contextual variations depending on the layout of the syllable.  The layout
that will be used by ``vjmo'' is actually chosen during execution of the
``ljmo'' feature, which must be applied before ``vjmo'' (as required by
Microsoft's specification) for ``vjmo'' to work properly.  Note that
Microsoft also describes, but we don't use, a similar ``tjmo'' feature for
reshaping the tail.

%%%%%%%%%%%%%%%%%%%%%%%%%%%%%%%%%%%%%%%%%%%%%%%%%%%%%%%%%%%%%%%%%%%%%%%%

\blsubsection{私用符号位置を避けて}{Limit PUA Code Points (xpua)}

This is not the usual kind of OpenType feature, but is configured through
the same interface as the others.

For software engineering purposes it is
convenient to have a unique Unicode-like code point for every glyph in the
font family.  However, some of those glyphs are not intended to be accessed
directly by users, and more than one of them may correspond to the same
Unicode character in ordinary text.  This feature removes the private-use
code points from many such glyphs.

For instance, in Jieubsida there is a glyph named ``uni112A'' for code point
U+112A, but there is also a glyph named ``uni112A.bug'' which looks
identical but is referenced in the substitution table in order to work
around a bug in the ICU library used by \XeTeX.  The uni112A.bug glyph, as
its name implies, is used to typeset the character U+112A, but it is
referred to internally using the code point U+FF72A.  That is a code point
in one of the Unicode Private Use Areas (PUAs), which are special ranges of
code points reserved by Unicode for purposes like this.  The U+FF72A code
point is the one that could, for instance, be used to enable or disable the
uni112A.bug glyph in the -{}-enable-chars option to the configure script.

The Tsukurimashou package uses hundreds of these PUA code points.  They are
necessary for keeping track of things during the compilation and assembly of
the fonts.  However, most of the glyphs with PUA code points are not
intended to be directly accessed through their code points, only indirectly
through glyph substitution.  Having an unnecessary PUA mapping in the font
for any given glyph may cause some buggy software to ignore the glyph's
name, resulting, for instance, in incorrect text when cutting and pasting
from a PDF file.  Encoding glyphs to PUA code points could also possibly
conflict with other uses of the same PUA code points by other software.  So
all in all it is usually preferable that these glyphs should not have any
code point mappings at all in the finished fonts.  On the other hand,
removing their code point mappings means it is no longer possible to access
the glyphs directly by code point; then short of using third-party software
to bypass the mapping tables and access glyphs by raw glyph numbers, users
\emph{must} use the glyph substitution features instead, which means both
that the glyphs are only available in the contexts where the substitution
features are designed to provide them, and that the users must have OpenType
rendering software that at least sort of works (which is not necessarily
common in the field).

The ``xpua'' feature allows a configuration choice between these two
situations.  With it turned on, which is the default, most of
Tsukurimashou's PUA glyphs will have their code points stripped in a late
stage of font packaging.  It is not universal: some glyphs that do not
correspond directly to any Unicode character remain accessible at PUA code
points because that's relatively harmless and they may be difficult to
invoke in other ways.  But glyphs that exist only to support OpenType
features (such as the ones needed for vertical fractions and Korean syllable
layout) end up as unencoded glyphs in the final fonts.  As of version 0.7,
the Tsukurimashou PUA-encoded characters that keep their encodings in the
final fonts notwithstanding ``xpua'' are exactly the glyphs in the U+F17xx
range.

Turning off ``xpua'' causes every glyph to be encoded at its internal code
point.  That was the state of affairs in versions of Tsukurimashou prior to
0.7, which didn't offer a choice.

%%%%%%%%%%%%%%%%%%%%%%%%%%%%%%%%%%%%%%%%%%%%%%%%%%%%%%%%%%%%%%%%%%%%%%%%
%%%%%%%%%%%%%%%%%%%%%%%%%%%%%%%%%%%%%%%%%%%%%%%%%%%%%%%%%%%%%%%%%%%%%%%%

\blsection{付加文字}{Extra glyph coverage}

%%%%%%%%%%%%%%%%%%%%%%%%%%%%%%%%%%%%%%%%%%%%%%%%%%%%%%%%%%%%%%%%%%%%%%%%

\blsubsection{げんじもん}{Genjimon}

Glyphs for the 54 Genjimon are encoded to private-use code points U+F17C1 to
U+F17F6, in the order of the corresponding Tale of Genji chapters
(``Kiritsubo'' to ``Yume no Ukihashi'').  The style of these glyphs varies a
fair bit between the different font styles.  Here are samples:

\begin{itemize}
  \item {\kaku Kaku: \Large 󱟕󱟖󱟗}
  \item {\maru Maru: \Large 󱟕󱟖󱟗}
  \item {\anbiruteki Anbiruteki: \Large 󱟕󱟖󱟗}
  \item {\tenshinokami Tenshi no Kami: \Large 󱟕󱟖󱟗}
  \item {\bokukko Bokukko: \Large 󱟕󱟖󱟗}
  \item {\mincho Mincho: \Large 󱟕󱟖󱟗}
\end{itemize}

A set of Genjimon-only fonts, derived from the Tsukurimashou code, exists as
a spin-off project in the genjimon subdirectory of the Tsukurimashou
distribution.  The documents there contain some more information about the
history of these symbols and what they signify.

%%%%%%%%%%%%%%%%%%%%%%%%%%%%%%%%%%%%%%%%%%%%%%%%%%%%%%%%%%%%%%%%%%%%%%%%

\blsubsection{えききょう}{I Ching}

All the I Ching-related characters defined by Unicode in the Basic
Multilingual Plane are supported:  trigrams U+2630 to U+2637, monograms
U+268A and U+268B, digrams U+268C to U+268F, and hexagrams U+4DC0 to
U+4DFF.  These characters are sized to fit in the same box as kanji, and so
in most cases you will probably want to scale them to a significantly larger
point size than that of nearby text.  The style does change somewhat from
one font to the next.

There are also some ``I Ching dot'' combining characters in the private-use
code points U+F1701 to U+F1709, intended to show movement of lines 1 through
6 of hexagrams or 1 through 3 of trigrams, encoded in that order.  The idea
is that you can typeset something like this:

{\Large\fontspec{TsukurimashouKaku}䷞󱜂󱜄→䷯}

using a sequence of codes like this:

U+4DDE U+F1702 U+F1704 U+2192 U+4DEF

%%%%%%%%%%%%%%%%%%%%%%%%%%%%%%%%%%%%%%%%%%%%%%%%%%%%%%%%%%%%%%%%%%%%%%%%

\blsubsection{解字の文字}{Ideographic Description Characters}

Unicode defines a set of special code points (U+2FF0 to U+2FFB) and a syntax
for using them to describe what Han-script characters look like.  For
instance, 「歯」 might be described by the string 「⿱止⿶凵米」, which
expresses that 歯 consists of 止 written above a composite sub-glyph that
itself consists of 凵 wrapped around 米.  Unicode's scheme for character
description appears to have been inherited more or less unmodified from a
similar scheme in the Chinese GBK standard.

Tsukurimashou includes glyphs for the special characters.  There is also
support in the build system for a ``make eids'' target, which generates a
file called tsukurimashou.eids in the txt/ directory.  That file is a
collection of Extended Ideographic Description Sequences (EIDSes) describing
the construction of the kanji in the font.  This is the interface to
IDSgrep, a program for searching kanji according to powerful
criteria on their visual structure.  IDSgrep is available from the same
SourceForge.JP project that hosts Tsukurimashou; see that package and its
documentation for more information on EIDS format and how to use it.

With or without IDSgrep, Tsukurimashou may be used to typeset the special
characters for the descriptions.  The glyphs look like this:
\char"2FF0\char"2FF1\char"2FF2\char"2FF3%
\char"2FF4\char"2FF5\char"2FF6\char"2FF7\char"2FF8\char"2FF9%
\char"2FFA\char"2FFB

%%%%%%%%%%%%%%%%%%%%%%%%%%%%%%%%%%%%%%%%%%%%%%%%%%%%%%%%%%%%%%%%%%%%%%%%

\blsubsection{公のユーログリフ}{Official Euro Sign}

Fonts that include a symbol for the European currency unit (Unicode U+20AC)
usually re-draw it to match the style of the rest of the font, and the
Tsukurimashou fonts are no exceptions.  However, there is an official glyph
design that apparently was intended to be normative---in theory, you're
supposed to use the official glyph, which does not change with the style of
the rest of the font, even if it clashes.  For those who want to do so, the
official Euro sign is included in the Tsukurimashou fonts at private-use
code point U+F1710.

It looks like this:  {\Large 󱜐}

%%%%%%%%%%%%%%%%%%%%%%%%%%%%%%%%%%%%%%%%%%%%%%%%%%%%%%%%%%%%%%%%%%%%%%%%
%%%%%%%%%%%%%%%%%%%%%%%%%%%%%%%%%%%%%%%%%%%%%%%%%%%%%%%%%%%%%%%%%%%%%%%%

\blsection{ハングル語}{Korean Language Support}

\expandafter\ifx\csname haveJieubsidaDodumPS\endcsname\relax

This user manual was built without the Korean fonts (in particular,
Jieubsida Dodum PS) and as a result, the section on Korean support cannot be
typeset.

\else

First, please note that in general (with some exceptions) I follow the
practices of Unicode for such things as the names of letters, the name of
the writing system itself, and so on.  Naming decisions are thought by some
people to have political implications.  It is not my purpose here to take a
position on any issues except technical ones of font design; but for
practical reasons I must settle on some set of names.

The Tsukurimashou project is generally focused on the Japanese language; but
Korean has some degree of connection to Japanese, the hangul writing system
used for it is technically interesting and superficially appears easy, and
for various reasons, I decided to extend Tsukurimashou to include some
support for Korean.  The result is the ``Jieubsida'' {\dodum 「지읍시다」}
series of fonts.  The name is as nearly as possible a direct translation
from Japanese to Korean of the name ``Tsukurimashou'' 「作りましょう」. 
Jieubsida builds from the same code base as Tsukurimashou, if you select the
appropriate options during configuration.

There are several ways to write Korean in Unicode, and they are
supported to varying degress by Jieubsida:
\begin{itemize}
  \item Individual jamo from the ``hangul compatibility jamo'' range
    of code points, U+3131 to U+318E.
  \item Precomposed syllables from the ``hangul syllables'' range,
    U+AC00 to U+D7A3.
  \item Conjoining jamo from the ``hangul jamo'' range, U+1100 to U+11FF, and
    its supplements.
  \item Hanja (Han Chinese characters) from the ``CJK unified ideographs''
    range, U+4E00 to U+9FFF, and its supplements.
\end{itemize}

Hangul is theoretically a purely phonetic alphabet.  Letters are called
jamo, and each one is supposed to correspond to exactly one phoneme of the
Korean language.  The set of jamo in Unicode and Jieubsida is a little
bigger than in the standard present-day Korean language, as a result of
historical changes in the centuries since the writing system was introduced. 
The rules for how jamo may be used together have also become more
restrictive.

Words are written divided into syllables, with each syllable arranged
according to certain rules to fit into a square box---much like the square
box used per character in writing Japanese or Chinese, but with Korean,
there are multiple jamo in each box, instead of one kana or kanji per box,
and the question of whether to count each jamo as a character or each box as
a character is the start of the fun.

Every syllable consists of a ``lead'' containing between one and three jamo,
a ``vowel'' containing between one and three jamo, and a ``tail'' containing
between zero and three jamo.  There are ``consonant'' and ``vowel'' jamo;
the lead and tail consist exclusively of consonant jamo and the vowel
consists exclusively of vowel jamo.  Leads, vowels, and tails are not
allowed to be just any combinations of the right number of the right kind of
jamo; there are relatively short lists of possible leads, vowels, and tails. 

In the relatively standardized present-day form of the Korean language,
there are 19 different leads, 21 different vowels, and 28 different tails
(including the empty tail of zero jamo), and only two of these (the vowels
{\dodum 「ㅙ」} and {\dodum 「ㅞ」}) contain more than two jamo.  One of the
19 leads actually corresponds to an empty or silent lead with no sounds in
it, but the empty lead is written using the single jamo {\dodum 「ㅇ」},
which is not otherwise allowed as the lead; thus
the lead is always written with at least one jamo, and in the relatively
standardized present-day language, at most two.  Also, two-jamo leads on the
list of 19 always consist of one jamo repeated twice (not two different
ones); some two-jamo tails contain two different jamo.  Combinations of more
than two jamo, and other single and double jamo not on those lists, occur in
archaic contexts, less-popular dialects, and so on; but there are very many
of those longer, or merely other, combinations defined by Unicode.

%%%%%%%%%%%%%%%%%%%%%%%%%%%%%%%%%%%%%%%%%%%%%%%%%%%%%%%%%%%%%%%%%%%%%%%%

\blsubsection{コンパチのジャモ}{Compatibility jamo}

The ``hangul compatibility jamo'' range is fully supported by Jieubsida. 
It's easy: one jamo per character box.  But it is not done to actually write
the Korean language that way; readers and writers want one syllable per
character box, with the individual jamo changing size and layout to pack
nicely.  As a result, the ``compatibility jamo'' glyphs are basically only
useful in documents like this one, which discuss the technical details of
the writing system.

Hangul compatibility jamo look like {\dodum 「ㅈㅣㅇㅡㅂㅅㅣㄷㅏ」}. 
Writing words that way leads to some ambiguity in syllable division, which
may be one reason it never caught on (though the same ambiguity exists in
most Romanization forms).

%%%%%%%%%%%%%%%%%%%%%%%%%%%%%%%%%%%%%%%%%%%%%%%%%%%%%%%%%%%%%%%%%%%%%%%%

\blsubsection{併合のシラブル}{Precomposed syllables}

The Unicode standard takes the 19 leads, 21 vowels, and 28 tails of the
relatively standardized present-day language and
multiplies them together to get 11172 possible syllables.  Each of those
syllables has its own code point in the ``hangul syllables'' range.  A font
with a glyph for each one, or even just a glyph for each of the fraction of
them that actually occurs in present-day relatively standardized Korean, can
be used to write the present-day language with a minimum of fuss.  That's
what most people do; most Korean documents on the Web are encoded into that
range of Unicode; and Jieubsida contains the full set of glyphs to support
it.

The precomposed syllables look like {\dodum 「지읍시다」}.

%%%%%%%%%%%%%%%%%%%%%%%%%%%%%%%%%%%%%%%%%%%%%%%%%%%%%%%%%%%%%%%%%%%%%%%%

\blsubsection{併合のジャモ}{Conjoining jamo}

If you want to represent Korean syllables that contain leads, vowels, or
tails other than those in the Unicode precomposed syllables, or if you want
to be able to process text at a sub-syllable level (which might be useful
for input methods, among other things), then you may end up dealing with the
``conjoining jamo.'' Unicode attempts to have a code point for every
possible lead; one for every possible vowel; and one for every possible
tail, including separate lead and tail code points in the fairly common case
of the same combination of jamo being allowed as both a lead and a tail. 
The main sequence of Unicode jamo code points is in U+1100 through U+11FF;
then there are ``extension'' blocks in U+A960 through U+A97C (extension A,
for leads) and U+D7B0 through U+D7FB (extension B, for vowels and tails). 
As of version 0.6, Jieubsida supports them all, although some rare vowels
that pose visual layout problems (such as U+D7BE {\dodum 「\char"D7BE」})
may not look as good as you might hope.

You are supposed to be able to write your documents using those code points,
and then magic beyond the scope of Unicode is supposed to typeset them
properly.  Unicode defines a ``canonical equivalence'' between the
precomposed syllables and the conjoining jamo, where every precomposed
syllable can be interchanged with the conjoining jamo from which it's made. 
It is even supposed to be possible to take a precomposed syllable with no
tail, follow it by a tail conjoining jamo, and have the result convert back
into the precomposed syllable with the tail.

Theoretically, to look right all the jamo in
the syllable should be able to change shape, size, and positioning in
response to the other jamo in the syllable.  That's difficult to implement
at the font level.

One simple way to make it work is to sacrifice the layout.  Suppose we split
up the character box in a fixed, compromise way, into sections for the lead,
vowel, and tail, and then define the glyphs for those things so that they
will nicely fill their respective areas.  Define the leads and tails to be
zero-width characters, with the lead glyphs appearing to the right of their
reference points and the tails to the left; and define the vowels to be
full-width characters.  Then if you typeset a lead, a vowel, and optionally
a tail (bearing in mind that every syllable contains one lead, one vowel,
and zero or one tails), with no kerning, the result is that the parts
overlay one another and produce a sort of okay-looking syllable box.  It
won't be great because the spaces assigned to each jamo don't change
depending on the other jamo in the character, so that for instance all the
leads get shoved over to the left even when the vowel is something
horizontal, because there could be a vertical vowel and there has to be
space reserved for it on the right in every lead jamo glyph.

Conjoining jamo overlaid in this way look like
{\fontspec[RawFeature={+ccmp,-ljmo,-vjmo,-liga}]{JieubsidaDodumPS}%
「\char"110C\char"1175%
\char"110B\char"1173\char"11B8%
\char"1109\char"1175%
\char"1103\char"1161」}; compare with precomposed {\dodum 「지읍시다」}.  It
does not look very appealing, especially because that's a near-worst-case
example where three out of four syllables have empty tails and the fixed
layout cannot stretch jamo to consume the resulting blank space; but this
approach has the big advantage that it can express syllables full of archaic
jamo combinations, such as {\dodum 「\char"113A\char"1194\char"11D7%
\char"1159\char"11A5\char"11EB」}.  Hundreds of thousands of syllables can
be constructed that way, far exceeding any reasonable set of precomposed
glyphs.

There are two layers of additional processing in place to improve the
typesetting of conjoining jamo.  First, OpenType substitution rules in the
``ljmo'' and ``vjmo'' features, which should be turned on by default (but
must be activated manually in current \XeTeX), recognize cases where a
different layout would be better. The precomposed jamo vary the entire
layout of each syllable depending on all the jamo in the syllable, so that
any given jamo might appear in hundreds of subtly different sizes and
locations.  It is not practical to store all those variations of every jamo
as individual glyphs, but Jieubsida compromises by recognizing six different
standardized layouts for a syllable, depending on the shape of the vowel and
the presence or absence of a nonempty tail.

{\hspace*{\fill}\begin{tikzpicture}
\draw (-4,3) rectangle (-2,5);
\draw (-4,4.2) -- (-2.8,4.2) -- (-2.8,5);
\draw (-4,3.8) -- (-2,3.8);
\node at (-3.4,4.6) {L};
\node at (-2.4,4.4) {V};
\node at (-3,3.4) {T};
\draw (-1,3) rectangle (1,5);
\draw (0.2,3.8) -- (0.2,5);
\draw (-1,3.8) -- (1,3.8);
\node at (-0.4,4.4) {L};
\node at (0.6,4.4) {V};
\node at (0,3.4) {T};
\draw (2,3) rectangle (4,5);
\draw (2,4.2) -- (4,4.2);
\draw (2,3.8) -- (4,3.8);
\node at (3,4.6) {L};
\node at (3,4) {V};
\node at (3,3.4) {T};
\draw (-4,0) rectangle (-2,2);
\draw (-4,0.8) -- (-2.8,0.8) -- (-2.8,2);
\node at (-3.4,1.4) {L};
\node at (-2.4,1) {V};
\draw (-1,0) rectangle (1,2);
\draw (0.2,0) -- (0.2,2);
\node at (-0.4,1) {L};
\node at (0.6,1) {V};
\draw (2,0) rectangle (4,2);
\draw (2,0.8) -- (4,0.8);
\node at (3,1.4) {L};
\node at (3,0.4) {V};
\node at (-3,2.7) {layout 0};
\node at (0,2.7) {layout 1};
\node at (3,2.7) {layout 2};
\node at (-3,-0.3) {layout 3};
\node at (0,-0.3) {layout 4};
\node at (3,-0.3) {layout 5};
\end{tikzpicture}\hspace*{\fill}\par}

The default forms of all the conjoining jamo, which appear if the
substitution features are turned off, are the layout 0 forms.  Some vowel
jamo extend in both the vertical and horizontal directions, but in practice
the most often-used ones are only vertical or only horizontal, and such
vowels can be reused for layouts 1 and 2; presence of the relevant vowel is
what triggers the switch to the other layout for the lead and tail.  In
addition to the layout 0/1/2 forms, each vowel jamo has an alternate form
for tail-less syllables.  Lead jamo have four alternate forms in addition to
layout 0, since the forms for layouts 1 and 3 are identical and can share a
glyph.  An additional copy of the layout 0 leads also exists, as part of the
workaround for a bug in \XeTeX's handling of GSUB substitutions.  Tail jamo
do not need alternate forms, because they only appear in layouts 0, 1, and
2, and have the same space allocation in all three layouts.  All these extra
glyphs are coded into the private use Unicode code points between U+FF200
and U+FF7FF, and can dangerously be turned off with the configuration code
point controls, but it is not intended that they actually be accessed
through the private-use code points; instead, the OpenType substitutions
will put them in in place of the layout 0 forms from U+1100 through U+11FF
and its supplements, where necessary.

With the alternate forms chosen by the substitution rules, but no
precomposed syllables, the font
family name is rendered as
{\fontspec[RawFeature={+ccmp,+ljmo,+vjmo,-liga}]{JieubsidaDodumPS}
「\char"110C\char"1175%
\char"110B\char"1173\char"11B8%
\char"1109\char"1175%
\char"1103\char"1161」}.
Note that is almost identical to the
precomposed-syllable version, {\dodum 「지읍시다」}.  There are some minor
differences in the size and spacing of the jamo, because the
glyph-overlay approach basically makes kerning impossible, and the six
canned layouts still do not quite match the per-syllable customized layouts
of the precomposed syllables.

The last layer of processing consists of ligature rules in the OpenType
``liga'' feature, turned on by default, which recognize jamo sequences that
correspond to the precomposed syllables and substitute precomposed glyphs
wherever possible.  In order to work properly (in particular, for correct
recognition of tail-less syllables) the alternate-layout substitutions must
have already run before these ones.  With the ligature substitutions in
effect, actually entering the conjoining jamo for the
standardized present-day Korean language will give the same result as using
the precomposed code points.  The family name entered this way appears as
{\dodum 「\char"110C\char"1175%
\char"110B\char"1173\char"11B8%
\char"1109\char"1175%
\char"1103\char"1161」}, which should be both visually and logically
(same glyph sequence) identical to {\dodum 「지읍시다」}.

Also worth mentioning is the ``ccmp'' feature, which is supposed to run
before all others.  It splits precomposed syllables that have no tails into
their component lead and vowel jamo code points, and joins consecutive
conjoining jamo into clusters, where possible.  If you are not doing
anything weird with Unicode canonical equivalence, this is unlikely to have
much effect; the split precomposed syllables will be recombined by ``liga,''
and you won't have any clusterable conjoining jamo in the input anyway. 
However, this level of processing is needed (with the others) in order to do
some of the strange things that the Unicode Consortium in its wisdom
thinks you want to do, such as putting conjoining tail {\dodum 「ㅁ」} after
precomposed {\dodum 「기」} and expecting it to turn into precomposed
{\dodum 「기\char"11B7」}.  I'm told that not many fonts actually support
that.  Jieubsida does!  However, I cannot promise that absolutely every
weird Unicode thing will work.  Most of them should.

The big win is that with all the substitution
features enabled it is possible to mix common conjoining jamo,
rare conjoining jamo, and precomposed syllable code points, and the result
will be the best that is reasonably possible, automatically invoking
precomposed syllables where possible and best-fit alternates otherwise. 
Here is a nonsense example demonstrating all the different layouts:
{\dodum%
「\char"110E\char"117F\char"11ED지%
\char"111B\char"116B읍%
\char"1140\char"1175\char"11F9시%
\char"114F\char"1199\char"110F\char"1187\char"11D4다%
\char"1124\char"1172」}

%%%%%%%%%%%%%%%%%%%%%%%%%%%%%%%%%%%%%%%%%%%%%%%%%%%%%%%%%%%%%%%%%%%%%%%%

\blsubsection{ハンジャ}{Hanja}

The above discussion applies to the hangul phonetic script.  Korean has
traditionally also been written with Han characters (called ``hanja'' in
Korean), and they are still used a little bit, in combination with the
phonetic script---somewhat like the use of kanji in Japanese, but with the
important difference that it is considered acceptable to write present-day
Korean entirely in hangul without any hanja, whereas kanji are a necessity
for present-day Japanese.  The best information I have is that most Korean
people today are actually able to read few to no hanja, but they remain in
use as abbreviations in contexts like corporate names and newspaper
headlines where exact comprehension is not terribly important.

The trouble is that Han characters as used in Japanese, Han characters as
used in Korean, and Han characters as used in other languages that might use
them (such as Vietnamese and various dialects of Chinese) are all
incompatibly different despite being ``unified'' to overlapping subsets of
the same Unicode code point range.  For instance, the character 「神」
meaning ``god'' is written as shown, in present-day standard Japanese, and is
very similar to that in both ``simplified'' and ``traditional'' Chinese.
The equivalent hanja character (at the same code point, U+795E) has a
left-side radical that looks like 示 instead of 礻.  In fact, I've seen the
Korean form of this character in some archaic and ceremonial contexts in
Japan, so the switch to the newer form in Japanese was probably quite
recent; but if you write the present-day Japanese form where you should
write the Korean form, it will look wrong.

The current versions of the Jieubsida/Tsukurimashou fonts address this issue
by simply not supporting hanja at all.

Supporting hanja would mean going through the entire set of kanji, finding
all the differences, and creating separate versions to be the hanja or
parameterizing the differences or otherwise dealing with them all, and that
is beyond the current planned scope of the project.  So is supporting Korean
in the first place, actually; I don't speak it, hangul got added because it
seemed interesting, it was likely to be a relatively large payoff for a
relatively small amount of work, and (as has been demonstrated by experience
with, among other things, FontForge hinting bugs) it provided a good testbed
for some techniques that will later be applied to the Japanese side of the
project.

It's easy to imagine that someone could build a font containing both the
kanji glyphs from Tsukurimashou and the hangul glyphs from Jieubsida, and
then try to use it to write hanja.  I hope you will not attempt that, and
the Tsukurimashou build system is designed not to build such fonts.  There
are too many fonts like that on the Net already as a result of people's
attempts to build fonts for ``all of Unicode'' without really understanding
the consequences of what they're doing.  Obviously, in an open source system
I can't stop people from creating such chimeras for their own use, but I
don't want someone to create a Korean-language document with Japanese fake
hanja in it, be asked ``What is that horrible font you're using that gets
the hanja shapes wrong?'' and have them say ``It's Jieubsiuda by Matthew
Skala!'' Please, if you're going to set kanji next to hangul and call them
hanja, try to make clear to people who see the results that it was your idea
to do that and not mine.

\fi

%%%%%%%%%%%%%%%%%%%%%%%%%%%%%%%%%%%%%%%%%%%%%%%%%%%%%%%%%%%%%%%%%%%%%%%%
%%%%%%%%%%%%%%%%%%%%%%%%%%%%%%%%%%%%%%%%%%%%%%%%%%%%%%%%%%%%%%%%%%%%%%%%

\blsection{フォントの名前ついて}{Regarding Font Names}
\label{sec:fontnames}

First, I will use the term ``font'' to refer to an OTF or similar file with
a complete glyph set (or as many glyphs as have been selected with
configure).  This will generally be a few thousand glyphs for the
Tsukurimashou family and about 13 thousand for the Jieubsida family. 
Because METATYPE1 is limited to producing Postscript font files with at most
256 glyphs each, the system builds ``subfonts'' each corresponding to a
``page'' of the Unicode character set---that is, an aligned 256-code-point
block.  For instance, the range U+0000 to U+00FF is the page containing
ASCII and Latin-1 characters.

The division into pages provides some advantages for the build system: when
something is changed on one page it may be possible to avoid re-compiling
the others, and if there are many pages to be compiled then they can be
compiled simultaneously on a multi-core computer.  So we would probably
choose to keep this design feature even if some future version of the system
were no longer bound by the Postscript 8-bit limit that originally made it
necessary.

The most authoritative form of the name of a font is the ``Hamlog name,''
which is an ordered quadruple of Hamlog atoms.  (The Hamlog language is
described elsewhere in this document.)  Hamlog names are manipulated
by, for instance, the code in select.hl.  The items in the quadruple are
called the family, the style, the weight, and the spacing.  An example
Hamlog name might be ``(tsukurimashou, bokukko, demibold, monospace).''

The family represents the broad category of the font.  Generally, all fonts
sharing the same family value will have basically the same glyph set and the
same overall shapes for all glyphs, though the visual style of the strokes
will vary.  If there are significant differences in shapes or glyph sets,
they will be put in a separate family.  At present the supported values for
the family are ``tsukurimashou,'' ``tsuita,'' and ``jieubsida,'' though
there is also some experimental (less than alpha status) code for a
family to be called ``blackletter\_lolita.''

The possible values for style will depend on the family.  Styles generally
correspond to sets of preset values for the adjustable parameters of a
family's letter-drawing code.  The style values currently allowed, per
family, are as follows.  Style names marked with an asterisk are not allowed
as part of the basic Hamlog name but may be introduced via rewriting, as
explained below.
\begin{description}
  \item{\anbiruteki tsukurimashou} kaku, maru, mincho, bokukko,
    anbiruteki$^*$, tenshinokami$^*$.
  \item{\anbiruteki tsuita} soku, atama.
  \item{\anbiruteki jieubsida} dodum, batang, sunmoon.
  \item{\anbiruteki blackletter\_lolita} cosette.
\end{description}

The weight may be ``extralight,'' ``light,'' ``normal,'' ``demibold,''
``bold,'' or ``extrabold.''  The spacing may be ``monospace'' or
``proportional.''

The Hamlog name may be subject to rewriting according to rules in the
select.hl file, specifically the ``do\_style\_xlat/8'' predicate.  These
rules support a few style/weight combinations that have been given unique
style names of their own, both because that's fun and as a legacy of earlier
versions in which the weight parameter was not fully implemented.  This code
is also meant to support future per-language localization of name components
like ``demibold,'' in particular so that Korean fonts won't end up with
Japanese in their names, though that aspect is not yet fully implemented.
At present, there are just two rewriting rules:
\begin{itemize}
  \item (tsukurimashou, maru, extralight, $X$) is rewritten to
    (tsukurimashou, tenshinokami, normal, $X$).
  \item (tsukurimashou, maru, extrabold, $X$) is rewritten to
    (tsukurimashou, anbiruteki, normal, $X$).
\end{itemize}

The Hamlog name is translated into a few other kinds of names, by a
combination of Hamlog code in select.hl and Perl code embedded in
Makefile.am.  One important translated name is the ``short name,'' which is
a sequence of two, three, or four short ASCII tokens, which in different
parts of the code may be separated by either hyphens or underscores.  An
example short name is ``tsuku-mi-ps,'' corresponding to the Hamlog name
``(tsukurimashou, mincho, normal, proportional).'' Each element in the
Hamlog name is translated into a token for the short name, according to the
table given by the short\_name/2 predicate in select.hl, but some Hamlog
atoms (namely ``normal'' and ``monospace'') translate to the special value
``l\_\_\_da'' (for ``lambda,'' meaning empty string; triple underscore marks
it as requiring special handling in the Perl wrapper) and are removed,
often leaving fewer than four tokens in the result.  Short names are always
derived from the Hamlog name \emph{before} rewriting.

Subfonts also have short names of their own, formed by adding an extra token
at the end which is the two- or three-digit hexadecimal representation of
the page number.  For instance, ``tsuku-mi-ps-00'' would be the short name
of the subfont covering U+0000 to U+00FF of ``tsuku-mi-ps''.  These names
are often used in constructing filenames for intermediate files during
build.

Anyone contemplating modification of this system should be aware that
short-name tokens other than page numbers must \emph{not} be valid
hexadecimal digit sequences that could be mistaken for page numbers; the
build system needs to be able to distinguish subfont and full-font short
names from each other by testing for a hexadecimal number as the last
token.  To prevent terms like ``extra bold'' and ``demibold'' from
abbreviating to valid hexadecimal numbers, terms derived from the word
``bold'' are abbreviated as if it were spelled ``qold'' or ``bqld.''

Fragments of short names are also used to name source code files.  Files
that follow this scheme (not all source files do) have names consisting of
one family token, a hyphen, and one other token, followed by the appropriate
extension.  Building a subfont involves creating a temporary driver file
that includes all the relevant source files for that font's short name.  For
instance, to build ``tsuku-mi-ps-00.pfb'' the build system would create and
execute a driver file that loads ``tsuku-mi.mp,'' ``tsuku-ps.mp,'' and
``tsuku-00.mp,'' in that order.  Some of those files might themselves load
other files outside this naming scheme.  Almost all file inclusion is done
via conditionals in METAFONT to prevent any file from being included more
than once, but double inclusion should also be both avoided by design and
usually harmless should it occur by accident.  The file inclusion macros in
preintro.mp also support a queue of ``late'' includes, which will be
processed after the others to allow (for instance) code associated with a
style to override code associated with a weight.

In the case of the family short name token not being ``tsuku,'' the build
system will use source files that match the subfont's family token where
those exist, and default to ``tsuku'' when such files don't exist.  For
instance, to build ``bll-co-01.pfb'' (Blackletter Lolita Cosette, page 01)
it might load ``bll-co.mp'' (which exists), but then load ``tsuku-01.mp''
because ``bll-01.mp'' doesn't exist.  Thus, the ``tsukurimashou'' family can
be thought of as a generic ancestor from which any others inherit.

Fonts also have ``long names,'' which are created by a similar translation
process.  Long names are always derived from the Hamlog name \emph{after}
rewriting.  Long names are generally mixed-case ASCII, and the elements are
separated by spaces and may also contain internal spaces.  As with short
names, normal weight and monospace translate to nothing and are removed when
forming the long name.  An example long name might be ``Tsukurimashou Tenshi
no Kami PS''; note that that would correspond to the short name
``tsuku-mg-el-ps'' (as if ``Tsukurimashou Maru Extra Light PS'') because of
rewriting.  There are also some places, notably in the filenames of final
OTF fonts, where a long name is used with all the spaces and hyphens
removed.

There are also ``JK names,'' which are constructed just like long names but
with elements written in Japanese or Korean, and thus generally in non-ASCII
Unicode characters.  For Jieubsida fonts these should be in Korean (but the
occasional Japanese or English word may appear because of a
yet-unimplemented translation feature) and have spaces between the elements;
otherwise JK names are in Japanese and have no spaces.  The main use of JK
names is for populating the font metadata tables.  An example Japanese name
might be 「作りましょう天使の髪PS」, corresponding to ``Tsukurimashou
Tenshi no Kami PS.''

Since there is some information loss in transliteration to English, and many
of the names come from Japanese and Korean words anyway, it may be best to
think of the JK name as the truest name of the font for human purposes, even
though from the code's perspective the unrewritten Hamlog name is the true
name of the font and all others are derived from it.  The source of the
translation table for JK names is in txt/jnames.txt, which is processed by
Perl to change the UTF-8 into a hexadecimal form that Hamlog can handle, the
result going into hamlog/jnames.hl.

%%%%%%%%%%%%%%%%%%%%%%%%%%%%%%%%%%%%%%%%%%%%%%%%%%%%%%%%%%%%%%%%%%%%%%%%
%%%%%%%%%%%%%%%%%%%%%%%%%%%%%%%%%%%%%%%%%%%%%%%%%%%%%%%%%%%%%%%%%%%%%%%%

\blsection{『作りましょう』を作りましょう!}{Building Tsukurimashou}
\label{sec:building}

Please note that if you just unpack the Tsukurimashou distribution and look
in the ``otf'' subdirectory, you will find some ready-made font files there. 
If you are content with them, then those are the only files you need; you
can safely delete everything else in the package (maybe save the PDF
documentation files, if you want) and ignore this section.  These notes
on building Tsukurimashou are only for expert users who want the greater
control, wider style coverage, and intellectual challenge of doing custom
compilation.

Assuming you wish to embark on the adventure of compiling Tsukurimashou, you
will need at least the following:

\begin{itemize}
  \item A reasonably standard Unix command-line environment.  I use
    Slackware Linux.  Anything branded as ``Unix'' should work.  MacOS X
    might work.  Windows with Cygwin might work.  
  \item A standard C-language tool-chain (normally comes included with Linux).
  \item Perl.
  \item GNU Make 3.82 or later (non-GNU versions will not work;
    earlier versions will not work).
  \item Metapost (this comes with most \TeX/\LaTeX\ installations)
  \item A version of FontForge that actually works.
\end{itemize}

Other things that might also be useful include:

\begin{itemize}
  \item \XeLaTeX\ (needed for compiling the documentation).
  \item Expect (makes the build system work better---exactly why is
    explained in more detail in the warning message
    you will get if you don't have it).
  \item The KANJIDIC2 database (needed for the kanji coverage chart and
    a planned future fine-grained character subset selection feature).
  \item A supported Prolog interpreter (not necessary, but if one is
    available then the build system will use it and run more efficiently).
  \item A multi-CPU computer (the build system will by default detect and
    use all your CPUs; since some stages of compilation require a lot of
    processing, it's nice to have several).
  \item A checksum program, preferably sha256sum (used for some subtle
    build-system optimizations).
\end{itemize}

If you have a current version of \TeX{}Live, then you probably have
Metapost and \XeLaTeX\ already.  Note that it does have to be fairly
up-to-date.  Tsukurimashou now bundles its own version of the public domain
METATYPE1 code, so the dependency of previous versions
on the mtype13 distribution of that has been removed from the list.  In
preparing METATYPE1 code for bundling I discovered a previously unknown
dependency of previous versions on the t1asm package from t1utils; that has
now been bundled as well.

I mentioned a requirement for a version of FontForge ``that actually
works.'' FontForge is plagued by many bugs and numerical instabilities in
its spline geometry code, and stock, distributed ``stable'' versions tend to
hang and/or segfault when they are used to compile Tsukurimashou.  As of
March 2013, the mainline development version distributed as source seems to
work well.  However, a really full build of all weights of all styles
involves tens of thousands of invocations of FontForge and is likely to
cause at least one or two segfaults or infinite loops with any
widely-available precompiled binary package.

There is also an issue relating to the syntax of the AddExtrema() command in
the FontForge native scripting language; until recently, no version of
FontForge was capable of generating (under native-script control) extrema
that would pass FontForge's own validation checks, and so ``make check''
would almost always fail despite the fonts being in fairly clean
condition.  A patch I merged into FontForge's GitHub repository\footnote{I
never asked for write access, but this is what they do to people who
complain.} on September 30, 2012 adds an optional argument to
AddExtrema() to get around this issue, and the Tsukurimashou build system
will check for an appropriately modified version of FontForge and use it if
possible, otherwise generating a warning during configuration.  As of this
writing, I think the only way to get a fully patched version of FontForge is
to check it out of GitHub.

In the recent past there've been bugs relevant to Tsukurimashou fixed in the
interface to FreeType for rasterization; spline geometry; and hinting.  I
can't promise that using the latest patched version will be enough to keep
FontForge from crashing; what I do myself is go in with gdb after each
sufficiently annoying crash, find the line that is segfaulting, and try to
fix it.  I'm not sure that I have reported or recorded all the changes I've
made as a result of this procedure.  Because of all this,
Tsukurimashou's configure script will check that your FontForge has been
compiled with debugging symbols, and generate a warning message if not.  You
can either get a debuggable version, or disable the message if you feel like
living dangerously.

%%%%%%%%%%%%%%%%%%%%%%%%%%%%%%%%%%%%%%%%%%%%%%%%%%%%%%%%%%%%%%%%%%%%%%%%

\clearpage
\blsubsection{ビルドのシステム}{Build System}

The build system is based on GNU Autotools and should be reasonably familiar
to anyone who has compiled popular free software before.  Run ``./configure
-{}-help'' for a description of available configuration options; run
``./configure'' possibly with other options as described in the help message
to set up the system the way you want it; then once it is configured to your
liking, run ``make'' to build it.  The completed font files end up in a
directory called ``otf'' in the distribution tree. A few ready-made ones
should be there already when you unpack the distribution, for the benefit of
the probable majority of users who can't do their own compilation.

If you run ``make install'' it will attempt to install the fonts and
documentation files in sensible locations, but it's not really customary for
fonts to be installed like other software, and you may be better off simply
taking the compiled files from the otf directory and doing with them
whatever you would normally do with fonts instead of using the
automated install.  At this time there is no \TeX-specific font installation
support, though that might be a nice feature for me to build in the future. 
The install target uses the ``-{}-prefix'' and similar options to configure
in a reasonably obvious and standard way.

If you have KANJIDIC2, place it (in the gzipped XML form in which it is
distributed) in the build directory, the doc subdirectory off of the build
directory, your system's dict or share/dict directories, possibly under
/usr or /usr/local or your configured installation prefix, or your home
directory or ``\textasciitilde/dict''; or put it
somewhere else and tell configure where with the ``-{}-with-kanjidic2''
option.  Although what's used by Tsukurimashou is factual information not
subject to copyright, and KANJIDIC2 (which also includes creative content
that might be subject to copyright) is distributed freely, it is not
distributed under a GPL-compatible license and so for greater certainty I'm
avoiding including KANJIDIC2 or any data extracted from it in the
Tsukurimashou distribution.  What you get if you have KANJIDIC2 is any
features that depend on knowing the grade levels and similar properties of
kanji characters---such as the chart showing how much of each grade has been
covered, which is important for my own development efforts in planning what
to design next and knowing when to release.

The build system for Tsukurimashou is fairly elaborate; it may seem like
overkill, but given that I expect to run this myself several times a day for
at least a couple years, it's important that it should be pleasant for me and
efficient in time consumption.  Thus it defaults to ``silent'' build rules,
uses pretty ANSI colours, and does a bunch of complicated checks on whether
file contents (not just modification times) have changed in order to avoid
expensive dependency rebuilds unless those are really needed, while
triggering them automatically when I add new source files.

One important tip is that if things are failing, you can add ``V=1'' to the
``make'' command line, as in ``make V=1 pfbtmp/tsuku-kg-00.pfb'' to
temporarily disable the silent build rules and see what's going on.  Note no
hyphen, because this is a variable setting instead of an option, and you
will probably also want to specify a target filename, which implicitly
overrides the default ``-j'' multi-CPU option.  In version 0.6, Expect was
rendered less necessary by the replacement of mtype13 with a homegrown and
less chatty Perl script; but the support remains in place to handle \TeX's
lesser jubilations.

You can also add ``KEEPTMP=1'' to prevent deletion of the temporary
directories created while running Metapost.  This feature is primarily
useful for debugging the pfb-generating scripts themselves.

Several options to configure share a common argument syntax under which you
can specify a comma-separated list of commands to include and exclude
subsets of things.  This syntax is actually translated into Hamlog code and
evaluated by the logic-programming system during build.  It may be easier
to understand by example than to explain rigorously, but here is an attempt
at a precise explanation: a specification consists of a sequence of commands
separated by commas or spaces.  Each command either adds or subtracts some
subset of the things being configured, to or from the currently selected
subset which is initially empty.  A command consists of one or more
alphanumeric tags representing subsets, separated (if more than one) by
ampersands or colons, which mean ``AND,'' and optionally prefixed by plus,
minus, or exclamation point.  If the prefix is ``-'' or ``!'' then the AND
of all the tags will be \emph{subtracted} from the current selection;
otherwise it will be \emph{added}.  The allowed values of the tags depend on
which option we are considering, but typically include ``all'' and ``none.''

For example, the specification ``mushroom\&purple'' would select purple
mushrooms.  The specification ``fungi,-mushroom\&purple'' would select all
fungi that are not purple mushrooms.  The specification
``mushroom,-red,tomato'' would select all mushrooms that are not red as well
as all tomatoes \emph{including red tomatoes}; tags are evaluated left to
right and each one overrides the results of all previous tags.

The ``-{}-enable-chars'' option uses the above syntax to describe which
characters (Unicode code points extended by internal private-use allocations
for some unencoded glyphs) should be included in the fonts.  The default is
``all,'' meaning all characters defined in the source code will be included
in the fonts.  Other tags currently supported are ``none,'' ``ascii,''
``latin1,'' ``mes1,'' and families of tags named like ``page12,''
``uni1234,'' and ``u123ab,'' with lowercase hexadecimal digits, which
correspond to 256-character blocks of Unicode code points and to individual
code points.  For example, the specifier ``all,-page02'' would include every
character it can except none in the range U+0200--U+02FF; the specifier
``uni0041'' would create very small fonts containing only the uppercase A
from the ASCII range.  The default is ``all.'' It is planned that in some
future version this option should also be able to use KANJIDIC2 data to
allow selection of kanji characters by criteria like school grade level, and
some support for that exists already, but it probably doesn't work yet.

The ``-{}-enable-ot-features'' option works the same way for OpenType
features; note that OpenType features may also be automatically and silently
disabled in whole or in part, overriding this setting, if you have used the
previous setting to disable characters necessary to implement the features. 
For instance, you can't have OpenType contextual substitution support of
fractions or enclosed numerals if you have disabled the numeral
characters---though you \emph{can} build a font with enclosed and not
regular numeric glyphs, because glyphs are mostly independent of each
other.\footnote{Mostly.  In the case of white-on-black reversed glyphs and
some fractions precomposed by FontForge instead of by METATYPE1, you
must include all the parts that FontForge will assemble in order to get the
combined glyph made by assembling those parts.  This is a sufficiently
arcane scenario that the build system will not check for it.}
This automatic disabling may not be perfectly clean, either; in some cases
disabling a character might not disable the feature code that mentions it,
and in that case FontForge may create an empty glyph for the character even
though you disabled it.  That shouldn't happen unless you attempt something
very unusual and non-standard, however. Currently supported tags are
``all,'' ``none,'' and a four-character tag corresponding to the OpenType
name of each feature that exists, and a few that don't.  See
Section~\ref{sec:opentype-features} for a detailed list.

The ``-{}-enable-styles'' option allows similar selection of type styles,
For instance, ``tsuku\&kg,-el'' would enable all
Tsukurimashou Kaku fonts, in all weights and both monospace and
proportional, \emph{except} no ``Extra Light'' fonts.  The atomic tokens for
this selection include the standard ``all'' and ``none,'' and the short
name tokens and Hamlog atoms for all the families, styles, weights, and
spacings currently supported.  See Section~\ref{sec:fontnames} for more
about the naming of fonts, and note that these are the pre-rewriting names,
so that for instance Anbiruteki might be selected via ``tsuku\&mg\&eq.''

The ``-{}-enable-parasites'' option allows this kind of include/exclude
selection for parasite packages (see next subsection).  The supported tokens
are ``all,'' ``none,'' and the names of the parasites, currently
``beikaitoru,'' ``genjimon,'' ``idsgrep,'' and ``ocr.''
The default is ``none.''

The ``make dist'' target defaults to building a ZIP file only, instead of
GNU's recommended tar-gzip.  This decision was made in order to be
friendlier to Windows users, who tend to have seizures when confronted with
other archive formats, and \TeX\ users, who are accustomed to ZIP as well. 
However, the makefiles should also support many other formats via ``make
dist-gzip'' and similar.

I am not confident that ``make dist'' will really include everything it
should if you have disabled optional features with the above options to
configure.  Please ``make me one with everything,'' as the Buddhist master
said at the hamburger stand, before trying to build a distribution.

The ``make clean'' target and its variations should make things pretty
clean, now that we're attempting to make the system pass ``make distcheck''
before each release.  Use them with caution, however.

There is a ``make check'' target, which will run FontForge's fontlint
program on all the OTF files.  This is a very stringent test and it's quite
likely that fonts will fail despite really being in pretty good shape;
fontlint flags a lot of things that I am not convinced are actually errors
at all; so if you run it and get nothing but complaints, don't panic.  At
the very least, you will need a FontForge with my all-extrema patch
(discussed in the previous section) to have much hope of building fonts that
can pass fontlint.  Running ``make check'' at the top will also recursively
run ``make check'' on any parasite packages that have been enabled, and the
parasite package test suites (in particular, the one for IDSgrep) may be of
more practical use than the main Tsukurimashou fontlint tests.

If you try to do ``make distcheck,'' it will fail if any of the tests fail;
and the tests probably \emph{will} fail because of the bugs in FontForge
spline geometry.  A workaround, used in my Tsukurimashou 0.8 distribution
packages, is to override the options for \texttt{fontlint.pe} as follows:
\begin{verbatim}
make DISTCHECK_CONFIGURE_FLAGS='FONTLINT_FLAGS="-w 2,3,5,23"' distcheck
\end{verbatim}

The \texttt{DISTCHECK\_CONFIGURE\_FLAGS} variable specifies additional flags
(actually a command-line fragment) that ``make distcheck'' will pass to
``configure.''  Then inside that, the \texttt{FONTLINT\_FLAGS} variable
specifies flags to pass to the font integrity checker.  The codes 2, 3, 5,
and 23 are the errors that I wasn't able to correct with the version of
FontForge installed on my development system as of the time of the release.

Don't bother with ``-{}-disable-dependency-tracking''; that is an
Autotools thing meant for much larger and more softwary packages.  It
applies only to code in languages supported by Autotools, which at
present means only the C kerning program whose dependency is trivial. 
The dependency tracking for METATYPE1 code is completely separate,
unaffected by this option, and trying to disable it would be a bad
idea.

Autotools encourages the use of a separate build directory, with the
sources remaining inviolate elsewhere, but that is not really
recommended for Tsukurimashou.  I try to make it pass ``make
dist-check'' right before each release, which implies making separate
build directories work, but if you are building Tsukurimashou from a
checked-out SVN version, it may or may not work.  It's safer
to build right in the main directory.  Even if VPATH builds work, they
are only intended for the case of having an untouched set of sources in
one directory and a build in another.  If you try to do the overlay
thing, with modified versions of some source files in your build
directory, it is unlikely to work, because of the large amount of
bolted-on filename and path logic that doesn't go through GNU Make for
name resolution.

If you look in the source of the build system, specifically files like
configure.ac, you'll see that I did a whole lot of work ripping out sections
of Autotools that were designed for installations of executable software. 
GNU standards require the definition of a ridiculous number of different
installation directories, almost none of which are applicable to a package
of this type, and I took out most of the support for those to reduce the
cognitive load for users who would otherwise have to think about their
inapplicability.  This package doesn't install any executables, libraries, C
header files, or similar, at all.  (Although some of the parasites may...)
Cross-compilation and executable name munging were removed for the same
reason; some C programs are compiled for kerning and Type 1 opcode assembly,
but these are only meant to run on the host system during build, with all
the installable files being architecture-independent.  The hacking I did on
Autotools means that if you modify the build system such that you would be
re-running Autotools, it's likely to break unless your version of Autotools
is close to the 2.65 version I used.  The configure script will try to
detect such a situation and warn you.

The design of this build system was influenced by Peter Miller's interesting
article ``Recursive Make Considered Harmful,'' available at
\url{http://miller.emu.id.au/pmiller/books/rmch/}.  The fonts as such are
all built from a single large Makefile.  However, in the current version
there is also some use of recursive Make and recursive Autotools to support
the parasite packages (next section), should those be enabled (by default,
they are not).  Some notes on the design of the build (prior to the
introduction of parasites) are in my Web log article at
\url{http://ansuz.sooke.bc.ca/entry/226}.  I will probably write a similar
article about the parasite mechanism at some point in the future.

%%%%%%%%%%%%%%%%%%%%%%%%%%%%%%%%%%%%%%%%%%%%%%%%%%%%%%%%%%%%%%%%%%%%%%%%

\blsubsection{寄生パッケージ}{Parasite packages}\label{sub:parasites}

Some of the technology developed to support Tsukurimashou has more general
applicability.  For instance, the stripped-down version of METATYPE1 bundled
with Tsukruimashou may be useful for other fonts that build from
METAFONT-language source files to produce modern vector file formats.  It
would be nice to be able to manage these kinds of associated packages in a
way that maintains their link to Tsukurimashou---so that for instance
updates in the shared code will propagate to all packages that use it---but
nonetheless allows the other packages to be distributed as separate entities
to users who don't want a copy of the entire Tsukurimashou system.  There
are relevant side issues, too, such as a hope for attention directed at one
package to attract interest to other packages.

One possible approach might be to attempt to factor out code that will be
shared into some kind of ``Tsukurimashou library'' package that could then
be distributed separately and made a dependency of all the other packages
that use it.  A problem with doing that is that which code is or isn't
shared will vary a lot due to the wide range of different packages that
might want to share some part of the system.  Instead, the approach taken as
of Tsukurimashou 0.7 is to introduce what I call ``parasite packages.''
These are sub-packages for Autotools purposes.

Each parasite appears in its own subdirectory of the full Tsukurimashou
Project distribution tree or SVN checkout.  By default, they are distributed
in distribution packages of the full project, but not built by a top-level
build.  They can be activated using the -{}-enable-parasites option to the
top-level configure script, but each one in its own subdirectory is also a
potentially standalone Autotools package, with its own configure script and
capable of building a standalone distribution of itself with the ``make
dist'' target.  Parasites will generally look for a Tsukrimashou Project
build in the parent directory and connect to it if one is found.

The current parasites are Beikaitoru, Genjimon, IDSgrep, Kleknev, and OCR. 
Each of these has its own detailed documentation, which see; the notes here
are just a summary to give some idea of what the packages are all about. 
Most of these existed in some form prior to the introduction of the parasite
build mechanism, and in some cases vestiges still exist of their old, less
systematic, ways of connecting to Tsukurimashou.  In the future, the
connections will be more streamlined and normalized.

Note that parasite packages generally have their own version numbers which
are not necessarily synchronized to specific Tsukurimashou releases; in some
cases a Tsukurimashou release will contain a pre-release or even a broken
version of a parasite.  But it should usually be reasonably safe to mix and
match parasites and Tsukurimashou versions of reasonably similar vintage. 
Parasite packages may also have their own license terms, but must always
have license terms that allow them to be distributed within the GPL3
top-level project distribution.

\begin{description}
  \item{Beikaitoru} A modern revival of the historic plotter fonts
    ``digitalized'' [sic] by A.\ V.\ Hershey of the U.\ S.\ Naval
    Weapons Laboratory during the 1960s, and subsequently distributed by the
    U.\ S.\ National Bureau of Standards and on Usenet.  This version
    includes the Japanese character range omitted from most other revivals.
    GPLv3 with font embedding exception.

  \item{Genjimon} A set of standalone fonts for the Genjimon symbols, which
    in the main Tsukurimashou fonts occupy the private-use code points
    U+F17C1 to U+F17F6.  In the standalone fonts, they are mapped into the
    ASCII range for easier access.  GPLv3 with font embedding exception.

  \item{IDSgrep} A program for searching databases of tree structures stored
    in ``extended ideographic description sequence'' (EIDS) format.  This is
    intended for searching kanji databases according to spatial structure
    using a powerful query syntax inspired by regular expressions.  The need
    for such queries came up during Tsukurimashou development, but the
    package may well be useful in other contexts as well.  Includes code to
    interface to several different character dictionaries, including
    structural data extracted from Tsukruimashou.  GPLv3.

  \item{Kleknev} A coarse-grained profiler for build systems, such as the
    Tsukurimashou build.  This is still rather experimental.  It basically
    consists of a wrapper program that Make can invoke instead of
    \texttt{/bin/sh}, which will (when told to do so by a special
    environment variable) collect a bunch of performance information that
    can be analysed by a separate reporting program to provide some insight
    into where all the time goes during a long build.  See my Web log
    article at 

  \item{OCR} Freely usable vector versions of the standard OCR A and OCR B
    fonts, based on the work of Tor Lillqvist, Richard B.\ Wales, and
    Norbert Schwarz.  License terms vary, but the fonts should all be
    freely usable.
\end{description}

%%%%%%%%%%%%%%%%%%%%%%%%%%%%%%%%%%%%%%%%%%%%%%%%%%%%%%%%%%%%%%%%%%%%%%%%

\blsubsection{ツールの文書化}{Tool documentation}

There are a number of scripts included in the \texttt{tools/} subdirectory. 
Most of these are intended only to be invoked automatically by the build
system, but two intended to be invoked by hand are described below.

\begin{itemize}

\item \texttt{autodep}: automatically maintain the header inclusion lines in
METAFONT files, so that each file will include the files that define macros
it uses, and only those.  Files intended for use with \texttt{autodep}
should include a comment line consisting of the case-sensitive text
``\texttt{\%\ AUTODEPS}'' and any automatically-maintained input lines
following that, terminating with a blank line.  When the tool is invoked
with the command line ``\texttt{tools/autodep mp/ mp/*.mp}'' from the root
of the source tree, it will scan all the METAFONT-language sources, find
definitions of macros, and where necessary rewrite the blocks of includes
tagged with the special comment line in order to ensure that every file
includes the definitions of all needed macros.  There are a number of
exceptions and special cases included to make it do the right thing with
macros defined in multiple places as a result of style overrides, and so on.

\item \texttt{udcpyright}: check that all years in which a file was modified
are included in its copyright notice.  This assumes one has an SVN checkout
in the current directory---intended to be a checkout from my private SVN
repository, but it would probably work with one from the SourceForge
repository as well.  It scans all SVN-controlled files for anything that
looks like the project's standard copyright notice, and then reports on any
files for which the set of years mentioned in the notice does not match the
set of years in which SVN has log entries for that file.  The idea is that
once a modified version of the file has been checked into SVN in a given
year, it has been ``published'' in that year and the year should be
mentioned in the copyright notice; but when a new year comes we don't want
to just go through and modify all files solely to insert another year in the
copyright notice.  By running this every so often, and manually updating the
notices when appropriate, the hundreds of copyright notices in the project
should painlessly converge on correctly reflecting the years in which files
were actually changed.

\end{itemize}

For completeness, here is a summary of the functions of all the other
scripts in the \texttt{tools/} subdirectory.  Most of them are written in
Perl.

\begin{itemize}

\item \texttt{add-flog}: attach the ``font log'' file to a FontForge
SFD-format font; this requires re-encoding it to UTF-7 format.

\item \texttt{livecode}: chase file inclusions and determine all the
``live'' (that is, actually invoked) macro definitions in a list of
METAFONT-language files.  The output of this is not (because lines may
appear in the wrong order) actually valid code one could run; but the output
of \texttt{livecode} is intended to have the property that its checksum will
almost always change when changes in the input necessitate recompilation,
and will seldom change otherwise.  Therefore the build system can use the
checksum of \texttt{livecode}'s output for fine-grained change detection, to
skip recompiling some files that (under GNU Make's timestamp-based scheme)
would otherwise appear to require recompilation.  As a special feature,
intended for maintainer use only, \texttt{livecode} also supports a ``find
dead code'' mode.  If its first command-line argument is ``\texttt{-d},''
then instead of its usual output, it will print a list of names of macros it
thinks are \emph{not} invoked.  Running the dead code check against the
entire code base may be a way to find superfluous METATYPE1 infrastructure
that could be removed.

\item \texttt{make-ass}: generate the \texttt{assemble-font.pe} script,
which in turn does most of the work of assembling 256-glyph subfonts to
create full-coverage OTFs.

\item \texttt{make-book}: make decisions on how to split the
multi-thousand-page proof document into smaller ``books.''  The output
is a set of \TeX\ files which then get processed further to generate the
actual PDFs.

\item \texttt{make-cfghl}: translate \texttt{configure} command-line strings
into Hamlog for inclusion in \texttt{config.hl}, which then feeds back to
control the build system.

\item \texttt{make-cover}: generate colour covers for the proof books.

\item \texttt{make-eids}: scan the proof files
and generate an EIDS-format dictionary of character decompositions.

\item \texttt{make-fea}: perform text substitutions on ``feature source
code'' (fsc) files to generate Adobe-format ``feature'' (fea) files
expressing code-like font metadata.

\item \texttt{make-flog}: collect a bunch of debugging information and
generate a font log file (see \texttt{add-flog} above).

\item \texttt{make-gpos}: compute information for GPOS features ``mark'' and
``mkmk'' by scanning the proof files (to determine anchor locations and
compatibility) and some generated FontForge scripts (to determine bearing
adjustments made during kerning).

\item \texttt{make-hglpages}: generate METAFONT source code for the hangul
precomposed syllable subfonts, which are all identical except for their
starting indices and generated from a template.

\item \texttt{make-kchart}: generate \TeX\ code for the kanji coverage
chart.

\item \texttt{make-kddata}: extract KANJIDIC2 data in Hamlog form.

\item \texttt{make-name}: generate a font's NAME metadata table.

\item \texttt{make-proof}: generate \TeX\ code for proofs and pretty-printed
source code.

\item \texttt{mp2pf}: stripped-down translation from Awk to Perl of a
similar program in METATYPE1; this invokes Metapost in the appropriate way
to generate something that can be processed into a Postscript font, and then
does process the result into a Postscript font.  Note that hinting support
has been \emph{removed} from this code, and the resulting Postscript fonts
may not really be good or entirely format-valid; they are only intended as a
way of getting the outline data into FontForge for later stages of
processing.

\item \texttt{mpdep}: generates a list of the dependencies of a
METAFONT-language file (or a set of them), for automated Makefile generation. 
Compare with \texttt{autodep}, which actually edits the files to include the
other files they should based on the macros they use; \texttt{mpdep} only
looks at what other files each file does include, and reports that.

\item \texttt{progress}: invoked during build to give progress reports and
completion-time estimates.

\end{itemize}

%%%%%%%%%%%%%%%%%%%%%%%%%%%%%%%%%%%%%%%%%%%%%%%%%%%%%%%%%%%%%%%%%%%%%%%%

\blsubsection{ビルドのシステムの図示}{Build system diagrams}

This section is still under construction, but it is
planned to be a graphical summary of what is going on in Tsukurimashou's
rather complicated Makefile.  Figure~\ref{fig:diagram-legend}
describes the symbols used, and Figure~\ref{fig:build-fonts} shows the
process by which fonts are built.  Proportional fonts go through additional
processing to generate the kern tables and so on, as shown in
Figure~\ref{fig:kern-fonts}.

Be aware that these diagrams are meant to clarify, not to formally describe,
the code; this is \emph{not} a formal modelling language like UML, and the
semantics of the symbols are not necessarily entirely consistent.  Also left
out of the diagrams are language interpreters like Perl, and a great deal of
information flow to and from the main Makefile itself.  It passes
configuration information, mostly in command-line arguments, to pretty much
all parts of the system.

\tikzset{distributed file/.style={draw,rectangle}}
\tikzset{generated file/.style={}}

\tikzset{dataflow arrow/.style={thick,arrows=-angle 45}}
\tikzset{creates arrow/.style={thick,arrows=-triangle 60}}
\tikzset{modifies arrow/.style={thick,arrows=-o}}
\tikzset{sometimes arrow/.style={thick,arrows=-angle 45,dashed}}

\tikzset{my note/.style={align=left,anchor=west}}

\tikzset{grouping box/.style={dashed,thick}}

\begin{figure}
\centering
\begin{tikzpicture}
  \node[distributed file] at (0,0) {foo.bar};
  \node[my note] at (3,0) {\small foo.bar comes with the distribution};
  \node[generated file] at (0,-1) {naninani.dat};
  \node[my note] at (3,-1) {\small naninani.dat is generated during build};
  \node[generated file] (a) at (-1,-2) {A};
  \node[generated file] (b) at (1,-2) {B};
  \path (a) edge[dataflow arrow] (b);
  \node[my note] at (3,-2) {\small A is input for B};
  \node[generated file] (c) at (-1,-3) {C};
  \node[generated file] (d) at (1,-3) {D};
  \path (c) edge[creates arrow] (d);
  \node[my note] at (3,-3) {\small C creates D};
  \node[generated file] (e) at (-1,-4) {E};
  \node[generated file] (f) at (1,-4) {F};
  \path (e) edge[sometimes arrow] (f);
  \node[my note] at (3,-4) {\small E sometimes creates F};
%  \node[generated file] (g) at (-1,-5) {G};
%  \node[generated file] (h) at (1,-5) {H};
%  \path (g) edge[modifies arrow] (h);
%  \node[my note] at (3,-5) {\small G modifies H in place};
\end{tikzpicture}
\caption{Legend for build system diagrams}
\label{fig:diagram-legend}
\end{figure}

\begin{figure}
\centering
\begin{tikzpicture}
  \node[generated file] (starsum) at (-4,0) {sum/*.sum};
  \node[distributed file] (starmp) at (0,0) {mp/*.mp};
  \node[generated file] (starpfbtmp) at (0,-2) {pfbtmp/*.pfb};
  \node[generated file] (starprf) at (2.8,-2) {prf/*.prf};
  \node[distributed file] (rmope) at (0,-4) {pe/rmo.pe};
  \node[generated file] (starpfb) at (0,-6) {pfb/*.pfb};
  \node[generated file] (assemblefont) at (-3,-8.3) {pe/assemble-font.pe};
  \node[generated file] (starotf) at (0,-10) {otf/*.otf};
  \node[generated file] (starsfd) at (-6,-10) {sfd/*.sfd};
  \node[generated file] (charslst) at (-5.6,-1) {chars.lst};
  \node[distributed file] (makeass) at (-3,-2) {make-ass};
  \node[generated file] (tsukunam) at (-4.3,-5) {tsukurimashou.nam};
  \node[distributed file] (tsukufsc) at (-7,-0) {fea/*.fsc};
  \node[distributed file] (makefea) at (-7,-4) {make-fea};
  \node[generated file] (tsukufea) at (-7,-6) {fea/*.fea};
  \node[distributed file] (jnamestxt) at (-9.5,-1) {txt/jnames.txt};
  \node[generated file] (jnameshl) at (-9.5,-3) {hamlog/jnames.hl};
  \node[distributed file] (makename) at (-9.5,-5) {make-name};
  \node[generated file] (namefea) at (-9.5,-7) {fea/*-name.fea};
  \node[generated file] (flogged) at (2.2,-7) {sfd/flogged.sfd};
  \path (starmp) edge[creates arrow] (starsum);
  \path (starsum) edge[dataflow arrow] (starmp);
  \path (starmp) edge[creates arrow] (starpfbtmp);
  \path (starmp) edge[creates arrow] (starprf);
  \path (starpfbtmp) edge[dataflow arrow] (rmope);
  \path (rmope) edge[creates arrow] (starpfb);
  \path (starpfb) edge[dataflow arrow] (assemblefont);
  \path (assemblefont) edge[sometimes arrow] (starotf);
  \path (assemblefont) edge[sometimes arrow] (starsfd);
  \path (charslst) edge[dataflow arrow] (makeass);
  \path (starmp) edge[dataflow arrow] (makeass);
  \path (makeass) edge[creates arrow,bend left] (assemblefont);
  \path (makeass) edge[creates arrow] (tsukunam);
  \path (tsukunam) edge[dataflow arrow] (assemblefont);
  \path (tsukufsc) edge[dataflow arrow] (makefea);
  \path (makefea) edge[creates arrow] (tsukufea);
  \path (charslst) edge[dataflow arrow] (makefea);
  \path (tsukufea) edge[dataflow arrow] (assemblefont);
  \path (jnamestxt) edge[creates arrow] (jnameshl);
  \path (jnameshl) edge[dataflow arrow] (makename);
  \path (makename) edge[creates arrow] (namefea);
  \path (namefea) edge[dataflow arrow] (assemblefont);
  \path (flogged) edge[dataflow arrow] (assemblefont);
\end{tikzpicture}
\caption{Building the fonts themselves.  Checksums in sum/ provide
fine-grained change tracking on the Metapost files in mp/, which define
Postscript ``page'' fonts each covering up to 256 code points.  Those are
processed to remove overlaps (RMO) and end up in pfb/.  The make-ass script
also scans the Metapost files to get a list of defined code points and some
other metadata; its main output is a FontForge
script that assembles the page fonts into full-coverage OpenType or
FontForge native fonts.  Non-proportional fonts become OpenType and are
finished here; proportional fonts become FontForge native and are processed
further.  The assemble-font script also makes use of per-family feature
files from make-fea (defining things like glyph composition) and per-font
feature files from make-name (defining metadata table contents).  The
``proof'' files in prf/ are generated as a side effect of compiling the
PostScript page fonts, and pass information about glyph construction to
both the automatic documentation generator and the GPOS table generator.}
\label{fig:build-fonts}
\end{figure}

\begin{figure}
\centering
\begin{tikzpicture}
  \node[generated file] (starsfd) at (4,-8) {sfd/*.sfd};
  \node[distributed file] (bdfpe) at (0,-10) {pe/bdf.pe};
  \node[generated file] (tempbdf) at (0,-12) {bdf/*.bdf};
  \node[distributed file] (kernerc) at (-2,-13) {kerner.c};
  \node[generated file] (kerner) at (0,-14) {kerner};
  \node[generated file] (kerntmp) at (0,-16) {fea/*-ktmp.fea};
  \node[generated file] (bearingpe) at (3,-15) {pe/*-bearing.pe};
  \node[distributed file] (kernfont) at (6,-19) {pe/kern-font.pe};
  \node[distributed file] (getbearpe) at (8,-10) {pe/getbearings.pe};
  \node[generated file] (bearings) at (8,-12) {*.bearings};
  \node[generated file] (starprf) at (10,-13) {prf/*.prf};
  \node[distributed file] (makegpos) at (8,-14) {make-gpos};
  \node[generated file] (gposfea) at (8,-16) {fea/*-gpos.fea};
  \node[generated file] (starfea) at (-2,-17) {fea/*.fea};
  \node[generated file] (kernfea) at (2,-18) {fea/*-kern.fea};
  \node[generated file] (starotf) at (10,-19) {otf/*.otf};
  \path (starsfd) edge[dataflow arrow] (bdfpe);
  \path (bdfpe) edge[creates arrow] (tempbdf);
  \path (kernerc) edge [creates arrow] (kerner);
  \path (tempbdf) edge[dataflow arrow] (kerner);
  \path (kerner) edge[creates arrow] (kerntmp);
  \path (kerner) edge[creates arrow] (bearingpe);
  \path (starsfd) edge[dataflow arrow] (getbearpe);
  \path (getbearpe) edge[creates arrow] (bearings);
  \path (bearings) edge[dataflow arrow] (makegpos);
  \path (bearingpe) edge[dataflow arrow] (makegpos);
  \path (makegpos) edge[creates arrow] (gposfea);
  \path (starprf) edge[dataflow arrow] (makegpos);
  \path (kerntmp) edge[creates arrow] (kernfea);
  \path (starfea) edge[creates arrow] (kernfea);
  \path (gposfea) edge[creates arrow] (kernfea);
  \path (kernfea) edge[dataflow arrow] (kernfont);
  \path (bearingpe) edge[dataflow arrow] (kernfont);
  \path (starsfd) edge[dataflow arrow,bend left=15] (kernfont);
  \path (kernfont) edge[creates arrow] (starotf);
\end{tikzpicture}
\caption{Additional processing for proportional fonts.  The SFD file
containing the font outlines is converted to a BDF bitmap font, which feeds
the kerner program.  That generates a ``ktmp'' feature file with the actual
kerning data, and a FontForge script to apply the chosen bearings.  At
right, the make-gpos script reads the proof files to determine accent anchor
locations.  But since those must be adjusted when the bearings are set, it
also needs ``before'' bearing information extracted by getbearings, and
``after'' obtained by reading the bearings script from kerner.  The feature
files for kerning, mark composition (GPOS), and family features are all
combined and applied by the kern-font script, which runs the bearings script
as a subroutine, finally producing a complete OTF font.}
\label{fig:kern-fonts}
\end{figure}

%%%%%%%%%%%%%%%%%%%%%%%%%%%%%%%%%%%%%%%%%%%%%%%%%%%%%%%%%%%%%%%%%%%%%%%%

\blsubsection{ハムログ}{Hamlog}

Evaluating the consequences of build options like ``-{}-enable-chars''
requires doing a certain amount of logical inference for which shell scripts
are not well-suited.  It might be possible to get GNU Make to do the
necessary computations, inasmuch as it's quite programmable, already
required to build Tsukurimashou, and fundamentally a logical inference
engine at heart.  But that would probably involve creating many tiny files
corresponding to logical propositions, which would waste space and cause
other problems on some filesystems.  A more elegant approach would be to use
a real logic programming system, i.e.\ Prolog---which happens to be one of
my research interests.  But I didn't want to create a dependency on a Prolog
interpreter because I think users will object to that; the existing
dependencies of this package are already hard enough to sell.  I also didn't
want to bundle a Prolog interpreter, even though good ones are available on
permissive licensing terms, because of the file size and build-system
complexity consequences of bringing a bunch of compiled-language software
into the Tsukurimashou distribution.

The solution:  Tsukurimasho's build system will look for Prolog, and use it
if possible.  But the package also ships with something called Hamlog, which
is a toy Prolog-like logic programming system written in Perl.  (A ham
is like a pro, but less so.)  If the build system can't find a Prolog
interpreter, it will use Hamlog instead.  Hamlog is slow, and internally
horrifying, but it works in this particular application.  It is not
particularly recommended for any other applications.

The configure script looks for SWI-Prolog,
ECL$^\textrm{i}$PS$^\textrm{e}$-CLP, and GNU Prolog.  If one of these can be
safely detected, it will be used.  ECL$^\textrm{i}$PS$^\textrm{e}$-CLP has
the issue that it shares a name with a widely-used programmer's IDE, so it
is not safe for the configure script to actually execute an executable
called ``eclipse'' if it finds one.  If something that might be
ECL$^\textrm{i}$PS$^\textrm{e}$-CLP is detected, the configure script puts
up a warning and the user is free to enable it explicitly.

The rest of this subsection can and probably should be skipped by
anyone who isn't both a Perl and a Prolog hacker.

Still with me?  The way Hamlog works is sort of fun and so I'm going
to spend a few pages describing it for those who are interested, if only as
an example of something you should Not Try At Home.  The idea is to use
regular expressions for the operation of finding a clause whose head matches
the current goal.  Hamlog reads its program into a Perl hash, where the key
is the functor and arity (like ``foo/3'') and the value is a
newline-separated pile of clauses in more or less plain text.  When it tries
to satisfy a goal, it takes the goal (which starts out as plain text) and
converts it to a regular expression that will match any appropriate lines in
the database.  Variables in the goal turn into wildcard regular expressions;
ground terms turn into literal text; and then when there's a match,
parenthesized sub-expressions are used to extract the body of that clause.

It is because of the use of regular expressions that Hamlog doesn't do
compound terms, and in turn is likely not Turing-complete (though I haven't
thought carefully through all the possibilities of using recursive
predicates or atom splitting to build stack-like data structures).  As
all the world knows, it is impossible to write a regular expression to match
balanced parentheses.  Current versions of Perl actually bend the theory
with experimental extensions that do allow the matching of balanced
parentheses, so that in a certain important sense Perl regular expressions
\emph{are not regular expressions anymore at all}, but even I am not quite
twisted enough to actually deploy such code.  Things in Hamlog that look
like compound terms (such as the sub-goals in a clause body) are handled as
special cases; but the point is that arguments to a functor that will be
used as a goal have to be either atoms or variables.  This also means Hamlog
doesn't do Prolog syntactic sugar things that expand to compound terms, such
as square brackets for lists.

Once there's a match, it does string substitution on the matching head, the
current partially-completed goal, and the body, to get a new modified body
for the clause, taking into account any variables assigned by the head
match.  The new clause body gets substituted into the current
partially-completed goal (which is a string) as a replacement for the head
that just matched.  So the partially-completed goal is a sort of stack of
comma-separated heads that grows from right to left and implicitly contains
all the assigned variables.

Because of the simplistic way variables are given their values, it is
dangerous to use the same variable more than once in the same head, so
constructions like ``foo(X,Y,X)'' should not be used.  If you want to do
that you should instead write ``foo(X,Y,Z):-~=(X,Z).'' Note the non-sugary
use of ``='' as a functor, since the more common infix notation isn't
supported.  Note also that there should be a space between ``:-'' and ``='';
Hamlog doesn't require that but it may reduce the likelihood of parsing
problems should the same code be run on interpreters other than Hamlog.

Variables in clause bodies are renamed once (using the clause serial
number), when the clauses are loaded; as a result if the same clause body
gets expanded a second time while variables from its earlier expansion are
still unassigned, there could be trouble.  This is not a very likely
scenario, but it's worth mentioning.

Clauses in the database have serial numbers; and when a choice point goes on
the stack, the serial number of the clause at which it matched is
part of the record on the stack.  Then if the interpreter backtracks to
re-satisfy a clause, it writes the regular expression in such a way that it
can match all and only serial numbers greater than the last place it
matched.  Creating an ``integer greater than N'' regular expression was
surprisingly difficult---it's a simple enough concept but there are several
cases that all must be handled properly or weird bugs turn up.

Syntax is simplified from Prolog.  Variables start with an uppercase letter
or an underscore and may contain uppercase alphanumerics and underscores. 
Atoms start with a lowercase letter or numeric digit and may contain
lowercase alphanumerics and underscores.  For Prolog compatibility, atoms
starting with digits should not contain anything other than digits, and the
only atom starting with zero that should be used is zero itself; but Hamlog
doesn't care about those points.  Things containing a mixture of
upper- and lowercase alphabetic characters should not be used.  The special
tokens ``!'' and ``='' are technically treated as atoms too, but you should
only use them in their typical meanings of cut and unification, and ``=''
should only be used with the general prefix syntax applicable to all
functors, not as an infix operator (see above).

Variable names starting with, and especially the unique variable name
consisting entirely of, an underscore, are not special in Hamlog.  Beware,
that means ``foo(\_,\_)'' contains only one variable occurring twice when
interpreted as Hamlog, not two distinct variables as in Prolog, and it
violates the ``only one appearance of each variable in a head'' guideline. 
The unique variable name consisting of one underscore is probably best
avoided entirely.  But it may be desirable to use variable names starting
with underscores anyway in some cases, because of their specialness to
Prolog interpreters.  I was really tempted to allow and use arbitrary UTF-8
(in particular, kanji) in atoms but refrained because of the desire for
Hamlog code to be easily readable by nearly all Prolog interpreters.

I tried to keep the number of built-in predicates to an absolute minimum,
partly because any that are not standard Prolog have to be re-implemented in
Prolog (and probably once for every supported Prolog) to build the shell
that executes Hamlog programs on a Prolog interpreter.  Here's an exhaustive
list.

\begin{itemize}
  \item ! [cut].  This is implemented by string substitution as well:  when
  a clause body gets added to the to-satisfy stack, there's an additional
  regular expression substitution pass that converts any instances of !/0 in
  the body into !/1 where the argument is an integer identifying the current
  height of the choice-point stack.  If at any point in the future we
  attempt to satisfy a !/1 goal, then the stack gets popped back to that
  point (discarding any choice points created between the time the
  ! got its argument and the present time).  For this reason, ! should not
  be used as an atom or functor for any other purposes than as the cut,
  even if to do so would otherwise be valid Prolog.

  \item fail, which causes immediate backtrack (useful in conjunction with
  cut to implement negation).

  \item true, which is not actually used, so maybe I should delete it.

  \item var, true if the argument is a variable (not yet bound to an atom). 
  This is important in Hamlog because many predicates need to be written to
  accept more than one instantiation pattern for their arguments.

  \item atom, true if the argument is an atom.  This is implemented by
  rewriting the database entry for atom/1 on the fly; when you call
  ``atom(foo)'' it magically changes to having been defined as either
  the single fact ``atom(foo).'' or nothing, depending on whether
  ``foo'' is an atom.
  
  \item atom\_final(AZ,A,Z), where AZ is an atom whose last character is Z
  and A is everything except the last character.  Used for building and
  tearing apart atoms like ``page00.''  This requires some careful handling
  in other interpreters because Hamlog has no concept of quotation marks and
  treats single-digit integers exactly the same as atoms whose names are the
  ASCII digits; real Prologs have more subtle type handling.  As with
  atom/1, this is implemented by magically creating appropriate clauses on
  the fly.
  
  \item =/2.  This also creates clauses on the fly.  At least one of the two
  arguments should already be atomic when the goal executes, though that is
  rarely difficult to guarantee in practice.
\end{itemize}

And that's it for built-in predicates.  Note that goals in a clause body
also can only be combined with comma for conjunction (no semicolons for
disjunction, and without them parentheses become unnecessary and are not
supported either).  There is also no syntactic support for negation. 
However, you can (and the existing code does) compute negation and
disjunction using multi-clause predicates, cut, and fail.  What you can't
build is any kind of I/O---so how can Hamlog programs communicate with the
outside world?

The interpreter (after checking for the ``-{}-version'' and ``-{}-debug''
options, which do fairly obvious things) interprets its first two
command-line arguments as a template and a query.  The template ought to be
a valid Prolog compound term for compatibility with other interpreters, but
Hamlog actually treats it as a string.  Then it backtracks through all
solutions to the query, attempting to instantiate all variables, and writes
(newline separated to standard out) all the \emph{distinct} values
assumed by the template.  This is basically the same operation as Prolog
findall/3 followed by sort/2, which is how the Prolog shell for Hamlog
implements it.  Any remaining command-line arguments, and standard input if
there are none of those, will be read in the usual <> way to fill the
Hamlog program database.  Hamlog code is conventionally stored in ``*.hl''
files.

In the build system, the string of comma-separated tags for things like
characters to be selected gets converted (by the ``make-cfghl'' Perl script)
into a few clauses of Hamlog and written into the file config.hl.  Also
written there is a list of page\_exists/1 facts naming the 256-code-point
blocks for source files that exist in the mp directory.  Then elsewhere in
the build system, it invokes Hamlog with appropriate queries against
config.hl and select.hl to get lists of characters, OpenType features, and
other things that the user does or doesn't want, based on knowledge built
into select.hl of what the different selection tags actually mean.

It is planned that in a future version, the KANJIDIC2 file will be
automatically translated to more Hamlog facts expressing which kanji are or
aren't included in the different grade levels; then it will be possible to
use options like ``-{}-enable-chars=kanji,-grade3' for finer-grained selection
of kanji.

In the case of a Prolog interpreter other than Hamlog, there is some
other code written in that interpreter's own language to allow it to execute
Hamlog programs and export something resembling this command-line interface
to the Makefiles.  Since Hamlog programs are also syntactically valid
Prolog, this support shouldn't be difficult in general.  See the swi-ham.pl
file for what currently exists of this nature.  The main advantage of using
a non-Hamlog interpreter is simply speed.

%%%%%%%%%%%%%%%%%%%%%%%%%%%%%%%%%%%%%%%%%%%%%%%%%%%%%%%%%%%%%%%%%%%%%%%%
%%%%%%%%%%%%%%%%%%%%%%%%%%%%%%%%%%%%%%%%%%%%%%%%%%%%%%%%%%%%%%%%%%%%%%%%

\blsection{カーニングしかた}{Kerning}

This is a summary of how the automated proportional spacing and kerning code
works.

First, the build system generates the future PS font as an OpenType file,
all complete except for widths and kerning.  It then calls bdf.pe, which
sets all the bearings to 50 font units and makes a BDF-format
bitmap font scaled so that the reference kanji square (1000 font units)
takes up 100 pixels.  This script filters out glyphs that should not be
subject to kerning, which includes those with zero width (mostly combining
characters), all the hangul individual-jamo code points (whose layout is
handled by OpenType substitution features, and kerning would just make it
too complicated), and the special glyphs used by vertical fraction
composition (much the same situation as the hangul jamo).

The ``kerner'' C program reads that BDF font, puts all the glyphs into a
common bounding box big enough to contain any of them, and finds the left
and right contours---basically, the x-coordinates of the leftmost and
rightmost black pixels on each row---for each glyph, as well as the margins,
which are defined as the x-coordinates of the leftmost and rightmost pixels
on \emph{any} row of the glyph.

There is some special processing applied to the contours to make them more
suitable. Consider what happens in a glyph like ``='': many horizontal rows,
namely those above and below the entire glyph and those in between the two
lines, contain no black pixels at all, and so the leftmost and rightmost
black pixels in those rows are formally undefined.  If we set it next to
another glyph like ``.'' which only has ink in rows where ``='' does not,
then a kerning algorithm that looked only horizontally might let the period
slide all the way under the equals and out on the other side.  There has to
be some vertical effect to prevent that. So the Tsukurimashou kerning
program makes a couple of passes over each contour (one forward, one
backward) to enforce the following rules, which (except for glyphs
containing no black pixels at all, which are removed from consideration)
fully define where the contour should be.

\begin{itemize}
  \item The right contour cannot be any further left than the rightmost
  black pixel in the row.
  \item The right contour cannot be more than 10 font units (one pixel)
  further left than its value in the next or the previous row.
  \item If the right contour in the next or previous row is left of the
  \emph{left} margin, then the right contour in this row cannot be
  more than 3 font units further left than in the next or
  previous row.
  \item Subject to the above rules, the right contour is as far left
  as possible.
  \item The left contour's rules are the mirror image of these.
\end{itemize}

These rules can be imagined as simulating something like the way letterpress
printers kern type by physically cutting the metal type bodies at an angle
to fit them more tightly together: it's bad to cut off part of the actual
printing surface; you basically cut at a 45° angle; but with glyphs that
only have a small vertical extent, so that the 45° angle would cut all the
way across to the other side, then you want to use a more vertical angle (in
this case, 16.7° from vertical) so you don't end up setting the next
character actually earlier on the line.  Figure~\ref{fig:kernmargins} shows
a typical example of the contour computation.

\begin{figure}
\centering
\begin{tikzpicture}[scale=0.5]
  \draw[thick,dotted] (-2,-9) -- (-2,9);
  \draw[thick,dotted] (2.5,-9) -- (2.5,9);
  \fill[fill=blue] (-2,-2) -- (0,-2)
    arc[start angle=-90,end angle=90,radius=2]
    -- (-2,2) -- (-2,2.5) -- (0,2.5)
    arc[start angle=90,end angle=-90,radius=2.5]
    -- (-2,-2.5) -- cycle;
  \draw[very thick] (3.1,-9) -- (2.5,-7) -- (-2,-2.5) -- (-2,-2) -- (0,0)
    -- (-2,2) -- (-2,2.5) -- (2.5,7) -- (3.1,9);
  \draw[very thick] (-3.04,-9) -- (-2,-5.54) -- (1.77,-1.77)
    arc[start angle=-45,end angle=45,radius=2.5]
    -- (-2,5.54) -- (-3.04,9);
\end{tikzpicture}
\caption{Glyph contour computation}
\label{fig:kernmargins}
\end{figure}

The right margin is subtracted from the right contours and the left margin
from the left contours to get, for each glyph, a vector of numbers
describing its shape along the left and right sides, independent of the
width of the glyph.

All the left contour vectors, and (independently) all the right contour
vectors, are subjected to a modified k-means classification.  Initially, the
contours are put in 200 classes, according to the values of a simple hash
function applied to the glyph names.  This is a change from earlier releases
in which a round-robin was used: the advantage of the hash function is that
although it remains deterministic, it helps break up a phenomenon that
tended to happen with the Genjimon glyphs, where because of excessive
symmetry in the initial arrangement, the classifier could never put the
similar glyphs together.

Then for each contour the program asks the
question ``How far is this contour from the centroid of its class, and if I
moved it to a different class, how far (after accounting for the fact that
the centroid changes when I add the glyph) would it be from the centroid of
the new class?'' If moving the glyph to some other class would make it
closer to the new centroid, then the glyph gets moved to the other class
where its distance to the centroid will be minimized.  Note that a glyph in
a class by itself will never want to move out of that class, because its
distance to the centroid is already zero.

There is an extra rule that the classification will never move a glyph into
a class in such a way as to make a class larger than the ``class limit,''
defined to be the larger of 100 glyphs, and three times the total number of
glyphs divided by the total number of classes.  The purpose of this rule is
to counteract a tendency seen in some experiments for the classifier to
create a few huge classes (for instance, a single class containing a large
fraction of the 11172 precomposed Korean syllables) that cause font subtable
size problems.  It is not clear just where the limits are on how big a class
may be, but this limit appears to work at the moment.

Glyphs are examined in this way until no more such moves are possible.  The
idea is that at the end of it, the glyphs will all be in classes that are as
tightly clustered as possible.  It's not guaranteed to be a global optimum
(in other words, it's possible that some other assignment of glyphs to
classes might be better; really optimizing this problem is difficult) but
it's guaranteed to be a local optimum in the sense that it can't be improved
by changing the assignment of just one glyph, and it's expected to be pretty
good overall.  Note that the initial assignment was deterministic (where
random would be more usual for this kind of algorithm) because it seems
undesirable for the kerning distances, overall, to be non-deterministic; my
copy of the font shouldn't have different metrics from yours if they were
compiled from the same sources with the same options.

After classification, we've got 200 classes of left contours and 200 classes
of right contours.  The actual kerning is done class-to-class, using the
class centroids, rather than glyph-to-glyph.  That way we will end up with
up to 40000 kern pairs instead of millions.  OpenType supports this kind of
kerning pretty well.  There will be a feature file generated listing the
contents of the classes and the distances for each pair of classes, and
that's much more efficient both in source and compiled form than specifying
a distance for every pair of glyphs.

The number 200 (up from 150 in earlier versions) was chosen by educated
trial and error.  Every glyph must be in a class for full kerning; but no
class can be too big or FontForge barfs.  That in turn limits the size of
the average class to some maximum, and so limits the number of classes to
some minimum.  Using 200 classes seems to be enough for the current
Jieubsida fonts, which are roughly 13000 glyphs each, and it is reasonable
to estimate that none of the fonts created in the planned scope of this
project will have much more than that many glyphs.

To kern two contours together, we can compute a closeness value for each row
by saying ``if we positioned the margins of the glyphs this much apart, how
far apart would the contours be in this row?'' That distance, divided by a
constant representing the optimal distance (currently 230 font units) and
raised to a power representing how much extra weight to give to the closest
points (currently 3), represents closeness for the row.  The sum of
closeness for all the rows would be equal to the number of rows in the case
of two perfectly vertical lines 230 units apart.  The kerner program adjusts
the margin-to-margin distance so that the sum of closeness is equal to that. 
It uses a binary search to do that adjustment, which is probably not optimal
for a fixed exponent (there should be an exact analytic solution possible
without iterating) but has the big advantage of not requiring a redesign
should the exponent or even the entire closeness function change.

The closeness computation is shown schematically in
Figure~\ref{fig:kernclose}.  Note that this closeness is calculated on the
contours, as defined previously, rather than the actual shapes of the
glyphs; it is also done on the centroids of the kerning groups, thus generic
contours each representing many glyphs, rather than the contours of any
specific individual glyphs.

\begin{figure}
\centering
\begin{tikzpicture}[scale=0.5]
  \fill[top color=white,bottom color=red!40!white,middle color=red]
    (0,0) rectangle (5,3.8);
  \fill[bottom color=white,top color=red!40!white,middle color=red]
    (0,0) rectangle (5,-3.8);
  \fill[fill=white!90!black] (-3.04,-9) -- (-2,-5.54) -- (1.77,-1.77)
    arc[start angle=-45,end angle=45,radius=2.5]
    -- (-2,5.54) -- (-3.04,9) -- cycle;
  \fill[fill=white!90!black] (8.1,-9) -- (7.5,-7) -- (3,-2.5) -- (3,-2) -- (5,0)
    -- (3,2) -- (3,2.5) -- (7.5,7) -- (8.1,9) -- cycle;
  \draw[very thick] (-3.04,-9) -- (-2,-5.54) -- (1.77,-1.77)
    arc[start angle=-45,end angle=45,radius=2.5]
    -- (-2,5.54) -- (-3.04,9);
  \draw[very thick] (8.1,-9) -- (7.5,-7) -- (3,-2.5) -- (3,-2) -- (5,0)
    -- (3,2) -- (3,2.5) -- (7.5,7) -- (8.1,9);
\end{tikzpicture}
\caption{Closeness computation}
\label{fig:kernclose}
\end{figure}

The effect of the exponent 3 in the calculation is to give much more weight
to points that are close together, as suggested by the shading in the
figure.  If we're kerning a pair like ``]<'', we want to pay more attention
to the point of the less-than than to the distant ends.  An exponent of 3
means that points at half the distance count eight times as much toward
overall closeness, so there's a strong bias toward seeing the points of
closest approach.  If we imagined using a larger exponent, this bias would
be even stronger; in the limit, with an infinite exponent, the kerning would
be determined solely by setting the closest approach to the optimum without
reference to any other points.  That is how most auto-kerning software
works; but the results tend not to be good because in a serif font with a
pair like {\fontspec{TsukurimashouMinchoPS}``AV,''} inserting the ideal
vertical-to-vertical distance between the serifs is going to place the
stems, which are much more visually important, too far apart.  Using an
exponent somewhat less than infinity causes the stems to still have some
significant weight.  The value 3 was chosen by trial and error and may be
subject to further adjustment.

Once all the class-to-class distances have been chosen, it remains to choose
the bearings for the characters.  Recall that kerning was computed from
margin to margin, that is the amount of space to insert between the strict
bounding boxes of the glyphs.  Adding a certain amount of extra space to the
glyphs themselves, and subtracting it from the kern distances, may result in
a better, more concise kern table, as well as better results when glyphs
from this font are set next to spaces, things from other fonts, and so on.

The first step is that the kerner program finds the average of all kern
table values, and puts half that much bearing space on either side of every
glyph, adjusting the kern values accordingly.  This has the effect of giving
every glyph an ``average'' amount of space, and changing the overall average
kern adjustment to zero.  If we were to throw away the kerning table at this
point and just use the bearings, then in some sense these bearings would
give the best possible approximation of the discarded kerning table.

Then for each left-class (which is actually a class of \emph{right}
contours: it is a class of glyphs that can appear on the left in a kerning
pair) the program finds the maximum, that is farthest apart, amount of
kerning between that left-class and any right-class.  Two thirds of that
kerning amount are added to the right-bearing of the left-class.  The
concept here is that we generally want kerning to be pushing things
together, not pulling them apart, so the ``default'' amount of kerning
indicated by the bearing should be near the maximum distance apart, from
which individual entries can then push things closer.  Also, we generally
want most (two thirds) of this adjustment to happen to the right bearing of
the left-hand glyph in the pair.

Then to clean up the rest, the program examines each right-class (which is a
class of left contours) and similarly finds the furthest-apart kern pair and
adds that to the left bearing of the right class, adjusting all kern pairs
appropriately.  At this stage it's guaranteed that all the kern table
entries will be zero or negative:  kerning only pushes glyphs together from
where they would otherwise be, it never pulls them apart.

Kern table entries are dropped if they are less than ten font units; that
cuts the size of the table considerably.  In a change from earlier versions,
the kern values are not otherwise rounded (beyond being integers).  The
table is written to a fragment of an Adobe .fea file, with subtable breaks
on left-class boundaries each time a subtable grows past 5000 entries;
that means subtables actually end up a little over 5000 entries each.  That
seems to be how big they can get without overflowing the OTF table-size
limits.  Bearings are written to a FontForge .pe script.

The build system runs kern-font.pe, which applies the output of the kerner
program to the font.  Something else that kern-font.pe does is to add a
hardcoded additional bearing of 40 on the left and 80 on the right to all
Japanese-script characters (kana and kanji); by trial and error, this seems
to make the results look better.  It seems to be simply a fact that Japanese
characters need more space between them to look right than Latin characters
do at the same type size.  Something similar should probably be done to
Korean-script characters, but that has not been determined yet.

That is how the horizontal spacing of the font is currently computed.  It
still isn't perfect, but some progress has been made.

%%%%%%%%%%%%%%%%%%%%%%%%%%%%%%%%%%%%%%%%%%%%%%%%%%%%%%%%%%%%%%%%%%%%%%%%

\Large さてさてなにが、できるかな?

\end{document}
