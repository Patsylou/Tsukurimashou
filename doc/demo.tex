\documentclass[14pt]{extarticle}

%
% Demo for Tsukurimashou
% Copyright (C) 2011  Matthew Skala
%
% This program is free software: you can redistribute it and/or modify
% it under the terms of the GNU General Public License as published by
% the Free Software Foundation, version 3.
%
% As a special exception, if you create a document which uses this font, and
% embed this font or unaltered portions of this font into the document, this
% font does not by itself cause the resulting document to be covered by the
% GNU General Public License. This exception does not however invalidate any
% other reasons why the document might be covered by the GNU General Public
% License. If you modify this font, you may extend this exception to your
% version of the font, but you are not obligated to do so. If you do not
% wish to do so, delete this exception statement from your version.
%
% This program is distributed in the hope that it will be useful,
% but WITHOUT ANY WARRANTY; without even the implied warranty of
% MERCHANTABILITY or FITNESS FOR A PARTICULAR PURPOSE.  See the
% GNU General Public License for more details.
%
% You should have received a copy of the GNU General Public License
% along with this program.  If not, see <http://www.gnu.org/licenses/>.
%
% Matthew Skala
% http://ansuz.sooke.bc.ca/
% mskala@ansuz.sooke.bc.ca
%

\input{version.tex}

\usepackage{fontspec}
\usepackage[margin=1.25in,top=0.85in]{geometry}
\usepackage{wrapfig}
\usepackage{xltxtra}

\usepackage{hyperref}

\defaultfontfeatures{Mapping=tex-text,Path=../otf/}

\setlength{\parindent}{0pt}
\setlength{\parskip}{\baselineskip}

%%%%%%%%%%%%%%%%%%%%%%%%%%%%%%%%%%%%%%%%%%%%%%%%%%%%%%%%%%%%%%%%%%%%%%%%

% Just because all these are conditional doesn't mean all invocations of
% them are conditional.  If you really disable core fonts, you'll lose.
% The main point is just to allow people to build a demo without
% Jieubsida, since that's the default configuration.

\expandafter\ifx\csname haveTsukurimashouKakuPS\endcsname\relax\else
  \newfontface\kaku{TsukurimashouKakuPS}
\fi

\expandafter\ifx\csname haveTsukurimashouMaruPS\endcsname\relax\else
  \newfontface\maru{TsukurimashouMaruPS}
\fi

\expandafter\ifx\csname haveTsukurimashouMinchoPS\endcsname\relax\else
  \newfontface\mincho{TsukurimashouMinchoPS}
\fi

\expandafter\ifx\csname haveTsukurimashouAnbirutekiPS\endcsname\relax\else
  \newfontface\anbiruteki{TsukurimashouAnbirutekiPS}
\fi

\expandafter\ifx\csname haveTsukurimashouTenshinoKamiPS\endcsname\relax\else
  \newfontface\tenshinokami{TsukurimashouTenshinoKamiPS}
\fi

\expandafter\ifx\csname haveTsukurimashouBokukkoPS\endcsname\relax\else
  \newfontface\bokukko{TsukurimashouBokukkoPS}
\fi

\expandafter\ifx\csname haveJieubsidaBatangPS\endcsname\relax\else
  \newfontface\batang[RawFeature={+ccmp,+ljmo,+vjmo,+liga}]{JieubsidaBatangPS}
\fi

\expandafter\ifx\csname haveJieubsidaDodumPS\endcsname\relax\else
  \newfontface\dodum[RawFeature={+ccmp,+ljmo,+vjmo,+liga}]{JieubsidaDodumPS}
\fi

\expandafter\ifx\csname haveJieubsidaSunMoonPS\endcsname\relax\else
  \newfontface\sunmoon[RawFeature={+ccmp,+ljmo,+vjmo,+liga}]{JieubsidaSunMoonPS}
\fi

\expandafter\ifx\csname haveTsukurimashouKaku\endcsname\relax\else
  \newfontface\kakumono[WordSpace={1,0,0},PunctuationSpace=3]{TsukurimashouKaku}
\fi

\expandafter\ifx\csname haveTsukurimashouMaru\endcsname\relax\else
  \newfontface\marumono[WordSpace={1,0,0},PunctuationSpace=3]{TsukurimashouMaru}
\fi

\expandafter\ifx\csname haveTsukurimashouMincho\endcsname\relax\else
  \newfontface\minchomono[WordSpace={1,0,0},PunctuationSpace=3]{TsukurimashouMincho}
\fi

\expandafter\ifx\csname haveTsukurimashouAnbiruteki\endcsname\relax\else
  \newfontface\anbirutekimono[WordSpace={1,0,0},PunctuationSpace=3]{TsukurimashouAnbiruteki}
\fi

\expandafter\ifx\csname haveTsukurimashouTenshinoKami\endcsname\relax\else
  \newfontface\tenshinokamimono[WordSpace={1,0,0},PunctuationSpace=3]{TsukurimashouTenshinoKami}
\fi

\expandafter\ifx\csname haveTsukurimashouBokukko\endcsname\relax\else
  \newfontface\bokukkomono[WordSpace={1,0,0},PunctuationSpace=3]{TsukurimashouBokukko}
\fi

\expandafter\ifx\csname haveTsukurimashouKakuPS\endcsname\relax\else
\expandafter\ifx\csname haveTsukurimashouMinchoPS\endcsname\relax\else
  \setmainfont[Mapping=tex-text,ItalicFont=TsukurimashouMinchoPS]{TsukurimashouKakuPS}
\fi\fi

\expandafter\ifx\csname haveTsukurimashouBokukko\endcsname\relax\else
  \setmonofont{TsukurimashouBokukko}
\fi

%%%%%%%%%%%%%%%%%%%%%%%%%%%%%%%%%%%%%%%%%%%%%%%%%%%%%%%%%%%%%%%%%%%%%%%%

\newcommand{\iroha}{%
いろはにほへとち~~~~イロハニホヘトチ\\
りぬるをわかよた~~~~リヌルヲワカヨタ\\
れそつねならむう~~~~レソツネナラムウ\\
ゐのおくやまけふ~~~~ヰノオクヤマケフ\\
こえてあさきゆめ~~~~コエテアサキユメ\\
みしゑひもせす。~~~~ミシヱヒモセス。\par
}

\newcommand{\gojuuonzu}{%
わらやまはなたさかあ~~~~ワラヤマハナタサカア\\
ゐり~~みひにちしきい~~~~ヰリ~~ミヒニチシキイ\\
~~るゆむふぬつすくう~~~~~~ルユムフヌツスクウ\\
ゑれ~~めへねてせけえ~~~~ヱレ~~メヘネテセケエ\\
をろよもほのとそこおん~~ヲロヨモホノトソコオン\par
}

\newcommand{\undecl}{%
모든 인류 구성원의 천부의 존엄성과 동등하고 양도할 수 없는 권리를 인정하는
것이 세계의 자유 、 정의 및 평화의 기초이며 、

인권에 대한 무시와 경멸이 인류의 양심을 격분시키는 만행을 초래하였으며 、
인간이 언론과 신앙의 자유、 그리고 공포와 결핍으로부터의 자유를 누릴 수 있는
세계의 도래가 모든 사람들의 지고한 열망으로서 천명되어 왔으며 、

인간이 폭정과 억압에 대항하는 마지막 수단으로서 반란을 일으키도록 강요받지
않으려면 、 법에 의한 통치에 의하여 인권이 보호되어야 하는 것이 필수적이며 、

국가간에 우호관계의 발전을 증진하는 것이 필수적이며 、

국제연합의 모든 사람들은 그 헌장에서 기본적 인권、 인간의 존엄과 가치 、
그리고 남녀의 동등한 권리에 대한 신념을 재확인하였으며、 보다 폭넓은
자유속에서 사회적 진보와 보다 나은 생활수준을 증진하기로 다짐하였고、

회원국들은 국제연합과 협력하여 인권과 기본적 자유의 보편적 존중과 준수를
증진할 것을 스스로 서약하였으며 、

이러한 권리와 자유에 대한 공통의 이해가 이 서약의 완전한 이행을 위하여 가장
중요하므로 、

이에、

국제연합총회는、

모든 개인과 사회 각 기관이 이 선언을 항상 유념하면서 학습 및 교육을 통하여
이러한 권리와 자유에 대한 존중을 증진하기 위하여 노력하며 、 국내적 그리고
국제적인 점진적 조치를 통하여 회원국 국민들 자신과 그 관할 영토의 국민들
사이에서 이러한 권리와 자유가 보편적이고 효과적으로 인식되고 준수되도록
노력하도록 하기 위하여 、 모든 사람과 국가가 성취하여야 할 공통의 기준으로서
이 세계인권선언을 선포한다。 
\par
}

\renewcommand{\labelitemi}{{\fontspec[RawFeature=+ornm]{TsukurimashouKaku}C}}

%%%%%%%%%%%%%%%%%%%%%%%%%%%%%%%%%%%%%%%%%%%%%%%%%%%%%%%%%%%%%%%%%%%%%%%%

\begin{document}
\pagestyle{plain}\thispagestyle{empty}

%%%%%%%%%%%%%%%%%%%%%%%%%%%%%%%%%%%%%%%%%%%%%%%%%%%%%%%%%%%%%%%%%%%%%%%%

\kaku
\begin{center}\LARGE

\vspace*{\fill}

{\Huge 作りましょう~\TsukurimashouVWide}\\
{\huge パラメタ方式フォントファミリ\\
デモ}

\vspace*{0.75in}

{\Huge Tsukurimashou~\TsukurimashouVersion}\\
{\huge Parametric Font Family\\
Demo}

\vspace*{1.5in}

Matthew Skala\\
mskala@ansuz.sooke.bc.ca\\
\TsukurimashouRDWide\qquad\TsukurimashouReleaseDate

\vspace*{\fill}

\end{center}
\clearpage

%%%%%%%%%%%%%%%%%%%%%%%%%%%%%%%%%%%%%%%%%%%%%%%%%%%%%%%%%%%%%%%%%%%%%%%%

\vspace*{\fill}

Demo for Tsukurimashou\\
Copyright © 2011~~Matthew Skala

This program is free software: you can redistribute it and/or modify
it under the terms of the GNU General Public License as published by
the Free Software Foundation, version 3.

As a special exception, if you create a document which uses this font, and
embed this font or unaltered portions of this font into the document, this
font does not by itself cause the resulting document to be covered by the
GNU General Public License. This exception does not however invalidate any
other reasons why the document might be covered by the GNU General Public
License. If you modify this font, you may extend this exception to your
version of the font, but you are not obligated to do so. If you do not
wish to do so, delete this exception statement from your version.

This program is distributed in the hope that it will be useful,
but WITHOUT ANY WARRANTY; without even the implied warranty of
MERCHANTABILITY or FITNESS FOR A PARTICULAR PURPOSE.  See the
GNU General Public License for more details.

You should have received a copy of the GNU General Public License
along with this program.  If not, see \url{http://www.gnu.org/licenses/}.

\clearpage

%%%%%%%%%%%%%%%%%%%%%%%%%%%%%%%%%%%%%%%%%%%%%%%%%%%%%%%%%%%%%%%%%%%%%%%%

\kaku

\Large
作りましょう角\\
Tsukurimashou Kaku [Square Gothic]

\normalsize

The most popular Japanese text typefaces may be classified into ``Goshikku''
(transliterated from ``Gothic'' and inspired by sans-serif Latin typefaces)
and ``Mincho'' (inspired by Chinese handwriting standards of the Ming
Dynasty, 1368--1644~\textsc{ce}.) Goshikku typefaces are often further
divided into ``Kaku'' (``square,'' referring to the ends of the strokes) and
``Maru'' (``rounded'').  As is common with Goshikku faces, the standard
weight of Tsukurimashou Kaku is a little darker than the standard weight of
the corresponding Mincho face.  Its simple, readable characters form the
core of the Tsukurimashou family, and are then modified to create the other
styles.

\kakumono
\iroha
\gojuuonzu

ABCDEFGHIJKLM~~~~abcdefghijklm\\
NOPQRSTUVWXYZ~~~~nopqrstuvwxyz\\
ABCDEFGHIJKLMNOPQRSTUVWXYZ~~~~abcdefghijklmnopqrstuvwxyz

\clearpage

%%%%%%%%%%%%%%%%%%%%%%%%%%%%%%%%%%%%%%%%%%%%%%%%%%%%%%%%%%%%%%%%%%%%%%%%

\maru

\Large
作りましょう丸\\
Tsukurimashou Maru [Round Gothic]

\normalsize

Tsukurimashou Maru is closely connected to Tsukurimashou Kaku, but its
rounded stroke ends make it somewhat less formal. 
There is also a little more variation in stroke width, inspired by the
``humanist'' category of Latin sans serif typefaces.

\marumono
\iroha
\gojuuonzu

ABCDEFGHIJKLM~~~~abcdefghijklm\\
NOPQRSTUVWXYZ~~~~nopqrstuvwxyz\\
ABCDEFGHIJKLMNOPQRSTUVWXYZ~~~~abcdefghijklmnopqrstuvwxyz

\clearpage

%%%%%%%%%%%%%%%%%%%%%%%%%%%%%%%%%%%%%%%%%%%%%%%%%%%%%%%%%%%%%%%%%%%%%%%%

\expandafter\ifx\csname haveTsukurimashouAnbirutekiPS\endcsname\relax\else

\anbiruteki

\Large
作りましょうアンビル的\\
Tsukurimashou Anbiruteki [Anvilicious]

\normalsize

The Web site TV Tropes coined the term ``anvilicious'' to describe any work
of art that attempts to Make a Point (for instance, about politics or
morality) in such a heavy-handed manner that the message overwhelms any
other attribute of the work; it's like dropping an anvil on the viewer.  A
similarly-constructed adjective in Japanese might be rendered as
``Anibiruteki.'' Tsukurimashou Anbiruteki is so heavy that fine details of
the letters are obscured, though they generally remain recognizable in
silhouette.  Punctuation marks are also enlarged until they sometimes
overlap nearby characters.

\anbirutekimono
\iroha
\gojuuonzu

ABCDEFGHIJKLM~~~~abcdefghijklm\\
NOPQRSTUVWXYZ~~~~nopqrstuvwxyz\\
ABCDEFGHIJKLMNOPQRSTUVWXYZ~~~~abcdefghijklmnopqrstuvwxyz

\clearpage

\fi

%%%%%%%%%%%%%%%%%%%%%%%%%%%%%%%%%%%%%%%%%%%%%%%%%%%%%%%%%%%%%%%%%%%%%%%%

\expandafter\ifx\csname haveTsukurimashouTenshinoKamiPS\endcsname\relax\else

\tenshinokami

\Large
作りましょう天使の髪\\
Tsukurimashou Tenshi no Kami [Angel Hair]

\normalsize

Tsukurimashou Tenshi no Kami reduces the letter forms to fine hairlines.  It
might be most suitably used in large size for display purposes.

\tenshinokamimono
\iroha
\gojuuonzu

ABCDEFGHIJKLM~~~~abcdefghijklm\\
NOPQRSTUVWXYZ~~~~nopqrstuvwxyz\\
ABCDEFGHIJKLMNOPQRSTUVWXYZ~~~~abcdefghijklmnopqrstuvwxyz

\clearpage

\fi

%%%%%%%%%%%%%%%%%%%%%%%%%%%%%%%%%%%%%%%%%%%%%%%%%%%%%%%%%%%%%%%%%%%%%%%%

\bokukko

\Large
作りましょう僕っ娘\\
Tsukurimashou Bokukko [Tomboy]

\normalsize

The slang term ``bokukko'' could reasonably be glossed ``tomboy,'' but
it refers more specifically to a girl or woman who
uses the traditionally masculine pronoun ``boku'' in her speech.  One often
hears that mannerism from fictional characters and in song lyrics.  It's
typically more a statement of personal aggressiveness than any kind of
transgender identity, though the two may be linked. It is not so common in
real life.  Tsukurimashou Bokkuko originated in an experiment with
Metafont's path expansion features, not meant to be kept, but the results
looked good enough to be worth giving it a name and some further development
work.  The visual style is defined in abstract terms, but resembles
characters written with a chisel-tip felt marker.

\bokukkomono
\iroha
\gojuuonzu

ABCDEFGHIJKLM~~~~abcdefghijklm\\
NOPQRSTUVWXYZ~~~~nopqrstuvwxyz\\
ABCDEFGHIJKLMNOPQRSTUVWXYZ~~~~abcdefghijklmnopqrstuvwxyz

\clearpage

%%%%%%%%%%%%%%%%%%%%%%%%%%%%%%%%%%%%%%%%%%%%%%%%%%%%%%%%%%%%%%%%%%%%%%%%

\mincho

\Large
作りましょう明朝\\
Tsukurimashou Mincho [Ming Dynasty]

\normalsize

Mincho type features less abstract forms than Goshikku, closer to the
handwritten roots of the Japanese writing system.  This style is often
treated as equivalent to serif Latin typefaces.  Tsukurimashou Mincho is
somewhat experimental: it draws heavily on the Goshikku face, with just a
few shape modifications.  Mincho character shapes are controlled by a
variable representing the amount of ``minchoviality,'' which is 1 for this
style and 0 for most of the others.  (It is 0.3 for Tsukurimashou Bokukko.)
Further modifications come from adding serifs, and a more complicated
parameterization of whether a given style is more like ``print'' or
``handwriting''; the code actually simulates moving a brush through
four-dimensional space, and different styles convert that movement into two
dimensions via differing linear projections and threshold functions. Since
it is largely based on Tsukurimashou Kaku, Tsukurimashou Mincho has a more
modern and stylized look than would an independently-designed Mincho face
drawn directly from brush strokes.  Accordingly, a Modern style was chosen
for the associated Latin alphabet.

\minchomono
\iroha
\gojuuonzu

ABCDEFGHIJKLM~~~~abcdefghijklm\\
NOPQRSTUVWXYZ~~~~nopqrstuvwxyz\\
ABCDEFGHIJKLMNOPQRSTUVWXYZ~~~~abcdefghijklmnopqrstuvwxyz

\clearpage

%%%%%%%%%%%%%%%%%%%%%%%%%%%%%%%%%%%%%%%%%%%%%%%%%%%%%%%%%%%%%%%%%%%%%%%%

\expandafter\ifx\csname haveJieubsidaDodumPS\endcsname\relax\else

\dodum

\Large
지읍시다돋움\\
Jieubsida Dodum

\normalsize

``Dodum'' means ``stand out'' and is the Korean typographic term equivalent
to ``Goshikku.''  This font uses the same style parameters as Tsukurimashou
Kaku, but it covers Korean letter and syllable glyphs.

\undecl

\clearpage

\fi

%%%%%%%%%%%%%%%%%%%%%%%%%%%%%%%%%%%%%%%%%%%%%%%%%%%%%%%%%%%%%%%%%%%%%%%%

\expandafter\ifx\csname haveJieubsidaBatangPS\endcsname\relax\else

\batang

\Large
지읍시다바탕\\
Jieubsida Batang

\normalsize

This typeface shares the style parameters of Tsukurimashou Mincho, with
Korean glyph coverage.  The name ``Batang'' translates as ``background''
(roughly opposite to ``dodum'') and is the currently-popular term for basic
serif type in Korean.

\undecl

\clearpage

\fi

%%%%%%%%%%%%%%%%%%%%%%%%%%%%%%%%%%%%%%%%%%%%%%%%%%%%%%%%%%%%%%%%%%%%%%%%

\expandafter\ifx\csname haveJieubsidaSunMoonPS\endcsname\relax\else

\sunmoon

\Large
지읍시다선문\\
Jieubsida Sun-Moon

\normalsize

Jieubsida Sun-Moon has a similar look to Tsukurimashou Bokukko, reminiscent
of hand lettering with a felt-tip marker.  It is named after a character
from the Web comic Bonobo Conspiracy.

\undecl

\clearpage

\fi

%%%%%%%%%%%%%%%%%%%%%%%%%%%%%%%%%%%%%%%%%%%%%%%%%%%%%%%%%%%%%%%%%%%%%%%%

\kaku

\Large
Special OpenType and Unicode features

\normalsize

This is a quick sample of some of the OpenType features in the font; all of
them should be considered experimental.

Combining sound marks, precomposed and
on-the-fly:  {\kakumono ぎき{\char"3099}~~ぱは{\char"309A}%
~~ゴコ{\char"3099}~~プフ{\char"309A}}\\
Can you spot the difference?

へんなひらがなを作れます:~~{\kakumono ら{\char"3099}め{\char"309A}%
え{\char"309A}ぬ{\char"3099}}。\\
カタカナも:~~{\kakumono ユ{\char"309A}ン{\char"3099}モ{\char"3099}リ%
{\char"309A}}。

{\fontspec[Letters=SmallCaps]{TsukurimashouKakuPS}
Optical small caps are available via the appropriate OpenType feature and
Adobe's deprecated PUA encoding.
}

{\fontspec[RawFeature=+ornm]{TsukurimashouKakuPS}
Tomoe ornaments: \Large A B C D E F G H
}

げんじもん:{\Large 󱟁󱟆󱟓󱟢}

I Ching: {\Large ䷂䷉䷙䷓}

{\fontspec[StylisticSet=1,StylisticSet=2]{TsukurimashouKakuPS}
Stylistic substitutions (heavy metal umlaut, enclosed chars):\\
 (1) Motörhead\\
 ((2)) Mötley Crüe\\
 \{3\} Mormoñ Tabärnacle Choïr}

{\fontspec[StylisticSet=2]{TsukurimashouKakuPS}
More enclosed chars: (A) (b) (シ) [D] <E>}

{\fontspec[Fractions=Alternate]{TsukurimashouKaku}
Fractions 1/2~~3/4~~1/2~~3/4~~12/34~~123/45~~9876/4321\\
フラクショ~~~~4/769~~8123/6~~93/401~~1001/11
}

{\fontspec[Fractions=Alternate]{TsukurimashouKakuPS}
Fractions 1/2~~3/4~~1/2~~3/4~~12/34~~123/45~~9876/4321\\
フラクショ~~~~4/769~~8123/6~~93/401~~1001/11
}

\clearpage

%%%%%%%%%%%%%%%%%%%%%%%%%%%%%%%%%%%%%%%%%%%%%%%%%%%%%%%%%%%%%%%%%%%%%%%%

\kaku

\Large
Multilingual support

\normalsize

Here is a familiar text in some of the many languages these
fonts cover:

And the LORD said, ``Behold, the people are one and they have all one
language, and this they begin to do; and now nothing will be withheld from
them which they have imagined to do.  Come, let Us go down, and there
confound their language, that they may not understand one another's
speech.''\footnote{English, KJ21---21st Century King James Version, Deuel
Enterprises, Inc.}

神様は、 「ほら、 人間は統一的です。 言語も、 統一的です。
これが始めてだから、 人間は何で
も思い描く事が出来ますよ。」
と言いました。 「じゃあ、 下りましょう! 分かり合わない事のために、
あそこで言語の変態をしましょう!」。%
\footnote{日本語 (Japanese), Tsukurimashou Project.}

\expandafter\ifx\csname haveJieubsidaDodumPS\endcsname\relax\else
{\dodum%
여호와께서 가라사대 이 무리가 한 족속이요, 언어도 하나이므로 이같이
시작하였으니 이후로는 그 경영하는 일을 금지할 수 없으리로다.
자, 우리가 내려가서 거기서 그들의 언어를 혼잡케 하여 그
들로 서로 알아듣지 못하게 하자 하시고.\footnote{{\dodum 한국말} (Korean),
Unbound Bible, Biola University.}
\fi

Und der HERR sprach: Siehe, es ist einerlei Volk und einerlei Sprache unter
ihnen allen, und haben das angefangen zu tun; sie werden nicht ablassen von
allem, was sie sich vorgenommen haben zu tun.
Wohlauf, laßt uns herniederfahren und ihre Sprache daselbst verwirren,
daß keiner des andern Sprache verstehe!\footnote{Deutsch (German),
Luther Bibel 1545}

Y dijo Jehová: «El pueblo es uno, y todos estos tienen un solo lenguaje; han
comenzado la obra y nada los hará desistir ahora de lo que han pensado
hacer.  Ahora, pues, descendamos y confundamos allí su lengua, para que
ninguno entienda el habla de su compañero».\footnote{Español (Spanish),
Reina-Valera 1995, United Bible Societies}

Epi li di. Koulye a, gade! Yo tout fè yon sèl pèp. Yo tout yo pale yon sèl
lang. Gade sa yo konmanse ap fè. Talè konsa y'ap pare pou yo fè sa yo vle.
Bon. N'ap desann, n'ap mele lang yo. Konsa, yonn p'ap ka konprann sa lòt ap
di.\footnote{Kreyol (Haitian Creole), United Bible Societies}

og han sagde: ``Se, de er eet Folk og har alle eet Tungemål; og når de nu
først er begyndt således, er intet, som de sætter sig for, umuligt for dem;
lad os derfor stige ned og forvirre deres Tungemål der, så de ikke
forstår hverandres Tungemål!''\footnote{Dansk
(Danish), unidentified public domain version from Bible Gateway}

Et l'Éternel dit: Voici, ils forment un seul peuple et ont tous une même
langue, et c'est là ce qu'ils ont entrepris; maintenant rien ne les
empêcherait de faire tout ce qu'ils auraient projeté.
Allons! descendons, et là confondons leur langage, afin qu'ils
n'entendent plus la langue, les uns des autres.\footnote{Français (French),
Louis Segond}

Og Drottinn mælti: ``Sjá, þeir eru ein þjóð og hafa allir sama tungumál, og
þetta er hið fyrsta fyrirtæki þeirra. Og nú mun þeim ekkert ófært verða, sem
þeir taka sér fyrir hendur að gjöra.
Gott og vel, stígum niður og ruglum þar tungumál þeirra, svo að enginn
skilji framar annars mál.''\footnote{Íslenska (Icelandic), unidentified
public domain version from Bible Gateway}

Il Signore disse: «Ecco, essi sono un solo popolo e hanno tutti una lingua
sola; questo è l'inizio della loro opera e ora quanto avranno in progetto di
fare non sarà loro impossibile. Scendiamo dunque e confondiamo la loro
lingua, perché non comprendano più l'uno la lingua
dell'altro».\footnote{Italiano (Italian), Conferenza Episcopale Italiana}

\end{document}
