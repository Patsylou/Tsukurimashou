\documentclass{ltugboat}

\usepackage{graphicx}
\usepackage{ifpdf}
\ifpdf
\usepackage[breaklinks,colorlinks,linkcolor=black,citecolor=black,
            urlcolor=black]{hyperref}
\else
\usepackage{url}
\fi

%%%%%%%%%%%%%%%%%%%%%%%%%%%%%%%%%%%%%%%%%%%%%%%%%%%%%%%%%%%%%%%%%%%%%%%%

\title{Tsukurimashou: a Japanese-language font meta-family}

\author{Matthew Skala}
\address{Department of Computer Science\\
E2--445 EITC\\
University of Manitoba\\
Winnipeg MB\ \ R3T 2N2\\
Canada}
\netaddress{mskala@ansuz.sooke.bc.ca}
\personalURL{http://ansuz.sooke.bc.ca/}

%%%%%%%%%%%%%%%%%%%%%%%%%%%%%%%%%%%%%%%%%%%%%%%%%%%%%%%%%%%%%%%%%%%%%%%%

\begin{document}

\maketitle

\begin{abstract}
\MF-based font projects for the Chinese, Japanese, and Korean (\CJK)
languages have been announced every few years since the early 1980s,
even predating the current form of the \MF\ language.  Except for a few
non-parameterized conversions of fonts that originated in other formats,
in 30 years every \MF\ \CJK\ font has been abandoned at or before the
8-bit barrier of 256 \emph{kanji}, nowhere near the thousands required for
practical typesetting.  In this presentation I describe the first
project to break that barrier: Tsukurimashou
(\url{http://tsukurimashou.sourceforge.jp/}), currently at over 1500
\emph{kanji} (as well as kana, Latin, and Korean hangul) and steadily growing.
I discuss technical and human challenges facing this kind of project,
how to solve them, and spin-off technologies such as the IDSgrep
\emph{kanji} structural query system.
\end{abstract}

%%%%%%%%%%%%%%%%%%%%%%%%%%%%%%%%%%%%%%%%%%%%%%%%%%%%%%%%%%%%%%%%%%%%%%%%

\section{Introduction}

The Han script, used by the Chinese, Japanese, and Korean (\CJK) languages
among others, includes very many characters.  Just counting them is tricky,
but a human being might typically need to know a few thousand for basic
literacy in a Han-script language.  The list of 2136 characters taught in
the Japanese school system (the \emph{jouyou kanji}) is one benchmark, near
the low end.  Chinese requires more, and a typesetting system may require
more still, because of rare characters found in names, historical contexts,
and so on.  A human being can get away with failing to read the occasional
character; typesetting systems need to be able to print nearly all of them. 
Computer fonts considered usable for Japanese typically cover between six
and twelve thousand characters.  Databases of rare characters used in
linguistic research cover tens or hundreds of thousands.

The sheer number of characters that go into a \CJK\ font, and the quantity
of work implied by that number, is daunting.  Considering the difficulty of
building even a simple Latin font with \MF, it may be no surprise that there
are no complete \MF-native \CJK\ typefaces.  But on the other hand,
examination of Han-script text (even, or especially, by someone who cannot
read it) quickly reveals that characters can be decomposed into smaller
parts, as shown in Figure~\ref{fig:tree}.  Computer scientists who examine
Figure~\ref{fig:tree} are likely to believe they understand it.  ``Of
course,'' one supposes, ``the tens of thousands of Han characters are just a
small vocabulary of primitive shapes, perhaps only a few dozen of those,
which combine in straightforward ways according to a spatial grammar to form
tree structures!''

\begin{figure}
\includegraphics[scale=0.47]{tree.pdf}
\caption{Breaking a character into its parts.}
\label{fig:tree}
\end{figure}

Computer scientists know how to deal with such things.  It should be only
the work of a week or two for a good programmer to lash together a prototype
\CJK\ font generator.  Each primitive shape can be a subroutine; there can
be other subroutines expressing the combining operations such as ``place
this one above that one''; a few parameters applied to the low-level shapes
can allow for creating a wide range of styles; and the only real challenge
is looking in the dictionary that lists the tree decompositions of all the
characters.  That book must exist in China, so we'll get it by interlibrary
loan.  This project might even be easier than building a Latin font
meta-family.

The earliest \MF\ \CJK\ project I know of was \acro{LCCD}, the Language for
Chinese Character Design, described in a 1980 Stanford technical report by
Tung Yun Mei~\cite{Mei:LCCD}.  The \MF\ language in its current form did not
exist at the time, but Mei collaborated with Knuth and based \acro{LCCD} on
the early \MF\ work.  Even in 1980, many of the ideas were already in place
that a present-day computer scientist would naturally think of on viewing
Figure~\ref{fig:tree}.  Mei's report includes images of 346 ``basic strokes
and radicals,'' and 112 completed characters.

Subsequent work on \MF-native \CJK\ fonts includes that of Hobby and Guoan
in 1984, who created 128 characters~\cite{Hobby:Chinese}; Hosek in 1989,
character count unknown but two are displayed in the \TUB\
article~\cite{Hosek:Design}; Yiu and Wong in 2003, in a project that
targeted on-demand creation of rare characters rather than a font as
such~\cite{Yiu:Chinese}; and Laguna circa 2005, with 130 characters in the
last available version~\cite{Laguna:Hong}.  All these used a relatively
small number of basic components, combining according to a spatial grammar
to form more complicated characters.

I listed published \MF-related projects.  Similar ideas have also been used
behind closed doors in commercial font foundries (\acro{CDL} from Wenlin
Institute seems to be an example~\cite{Wenlin:CDL}), and non-\MF\ research
projects like the \acro{LISP}-based Wadalab toolkit~\cite{Tanaka:Wadalab}. 
The Wadalab font project ran during the 1990s; much of the work was lost or
withdrawn, but some of its fonts survived to become widely used
in the free software world.  These kinds of projects use grammars of
character parts, but they lack the full parameterization that \MF\ users
expect.  There has also been work on using \CJK\ fonts from other sources in
\TeX\ documents, sometimes including \MF\ incidentally in the workflow, but
again without parameterization.  For instance, the Poor Man's Chinese and
Japanese package~\cite{Ridgeway:Poor} converts bitmap fonts into \MF\ code
that renders scaled versions (without smoothing!) at arbitrary resolution.

It may be difficult to create fonts in \MF\ in general, regardless of the
script; but human beings have done it.  Several if not many \MF-native Latin
fonts exist, and we can typeset a wide range of
documents in Latin-script languages with
parameterized \MF-native fonts.  So after more than three decades of work,
why are there no usable, parameterized, \MF-native \CJK\ fonts at all?

%%%%%%%%%%%%%%%%%%%%%%%%%%%%%%%%%%%%%%%%%%%%%%%%%%%%%%%%%%%%%%%%%%%%%%%%

\section{Scaling issues}

It is no coincidence that past attempts to build \CJK\ fonts in \MF\ have
been abandoned at the same stage in development, around 120 characters. 
\emph{That is the roughly the size of a Latin font.} \MF\ was designed to
build fonts with sizes on that order, and \MF\ users have built expertise
and developed tools for building fonts the size of Latin fonts.  When fonts
get larger, unforeseen difficulties show up like \emph{nurikabe}\Dash the
plaster wall monsters of Japanese folklore blamed for
delaying travellers by night.

\subsection{Technical limitations}

Many font file formats are limited to 256 glyphs by their use of 8-bit
character codes.  People who attempt to typeset \CJK\ documents in classical
\TeX\ use elaborate workarounds involving slicing their fonts into 256-glyph
sub-fonts.  Handling the input encoding for documents written in large
character sets with these slicing schemes is a tough problem too, but
fortunately not one we must solve as font designers.  There are extended
versions of the \TeX\ interpreter designed to use longer character codes
directly (\XeTeX\ is one), and those may also be able to work with font
formats that store tens of thousands of glyphs per file and don't need to be
sliced; but there is no similarly extended \MF\ to produce fonts in such
formats.

Thousands of glyphs in a font does not just mean a bigger file.  It also
means more time spent compiling, and more memory consumption.  One run of
\MF\ may run out of memory or other resources trying to process an entire
multi-thousand-glyph \CJK\ font, and the user may run out of patience
recompiling the whole thing after changing one glyph.  To succeed at the
thousand-glyph level, a project must have build tools allowing separate
compilation of parts of the project.  There should be tracking of
dependencies among the different parts.  Just being able to \emph{find}
pieces of code in a project this size\Dash answering questions like ``what
was the name of the subroutine for such and such a shape?''\Dash is an
issue.  These are elementary problems in software engineering, but there is
little or no previous work on them in the \MF\ context because nobody has
built systems this size in \MF\ before.

Classical \MF\ is designed to produce bitmap fonts, but bitmap fonts are no
longer such a desired commodity.  A present-day \CJK\ font project will
presumably target a vector format, but making \MF\ or some variation of it
produce vector fonts requires additional layers of software, all of which
are to some extent experimental.  Bugs in the beyond-\MF\ software,
previously undetected because previous fonts were smaller, will show up and
need to be fixed.  Keeping a handle on the bugs requires a test suite.  The
need for multiple steps in font compilation underscores the need for a
capable build system.  Human designers cannot be expected to issue five or
six different commands in the right order to recompile every font, every time.

Earlier work on \MF\ \CJK\ fonts has concentrated on writing code in \MF\ to
draw the shapes of Han characters, as if that were the only problem to solve. 
Infrastructure that can scale to the size of the finished product is at
least as significant.

\subsection{Human factors}

It is easy to underestimate how much work is involved in building a \CJK\
font.  We know how much work it is to design a Latin font.  We know a \CJK\
font has about 30 times as many glyphs.  But it is easy to think, looking at
Figure~\ref{fig:tree}, that the \CJK\ font should actually only be something
like two or three times as much work as the Latin font (or even less),
because so much code can be reused.  In fact, less work is saved by code
reuse than one might hope: every glyph requires some human attention.  In
computer science terms, font design is not much less than $\Omega(n)$.

Once it becomes clear that a human being must spend time on every single
glyph\Dash it gets easier as more code exists to reuse, but there is no
break point after which hundreds of characters will suddenly come for
free\Dash it is natural to hope for that human being not to be oneself.  If
we can just build a sufficiently good, easy to use set of tools, we can put
them on the Web, maybe use a Wiki, and have many people in the community
build a few glyphs each.  Many hands make light work, once the
infrastructure exists.

But to hope for someone else to build the actual glyphs after the tools are
designed is to ignore why people participate in free software projects in
the first place.  Designing tools for glyph construction is \emph{fun}. 
Going through a list of 6000 glyphs one by one, doing simple repetitive
tasks on each of them, is \emph{work}.  It is not easy to get volunteers for
that sort of thing at the best of times, let alone when the volunteers must
also have proficiency in an obscure programming language.  The most
successful large-scale collaboration is probably
GlyphWiki~\cite{Kamichi:GlyphWiki}, which sacrifices parameterization for a
more purely graphical approach that demands less from the participants.

Finally, many of the potential rewards of a \MF\ \CJK\ project, such as
academic publications, can be had at the start, before the boring part; and
then there are no more rewards until the end, and few then.  You can publish
one paper about your innovative techniques for building fonts; and you can
publish one paper saying you have finished, years later.  There is little in
between.  Knowing that this is the reward structure makes it tempting to
write only the first paper and then start work on something else.

\subsection{The script itself}

The Han script itself may be the most ferocious \emph{nurikabe}. 
Figure~\ref{fig:tree} with its clean decomposition of ``language'' into
``speak,'' ``five,'' and ``mouth,'' is deceptive.  Many characters can be
described as simply as that, but many others cannot.  Consider
Figures~\ref{fig:forest}, \ref{fig:outlook}, \ref{fig:reach}, and I could
draw many more.

In Figure~\ref{fig:forest}, ``forest'' is two copies of ``tree'' placed side
by side.  But the ``tree'' on the left is different from the ``tree'' on the
right.  If you make the two sides of ``forest'' look identical, readers will
still know that you meant to write ``forest,'' but it will not look right. 
For a high-quality font, it has got to look right.  This entails either
creating two different primitives for the two trees, or having a smarter
tree that knows how to change itself when it is on the left.  Many character
components change when they appear on the left.  The modifications made when
a component appears on the left are partially systematic, so we might hope
to write code that can derive the left side shape automatically from the
other shape, but it will not be simple, it will require manual supervision,
and some projects have not gotten as far as noticing that it was an
issue in the first place.

\begin{figure}
\includegraphics[scale=0.70]{forest.pdf}
\caption{A forest is not two identical trees.}
\label{fig:forest}
\end{figure}

In Figure~\ref{fig:outlook}, the left side of ``outlook,'' in addition to
not being a character in its own right, is some kind of hard to describe
combination of ``arrow'' and ``old bird.'' It is not good enough to just
print a scaled copy of ``arrow'' on top of ``old bird'' and hope for the
best; getting it right requires modifying and deleting strokes in both
parts.  A generic overlap operation is unlikely to be flexible enough to do
the right thing here.  Every character that contains this sort of thing will
require specific human attention to adjust it beyond just saying
``overlap.'' If the components change parametrically, then making sure they
look right for all parameter values becomes even more complicated.

\begin{figure}
\includegraphics[scale=0.47]{outlook.pdf}
\caption{Combining operations are not always simple.}
\label{fig:outlook}
\end{figure}

In Figure~\ref{fig:reach}, two different styles of the same character are
topologically different: one contains a single zigzag stroke that in the
other is made up of two separate pieces.  It is not easy to parameterize
that in a way that will look good at every step in between, and if we
make it a binary choice, giving up on the idea of interpolation, this
difference will require some sort of ``if'' statement in the character
description.  A straightforward implementation of the grammar of shapes and
combining operations suggested by Figure~\ref{fig:tree} would not provide
for ``if'' statements.

\begin{figure}
\includegraphics[scale=0.63]{reach.pdf}
\caption{Two styles of U+53CA (\emph{oyo}, ``reach'').}
\label{fig:reach}
\end{figure}

These issues in the Han writing system point to an important conclusion:  a
simple grammar of parts and combining operations is not enough for building
parametric fonts, even though it may be a useful starting point.  Many
characters can be decomposed into parts in the clean way implied by
Figure~\ref{fig:tree}, and such decompositions may be enough to support
dictionary searches.  It is easy to find enough well-behaved characters to
put together a slide show or grant application, and to fool others or even
oneself into thinking the whole character set will be easy.

But in order to produce high-quality fonts with full parameterization, with
all the characters needed to typeset real documents, we must be able to
override the simple descriptions and combinations of parts in arbitrarily
complicated ways\Dash per character and depending non-linearly on the
parameters.  To work at full scale, the font description language must have
the power of a general-purpose programming language.

%%%%%%%%%%%%%%%%%%%%%%%%%%%%%%%%%%%%%%%%%%%%%%%%%%%%%%%%%%%%%%%%%%%%%%%%

\section{Tsukurimashou}

My own attempt at building a \MF\ \CJK\ font family is called the
Tsukurimashou Project.  The name means ``Let's
make something!''; it is an \emph{anime} reference.  As of version 0.8,
released 26 August 2013, Tsukurimashou covers 1502 Japanese \emph{kanji}
(Han script) characters including all those taught in Japanese schools
through Grade Four, as well as essentially complete coverage of \emph{kana}
(Japanese phonetic script), Latin, \emph{hangul} (Korean alphabetic script),
punctuation, and some miscellaneous ornaments and graphical characters. 
This is the work of one person, on a hobby basis while doing other things
full-time for pay, since late 2010.  It remains far from being a complete
font family usable for typesetting general documents in Japanese, but it is
already far past the point reached by any previous parameterized \MF-native
\CJK\ font project, and I believe my project is the first with a credible
prospect of eventually reaching complete coverage.

Here are some terms of reference distinguishing Tsukurimashou from other
projects already discussed:
\begin{itemize}
\item Tsukurimashou is a parameterized meta-family, not a single font or
a collection of independent fonts.
\item Tsukurimashou is a font project, not primarily a dictionary of
characters.
\item Tsukurimashou is code, not data.
\item Tsukurimashou is intended to achieve full coverage, at least of the
characters needed for basic literacy in Japanese; it is not a
proof of concept.
\item Tsukurimashou is one person's non-commercial project; not a for-profit
corporate or large-scale collaborative effort.
\end{itemize}

Tsukurimashou is hosted as a free software project on SourceForge Japan,
with the bilingual project home page at
\url{http://tsukurimashou.sourceforge.jp/} featuring downloadable packages,
a Subversion repository for the source code, a bug tracker, mailing list,
and so on.  The package as a whole is distributed under the \GNU\ General
Public License, version 3, with a clarifying paragraph added to explicitly
permit embedding the fonts in documents.

\subsection{Motivation}

The issues of human labour described in the previous section make it
difficult for a \CJK\ \MF\ project to reach complete coverage. 
Tsukurimashou's solution to the amount of work involved in font design is to
redefine that large amount of work as \emph{the main goal} of the project
instead of \emph{an unfortunate cost} of the project.  This point alone
seems to be largely responsible for Tsukurimashou's success to date.

I want to learn to read Japanese.  Learning to read entails spending some
time practicing and studying every character.  But just studying a book and
tracing copies on paper, as well as being boring, is not a particularly
effective way to learn.  I would also like to become skilled at using \MF\
and related font technologies.  I believe I acquire skills best by
completing tasks that require the skills.  Designing a font family for
Japanese, as a project that requires knowledge of the \emph{kanji} and of \MF,
including concentration on every character in turn, is a good way to acquire
that knowledge.  And from that point of view, the actual finished fonts are
not even important.  The fonts are my excuse for spending time thinking
about every character, which is the real goal.  With that goal in mind,
\emph{avoiding} human attention to every character stops being necessary or
even desirable.

Of course, the project may have desirable side effects.  Work on
Tsukurimashou has required me to invent new technology that may be useful in
other projects.  Some of it is publishable research in computer science,
certainly welcome for someone hoping to establish an academic career.  And
because it places heavy (in some cases unprecedented) demands on other free
software systems, Tsukurimashou has proven useful in the development of
those systems.  Given that I am already committing to spend some time per
character on learning the language, the hope is to make that time pay off in
as many ways as possible.

\subsection{A brief tour of the fonts}

Tsukurimashou as a software package generates OpenType font files as its
main output.  Those are intended for use in general typesetting and word
processing, not only within the \TeX\ world.  I most often use them with
\XeTeX.  The OpenType fonts are divided up into families, of which the main
supported ones are named Tsukurimashou, TsuIta, and Jieubsida; then there is
parameterization within each family for overall style, boldness, and
monospace or proportional spacing.  The main supported styles for the
Tsukurimashou family are ``Kaku'' (a traditional sans-serif style),
``Maru'' (sans-serif with rounded stroke ends),
``Mincho'' (a less traditional version of the common Mincho serif style),
and ``Bokukko'' (which somewhat resembles handwriting with a felt-tipped
pen).  Finer-grained parameters are used internally and could be made
visible by modifying the code, much in the way that Computer Modern has
internal parameters like ``\verb|stem_corr|'' as well as preset styles like
``Roman.'' Figure~\ref{fig:styles} shows a sample of the font styles;
Figure~\ref{fig:mincho} shows more of the Japanese characters in the Mincho
style.  Version 0.8
with all options enabled will build a total of 120 OpenType files, including
some that are experimental and not intended for actual use.

\begin{figure}
\includegraphics[scale=0.80]{styles.pdf}
\caption{A sample of the Tsukurimashou meta-family of fonts.}
\label{fig:styles}
\end{figure}

\begin{figure}
\includegraphics[scale=0.69]{mincho.pdf}
\caption{\emph{Kana} and Grade One \emph{kanji} in Tsukurimashou Mincho.}
\label{fig:mincho}
\end{figure}

These are outline fonts intended for high-resolution printing.  They contain
hinting for bitmap conversion, but it is done automatically and not expected
to be extremely high quality.  Japanese-language typesetting has
traditionally used monospace metrics, simple scaling (i.e., no corrections
for optical weight), and no slanting or italicization; Tsukurimashou
currently offers a choice between monospace or proportional, no optical
weight features, and italics for the Latin script only.

Although the largest use of Tsukurimashou fonts to date has been for
typesetting the project's own documentation in English, the design of the
Tsukurimashou Latin glyphs, especially in the Mincho style, is intended
primarily for setting the short fragments of English that sometimes occur in
Japanese text.  Tsukurimashou Mincho used for pure English text ends up
looking like a display face and might not be appropriate for entire
sentences and paragraphs.  Tsukurimashou Kaku is more suitable for extended
settings in English.

The Jieubsida\footnote{Intended as a translation to Korean of the name
``Tsukurimashou,'' but I am informed that ``Mandeubsida'' would be a better
translation, and am considering changing it.} family is intended to support
Korean \emph{hangul} (alphabetic) script.  \emph{Hanja} (the Korean
equivalent of \emph{kanji}) are not included.  This character set is
relatively orthogonal: the main sequence of 11172 glyphs is algorithmically
generated from a few tens of basic parts, though many less common letters
had to be defined with more human intervention.  Work on these fonts has
proven useful in debugging the infrastructure at full scale, given that the
Tsukurimashou series of fonts will eventually grow to a significant fraction
of the size already reached by the Jieubsida series.

Beyond the main Tsukurimashou package, there are several smaller software
packages called ``parasites,'' which appear in subdirectories of the
distribution or may be detached.  Some of these are font packages that share
some of the Tsukurimashou infrastructure without really being part of the
same meta-family; others are related software of other kinds.  The only one
discussed here will be the IDSgrep structural query system.

\subsection{The infrastructure}

Tsukurimashou's infrastructure is designed like a typical free software
project.  It has source code that compiles into binary files, it has build
scripts to accomplish that, and a would-be user can download a tarball,
unpack it, and type \texttt{./configure} and \texttt{make}.

The build system is based on \GNU\ Autotools.  Choosing which source code
files are needed for which font styles involves doing some logical inference
that would not be convenient to do in a Makefile, so the Makefiles invoke
additional code written in a subset of Prolog to evaluate the style
selections, then run Perl scripts that scan the \MF\ sources to look for
dependencies.  The results of that computation are written into additional
Makefiles, which guide the actual compilation process.

Knuth's \MF\ was designed with bitmap fonts in mind, whereas Tsukurimashou's
target is OpenType outline fonts.  There are several \MF\ variants that can
produce outline output from \MF\ source.  I chose
MetaType1~\cite{Jackowski:Programming} for
Tsukurimashou.  This package originates with the Polish \TeX\ users group
\acro{GUST}\ and may be most famous for its use in the Latin Modern
project~\cite{Jackowski:Latin}.  It consists
primarily of a macro package for Metapost and a postprocessing script for
\GNU\ \texttt{awk}.  One run of Metapost generates the glyphs of a font as
\EPS\ files; another generates metrics; then the \texttt{gawk} script merges
those and does some rewriting of the Postscript code to turn them into a
single Postscript Type 1 font.

In recent versions, Tsukurimashou's version of MetaType1 has diverged
somewhat from the one distributed by \acro{GUST}.  I started with the (very
old) \texttt{mtype13} distribution, tried to upgrade it to use the latest
MetaType1 scripts, and ended up rewriting large sections of code.  Many
features of MetaType1 are not used in Tsukurimashou (for instance, hinting;
the ``metrics'' pass; and the entire processing chain in the reverse
direction from Postscript back to \MF), and it proved useful to remove them,
streamlining the code considerably.  The core flow of information through
Tsukurimashou's version of MetaType1 remains similar to that of the
original, however: the Metapost interpreter executes code in the \MF\
language, writing one \EPS\ file for each glyph, and then those are
postprocessed into Postscript Type 1 fonts.

Each Postscript font contains up to 256 glyphs (but usually far fewer than
that), corresponding to a 256-character block of the Unicode character
space.  Many of these Postscript fonts are needed for each full-coverage
OpenType font.  The build system runs them individually through a FontForge
script that removes overlapping sections of splines, this being an easier
operation in FontForge than on the \MF\ side, and then once all Postscript
fonts for an OpenType font have had their overlaps removed, it runs another
FontForge script to combine them into the final OpenType font.  Doing the
overlap removal as a separate step is an optimization for the common case
during development where only some of the Postscript fonts have changed: it
reduces the amount of work needed to reassemble the updated OpenType font.

There are additional stages of processing in FontForge after the Postscript
fonts are merged.  The raw outlines generated by \MF\ may contain excessive
or poorly-located spline control points; scripts in FontForge attempt to
remove those.  Similarly, some technical rules of the font formats (such as
having points at the $x$ and $y$ extrema of each curve) need to be enforced. 
There is another processing chain for automated horizontal spacing and
kerning of the proportionally-spaced styles.  In that chain, the build
system generates bitmap fonts in \acro{BDF} format and a C program
calculates spacing corrections, which are then applied back to the merged
OpenType fonts.  Other scripts run on the side do things like constructing
OpenType glyph-substitution tables for Korean \emph{hangul} support, and
collecting data for proof generation.  According to recent statistics from
Ohloh~\cite{Ohloh:Languages}, 63\% of the project's code is written in
Metapost (the font descriptions proper), 8\% is in \LaTeX\ (documentation),
and the remaining 29\% is spread among 11 other programming languages: the
infrastructure and some small spin-off packages.

\subsection{The \MF\ code}

Here is Tsukurimashou's code defining the ``language''
glyph of Figure~\ref{fig:tree}; three styles of it are shown at the top of
Figure~\ref{fig:threestyle}.  This glyph is of about
average complexity; some are even simpler, and a few involve much more
complicated operations, such as calculating positions of strokes based on
the intersections of other strokes, or doing interpolation and conditional
processing on style parameters.

\begin{verbatim}
vardef kanji.grtwo.language =
  push_pbox_toexpand(
    "kanji.grtwo.language");
  build_kanji.level(build_kanji.lr(450,0)
    (kanji.grtwo.word)
    (tsu_xform(identity yscaled 0.95)
      (kanji.grnine.my)));
  expand_pbox;
enddef;
\end{verbatim}

\begin{figure}
\includegraphics[scale=0.63]{threestyle.pdf}
\caption{Three styles of ``language'' and ``five.''}
\label{fig:threestyle}
\end{figure}

This code exists in a file named \verb|tsuku-8a.mp|, which covers the
Unicode code points U+8A00 to U+8AFF.  A character like this one, which
happens not to be used as part of any other character, is defined right
there in the Unicode-range Metapost file.  Parts that are shared among more
than one such file are moved to other files that can be included in multiple
places; for instance, \verb|kanji.grtwo.word| is in \verb|gradetwo.mp|. 
Splitting macro definitions across many files like this makes it easier to
avoid recompiling the whole system when something changes, but it also
requires the build system to keep track of all the inter-file dependencies.

Tsukurimashou frequently uses a sort of functional programming via \MF's
concept of text arguments to macros.  There is a global stack data structure
of objects (several kinds) that will eventually be rendered into the glyph. 
A macro will receive one or more arguments that are themselves fragments of
code; it runs them, then examines the objects they added to the stack and
possibly makes modifications.  Macros that create \emph{kanji} or parts of \emph{kanji}
normally put them into a square of arbitrary two-dimensional space
defined by the coordinates from $(50,-50)$ to $(950,850)$; the outer-level
macros can then shift and scale that square into its final location in the
finished glyph.

The macro
\verb|build_kanji.lr|, for combining things left-to-right, allows its two
arguments to run, then scales and shifts their results to cover two smaller
rectangles.  The numeric arguments $(450,0)$ specify that in this case, the
dividing line is at $x$ coordinate $450$, and the two rectangles overlap by
an amount of $0$.  So the left side runs from $(50,-50)$ to $(450,850)$ and
the right side is from $(450,-50)$ to $(950,850)$.

Many of the visual adjustments needed when parts are combined, can be had
just by choosing the right values for the dividing line and overlap amount. 
But other macros seen in this sample include \verb|build_kanji.level|, which
adjusts the stroke widths in its argument to all be the same (which often,
but not always, looks better) and \verb|tsu_xform|, which applies an additional
\MF\ transformation matrix to make \verb|kanji.grnine.my| a little smaller.  Even
in this very simple glyph, some tweaking was necessary beyond just
putting together existing pieces in a standardized way.

Here is code for the \emph{kanji} numeral ``five,'' which is invoked indirectly by
\verb|kanji.grtwo.language| when it calls \verb|kanji.grnine.my|.  This
glyph is shown at the bottom of Figure~\ref{fig:threestyle}.  This is
typical of the basic shapes that are not made up of smaller components.
\begin{verbatim}
vardef kanji.grone.five =
  push_pbox_toexpand("kanji.grone.five");
  push_stroke((170,740)--(830,740),
    (1.6,1.6)--(1.6,1.6));
  set_boserif(0,1,9);
  push_stroke((500,740)--(350,20),
    (1.6,1.6)--(1.6,1.6));
  push_stroke(
    (220,410)--(730,410)--(720,20),
    (1.5,1.5)--(1.5,1.5)--(1.4,1.4));
  set_boserif(0,1,4);
  set_botip(0,1,1);
  push_stroke((120,20)--(880,20),
    (1.6,1.6)--(1.6,1.6));
  set_boserif(0,1,9);
  expand_pbox;
enddef;
\end{verbatim}

The \verb|push_stroke| macros save paths on the stack, with each stroke
defined by one path for the spine of the stroke, and a second path
describing how the stroke weight (eventually translated to ``width'' through
a style-dependent matrix) changes along the length of the stroke.  Other
macros, such as \verb|set_boserif|, push other objects on the stack to indicate
where serifs (\emph{uroko}) should be added in styles that use them.  The
whole thing, like \verb|kanji.grtwo.language| before it, is bracketed by
\verb|push_pbox_toexpand| and \verb|expand_pbox|, which respectively save, and adjust
the size of, an object called a ``proof box.''

After all the macros for a glyph have run, rendering code unwinds the stack
and generates outlines for all the objects, writing them to the Postscript
output.  This code is where most aspects of the font style are applied. 
Styles define the pens used for stroking, transformations for calculating
pen size, the shape of serifs and whether to use them, and can potentially
override parts of the rendering code by defining hook macros to apply
further effects.

I have never fully understood \MF's traditional proof system based on
greyscale fonts and ``literate'' programming, and in any case its reliance
on the standard coordinate array \verb|z[]| would not mix well with
Tsukurimashou's object stack concept.  Tsukurimashou generates proofs in a
completely different way.  When unwinding the stack the rendering code
writes a ``proof file,'' essentially a machine-readable log of all the
things it is rendering.  The build system collects the proof files and runs
them through Perl scripts which generate \TikZ/\LaTeX\ files for an
illustrated and cross-referenced edition of the source code.  The proof
boxes from \verb|push_pbox_toexpand| result in annotations on the pictures,
showing which part of each glyph came from which macro.  Some information
from the proof files also feeds into the kerning program, and is used for
purposes like advising FontForge of white-on-black reversed glyphs, which
represent exceptions to the overlap-removal rules otherwise
applied.

%%%%%%%%%%%%%%%%%%%%%%%%%%%%%%%%%%%%%%%%%%%%%%%%%%%%%%%%%%%%%%%%%%%%%%%%

\section{Character databases and IDSgrep}

Adding characters to Tsukurimashou requires knowing what is already in the
system and what is in the language: when looking at something like the left
side of ``outlook,'' I need to know whether such a thing already exists as a
macro somewhere in the code base; whether many other characters in the
language also include it, which would support the decision to create a new
macro for future use; and which of its parts may be related to common shapes
that could be used as guides for the new code.  There are also simple coding
questions like ``What was the name of that macro?'' and ``Which source code
file is it in?''

More generally, anyone working with Han characters who does not read them
fluently may wish to search a dictionary on partial descriptions: ``What
is this character I don't recognize that has `speak' on the left and `five'
at the upper right?'' Existing dictionaries sometimes offer what is called
``multi-radical'' search, whereby the user can specify one or more
components and then see a list of all \emph{kanji} that contain all those
components.  But multi-radical search features seldom if ever capture
structural information like ``on the left''; such a system would just show
all the characters that contain ``speak'' in one pile for the user to dig
through.  In the initial stages of laying out Tsukurimashou's \emph{kanji}
support, I frequently found myself wishing I could use the power of Unix
regular expressions, or something like them, to make more precise queries:
why not run \verb|grep| on the writing system itself?

The IDSgrep package attempts to serve that need.  With some irony intended,
IDSgrep's stated goal is to bring the user-friendliness of \verb|grep| to
Han character dictionaries.  IDSgrep is one of the Tsukurimashou parasites:
it comes included with the full distribution in a separate directory, or can
be distributed on its own.

Recall the tree decomposition of Figure~\ref{fig:tree}.  That tree might be
rendered into a simple \acro{ASCII}-based prefix notation as
``\verb|[lr](speak)[tb](five)(mouth)|'':  it is a left-right combination of
two things, the first of which is ``speak'' and the second is a top-bottom
combination of ``five'' and ``mouth.''  As argued earlier in this paper,
such descriptions are not enough to render high-quality glyphs; but maybe if
we include a few general catch-all categories like ``overlap,'' and accept
that not all descriptions will be detailed enough for rendering graphics, we
can come up with a description for every character sufficient to offer
useful dictionary searches.

The Unicode standard specifies syntax for Ideographic Description Sequences
(\acro{IDS}es), intended to support exactly this kind of
pursuit~\cite{Unicode:IDS}.  There are special characters defined in the
range U+2FF0 to U+2FFB to represent the prefix operators. 
Figure~\ref{fig:ids} shows some examples of the notation.  Note the way the
\acro{IDS} notation conceals some details: for instance, the two sides of
``forest'' are both denoted by the same character, even though they look
different when rendered.  This looks promising:  maybe we could get away
with ``just running \verb|grep|'' on a database of such decompositions.

\begin{figure}
\centering
\includegraphics[scale=1]{ids.pdf}
\caption{Unicode Ideographic Description Sequences.}
\label{fig:ids}
\end{figure}

In practice there are some additional challenges.  For theoretical reasons,
namely the difference between regular and context-free languages, a true
regular expression search on these descriptions may be less than
satisfactory.  IDSgrep implements a tree-matching query language in which
the user can specify character components to search for explicitly, or use
matching operators like wildcard, match-anywhere, Boolean operations, and so
on.  The \acro{IDS} syntax is not quite sufficiently flexible and
well-defined to encompass all the tasks IDSgrep demands of it, and the
special Unicode combining operation characters are difficult to type (and to
typeset in Computer Modern!); so IDSgrep defines extensions to the syntax
and \acro{ASCII} synonyms for the special characters,
forming a language of Extended Ideographic Description Sequences
(\acro{EIDS}es) that subsumes the Unicode \acro{IDS} syntax.

IDSgrep's user interface consists of a Unix command-line utility similar to
\verb|grep|.  It reads a database of trees in \acro{EIDS} syntax, from files
or standard input, and writes out any that match the matching pattern
specified on the command line:  just like \verb|grep|.  The syntax for
matching patterns is complicated because it is powerful, but no worse for
skilled users than standard regular expressions.  After learning the syntax,
a user can easily and quickly compose queries like ``What characters have
this component in that location, but not that other component anywhere?''

The latest version, IDSgrep 0.4, uses Bloom filters and binary decision
diagrams to speed up searches.  Although the full tree-matching algorithm is
not slow, a complete search of hundreds of thousands of \emph{kanji}
dictionary entries may take a few seconds.  So during installation, IDSgrep
precomputes bit vector indices for the databases being installed;
when searching those databases, it can do quick tests on the bit
vectors to reject the large majority of possible matches, running the more
expensive tree match on the candidates that make it past the bit vector
check.  The amount of speed-up is variable, but typically around a factor of
15.

But a critical question remains:  where does the data come from?  Databases
of \emph{kanji} marked up with structural data are not easy to find, let
alone in IDSgrep's native format.  The Tsukurimashou fonts generate (using
information extracted from the proof files) a dictionary of character
decompositions \emph{as the characters appear in the fonts}.  Querying how
Tsukurimashou decomposes a character is often useful, but Tsukurimashou by
definition does not cover the characters I have yet to add, and its
decompositions may not reflect traditional etymology and other concerns. 
IDSgrep also ships with code to extract \acro{EIDS} character decompositions
from the KanjiVG Project's \XML\ files~\cite{KanjiVG} and from the \acro{CHISE}
\acro{IDS} database~\cite{CHISE}.  It can do a ``join'' of any of the
\emph{kanji} databases with \acro{EDICT2}~\cite{EDICT2} to create an
experimental dictionary of words and meanings with character decompositions. 
None of these databases is perfect; but especially by searching several at
once, users can usually succeed in finding what they are looking for.

%%%%%%%%%%%%%%%%%%%%%%%%%%%%%%%%%%%%%%%%%%%%%%%%%%%%%%%%%%%%%%%%%%%%%%%%

\section{Conclusions and future work}

There has been much past \CJK\ \MF\ work, with few results and no finished
fonts.  I have described my own project, the Tsukurimashou parametric font
meta-family, which is unfinished too.  However, Tsukurimashou has made more
progress than any similar system to date.  I have described issues facing
this kind of project, Tsukurimashou's solutions for some of them, and
associated technology including the IDSgrep \emph{kanji} structural query
system.

The obvious direction for future work is to complete Tsukurimashou's
\emph{kanji} coverage.  My hope, however, is that some of the code and ideas
from this project will also be applicable in other languages and other
projects.

%%%%%%%%%%%%%%%%%%%%%%%%%%%%%%%%%%%%%%%%%%%%%%%%%%%%%%%%%%%%%%%%%%%%%%%%

\bibliographystyle{plain}
\bibliography{tsuku}

\makesignature
\end{document}
