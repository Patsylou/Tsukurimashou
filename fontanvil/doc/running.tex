
\chapter{Running FontAnvil}

FontAnvil is a script interpreter, so in normal operation it is assumed you
already have a script file for it to interpret.  Scripts are written in the
PE script language described elsewhere in this document.  Script files for
FontAnvil are traditionally given the filename extension \texttt{.pe} (for
``PfaEdit''); the extension \texttt{.ff} is also popular.  Invoking the
FontAnvil interpreter then proceeds on more or less the same lines as
invoking any other script interpreter.

\section{Command-line options}

\begin{framed}
FontAnvil's command-line syntax attempts to achieve some degree of FontForge
compatibility.  However, as of March~2014, FontForge contains at least six
different command-line
parsers,\footnotemark{} and
also sometimes hands its command lines off to Python for parsing, so that
options interact in complicated ways with each other, with compile-time
settings, with operating system shebang support and whether stdin is a
terminal or pipe, and so on.  FontAnvil does not attempt to match all of
this behaviour exactly.
\end{framed}
\footnotetext{https://github.com/fontforge/fontforge/issues/1277}


FontAnvil uses GNU \texttt{getopt\_long\_only} to parse command-line
arguments; this has the consequence that long options may be specified with
\texttt{-} (one hyphen) or \texttt{-{}-} (two hyphens) as the flag sequence;
using \texttt{-{}-} is the modern-day Unix convention, but \texttt{-} may be
preferable for FontForge compatibility.  Short options require a single
hyphen.  Options recognized are as follows.

\begin{description}
  \item[\texttt{-command} $\langle cmd\rangle$,
    \texttt{-c} $\langle cmd\rangle$]
  Execute a PE script command given literally on the command line.  FontAnvil
  will \emph{not} look for a script file name on the command line if this
  option is specified; all arguments starting with the first non-option
  argument become arguments to the script.  Only the last invocation of this
  option will be used; unlike, for instance, Perl, it is not possible to
  build up a multi-line script by specifying \texttt{-c} multiple times.

  \item[\texttt{-dry}, \texttt{-d}]
  Activate a poorly-documented ``dry run mode'' built into some parts of
  the FontForge PE script interpreter.  This appears to
  be intended for syntax checking.  \emph{Most}, but not necessarily
  \emph{all}, commands will be skipped.  \emph{I do not promise that new
  code added in FontAnvil will necessarily respect this mode.}

  \item[\texttt{-help}, \texttt{-usage}, \texttt{-h}]
  Display a command-line option help message, and terminate without
  executing a script.

  \item[\texttt{-lang} $\langle cmd\rangle$,
    \texttt{-l} $\langle cmd\rangle$]
  Specify interpreter language.  If this option is given with the value
  ``\texttt{ff}'' then it will be ignored for compatibility.  Any other
  value is a fatal error.

  \item[\texttt{-nosplash}, \texttt{-quiet}, \texttt{-script}, \texttt{-i}]
  Ignored for compatibility.

  \item[\texttt{-version}, \texttt{-v}]
  Display a version and copyright banner, and terminate without executing a
  script.

  \item[\texttt{-{}-}]
  Terminate option scanning.  All subsequent arguments will be treated as
  ``non-option arguments'' (thus eligible to become script file names or
  script arguments) even if they resemble FontAnvil options.  This would be
  what you might use if for some reason you needed to execute a script file
  that was named exactly ``\texttt{-script}.''
\end{description}

The first non-option command line argument will be taken as the filename of
a script file to execute, unless the \texttt{-c} option or one of its
synonyms has overridden this behaviour.  If the filename so specified is a
single hyphen, or if there are no non-option command line arguments
at all, then FontAnvil will enter \emph{interactive mode}, reading
commands from standard input, as described later in this chapter.  Any
command line arguments after any script filename will be passed
into the script in the variables \texttt{\$1},
\texttt{\$2}, and so on---even in the cases of \texttt{-c} and interactive
mode.

\begin{framed}
Option scanning stops at the first non-option argument encountered, which
will usually be treated as the script filename.  Any arguments after that
become arguments to the script (passed in the variables \texttt{\$1},
\texttt{\$2}, etc.) and not options for the FontAnvil interpreter.  For this
reason, options \emph{must} precede the script file name on the FontAnvil
command line.  For maximum compatibility, the \texttt{-script} option should
be the last option if you use it at all, with the script file name in a
separate argument, not attached using \texttt{=}.
\end{framed}

\section{Shebang}

FontAnvil may be invoked using the shebang convention.  Place a line
something like ``\texttt{\#!/usr/local/bin/fontanvil}'' at the top of a
file, and make the file executable, to create a script that can be run like
any other program and will automatically use FontAnvil as the interpreter.

Details of shebang support vary depending on the operating system.  On most
systems, the shebang line must specify an absolute path, and the
\texttt{env} program may be used to search for a command name in the path to
avoid hardcoding the absolute location of the interpreter into a script. 
There are also special considerations applicable to the length of the
interpreter path, arguments specified in the shebang line, and so on. 

\emph{FontAnvil does not have any special support for shebang.}  In
particular, it does not scan the script to look for its own name in the
shebang line.  Since the shebang line by definition starts with the comment
character \texttt{\#}, it will be skipped as a comment.  FontAnvil just
takes the script file name as an argument from the operating system, and
(assuming the script name does not happen to be something weird that looks
like an option) executes it, with any remaining arguments becoming arguments
to the script.  This is normally the desired behaviour.  However, be aware
that it is a technical difference from FontForge, which attempts to
determine whether it was invoked via the shebang mechanism and do smart
things depending the answer, including working around operating systems that
support this feature only poorly.  FontForge may possibly \emph{require}
options in the shebang line in at least some cases, to select which
scripting language it will use.

If you try to specify command-line options in the shebang line, then
depending on your operating system's support it is possible that FontAnvil
will not see the options even though FontForge would.  Some operating
systems have unintuitive behaviour regarding options specified in the
shebang line; for instance, combining all options into a single string
passed as one argument instead of splitting them on spaces.  For this
reason, authorities on Unix often recommend against using options in the
shebang line at all; nonetheless, people continue doing it.

For maximum compatibility with both interpreters, I suggest writing shebang
lines in PE script files as you would write them for FontForge (including
mentioning the filename ``\texttt{fontforge}''), and then invoking FontAnvil
on the files by other means when desired.  That way, FontForge will see the
interpreter name and any options it wants, and FontAnvil will ignore them.

\section{Interactive mode and readline}

FontAnvil is intended primarily for non-interactive use.  However, if it is
invoked without a script file name, or with ``\texttt{-}'' (a single hyphen)
as the script file name, then it will enter a special \emph{interactive
mode}, where it reads commands from standard input and executes them
immediately, line by line, rather than reading from a script file.  This can
be convenient for one-off editing tasks and testing the syntax and behaviour
of script commands.

If FontAnvil was compiled with the GNU Readline library and detects that
standard input is a terminal, then interactive mode will also offer
command-line editing and history using Readline.  The usual Readline
keystrokes (such as up and down arrows to recall earlier-typed command
lines) become available in this mode, and there are some minor changes to
the output formatting (in particular, the display of a command prompt) to
make it friendlier for interactive users.

