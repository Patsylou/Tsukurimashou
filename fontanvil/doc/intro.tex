% $Id: intro.tex 4013 2015-06-13 13:58:54Z mskala $

\chapter{Introduction}

FontAnvil is a script language interpreter for manipulating fonts. 
FontAnvil is substantially compatible with the PfaEdit/FontForge native
scripting language, but FontAnvil is intended for non-interactive use; for
instance, invocation from the build systems of font packages like
Tsukurimashou.  To better serve font package build systems in general and
Tsukurimashou in particular, FontAnvil has no GUI and, to a reasonable
extent, avoids dependencies on external packages.

There was a program called PfaEdit for editing fonts in Postscript ASCII
format (\texttt{.pfa} files).  PfaEdit development continued for many years,
it changed its name to FontForge, and it became the \emph{de facto} standard
font editing program in the free software community.  FontForge is still
under active development to this day.  The main focus of FontForge is on
interactive editing by GUI users, and the proportion of its code and
development effort dedicated to such users is large and growing.

PfaEdit had a scripting language, which as far as I know never had an
official name.  I will call it ``PE script'' in the interests of neutrality;
the traditional filename extension for files in this language is
\texttt{.pe}.  Development of PE script continued into the FontForge era. 
Many free font packages use PE scripts executed by FontForge to process font
files non-interactively in the context of a build system.  I myself maintain
the Tsukurimashou Project (\url{http://tsukurimashou.osdn.jp/}),
which processes fonts using PE scripts on a massive scale (thousands of
script invocations per build).

Attempting to use a large and steadily-growing GUI program as a
non-interactive script language interpreter is not always convenient.  The
many external libraries needed to build FontForge implicitly become
dependencies of any font package that needs FontForge for its build system;
and it is not easy to get users to install them all correctly just to build
a font package.  It is also hard to predict whether a given version of
FontForge will actually work for a given script: with GUI enhancements, and
even social network features, as the strategic priorities for FontForge
development, it is frequently the case that the stable and correct execution
of scripts is not at the top of the priority list, and bugs in script
processing are fixed late if at all.

There was a recent proposal to remove PE script from FontForge entirely,
which if adopted would be fatal to systems that depend on it, and not at all
adequately addressed by the concommitant proposal to encourage current PE
script users to ``upgrade'' to Python.  That particular proposal was not
immediately implemented, but it seems clear that the writing is on the wall
for continued FontForge support of PE script.  The mere \emph{risk} of
possibly losing PE scripting in up-to-date versions of widely-used Linux
distributions at some unspecified point in the future is a big problem for
the Tsukurimashou Project.  FontForge for all its good points can no longer
be thought of as a stable platform for non-interactive font manipulation;
its priorities are other things entirely.  But my own project absolutely
requires a stable platform for non-interactive font manipulation.  FontAnvil
is intended to fill the gap.

FontAnvil's development is driven by the needs of the Tsukurimashou Project,
but my hope is that it will also be useful to the many other projects
currently using PE scripts for non-interactive font processing.  I am also a
member of the FontForge development team, and I hope that FontAnvil's
existence will reduce the pressure on the FontForge team to continue
development of code that is outside most of the team's interests and
expertise.  Splitting the PE script interpreter into a standalone package
should be beneficial to almost everyone involved.

Some plans and goals for FontAnvil are:
\begin{itemize}
\item Whatever Tsukurimashou needs from a font manipulation script
  interpreter.
\item Return statements indented to the same level as the surrounding code.
\item A ``remove overlaps'' command that works.
\item No GUI, Python, or exotic dependencies.
\item No recursive Autotools.
\item Correct memory management.
\item Simple directory and linking structure (in particular, no
  unnecessary shared libraries).
\item Main code repository in Subversion.
\end{itemize}

FontAnvil's departure point from FontForge was at this tagged revision on
Github: \url{https://github.com/mskala/fontforge/releases/tag/fontanvil}. 
That is not a mainline FontForge revision; it was synthesized by merging the
mainline master as of roughly February 5, 2014 with a few patches from other
branches that were current as of the beginning of March.  From there I
copied the code into a private Subversion server (private because I don't
want to publish some intermediate revisions that lack proper copyright
notices) and ripped out most of the code that was not required by FontAnvil,
that being the majority of the package.  I simplified the structure of the
package and the build system along the way.  Removing the last traces of
dead code will be a long-term project.  The first public revision of
FontAnvil was added as a subdirectory to the Tsukurimashou Project's
Subversion server on March 4, 2014.  Future releases will be available through
Tsukurimashou's Web site.

Since FontAnvil and FontForge are free software under compatible license
terms and share many potential users and one developer, there is some
possibility for cross-pollination and sharing of code and ideas between the
two.  However, I do not think it is likely that I will spend much time
trying to import future development from FontForge into FontAnvil, nor that
the FontForge team will spend much time trying to import future development
from FontAnvil into FontForge.  Not much future development on either
project will be particularly relevant to the other.  The two projects have
different goals and policies and are likely to diverge.  However, since PE
script is mature and neither side is likely to drastically change the
language, I think it is likely that for the most part FontAnvil and
FontForge will be able to run each other's scripts for as long as FontForge
chooses to include a PE script interpreter.

Of all the parts of a forge, an anvil is simple,
an anvil is trustworthy, and most of all, \emph{an anvil is stable}.

\vspace{1cm}

\noindent
Matthew Skala\\
mskala@ansuz.sooke.bc.ca\\
\url{http://ansuz.sooke.bc.ca/}

\clearpage

