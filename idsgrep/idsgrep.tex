\documentclass[twocolumn]{report}

\usepackage{fontspec}

\usepackage{calc}
\usepackage[rigidchapters]{titlesec}
\usepackage{tocloft}
\usepackage{url}

\title{IDSgrep, version \idsgrepversion}
\author{Matthew Skala}

\input{config.tex}

\setsansfont{Nimbus Sans L}

\setlength{\topmargin}{0.25in}
\setlength{\headheight}{0pt}
\setlength{\headsep}{0pt}

\setlength{\oddsidemargin}{\paperwidth/17-0.5in}
\setlength{\columnsep}{1em}
\setlength{\textwidth}{\paperwidth*15/17-1in}
\setlength{\textheight}{8.5in}

\setlength{\parindent}{1.5em}

\makeatletter
\dimen@=\f@size\p@\dimen@6\dimen@\divide\dimen@5\edef\l@rgesize{\the\dimen@}
\dimen@=\f@size\p@\dimen@2\dimen@\edef\@hugesize{\the\dimen@}

\renewcommand\maketitle{%
  \begin{titlepage}%
    \let\footnotesize\small
    \let\footnoterule\relax
    \let \footnote \thanks
    \vspace*{\fill}
    \vspace*{\fill}
    {\sffamily\bfseries\huge \hfill\@title\hfill\null\par}
    \vspace{\fill}
    {\sffamily\bfseries\Large \hfill\@author\hfill\null\par}
    \vspace{\fill}
    \vspace{\fill}
    \vspace{\fill}
    \vspace{\fill}
    \vspace{\fill}
    {\sffamily\bfseries\Large \hfill\@date\par}
    \vspace*{\fill}
    \@thanks\par
    \null
  \end{titlepage}%
  \setcounter{footnote}{0}%
  \global\let\thanks\relax
  \global\let\maketitle\relax
  \global\let\@thanks\@empty
  \global\let\@author\@empty
  \global\let\@date\@empty
  \global\let\@title\@empty
  \global\let\title\relax
  \global\let\author\relax
  \global\let\date\relax
  \global\let\and\relax
}

\def\@maketitle{%
  \newpage\noindent
  \null
  \begingroup
    \let\footnote\thanks
    {\fontsize{\@hugesize}{2\baselineskip}\sffamily\bfseries\selectfont
      \@title\,\leaders\hrule height 0.2ex\hfill\null}\par\noindent
    {\fontsize{\l@rgesize}{\baselineskip}\sffamily\bfseries
      \@author\hfill\@date}\par
  \endgroup
  \dimen@=0.5\baselineskip\relax\advance\dimen@-0.5\p@\relax
  \vspace{\dimen@}\noindent
}

\newcommand{\ch@pterform@t}[1]{%
  \begin{@twocolumnfalse}%
    \fontsize{\@hugesize}{3\baselineskip}\sffamily\bfseries\selectfont
    #1\,\leaders\hrule height 0.2ex\hfill\null
  \end{@twocolumnfalse}%
}%
\titlespacing*{\chapter}{0pt}{6\baselineskip}{2\baselineskip}
\titleformat{\chapter}[hang]{}{}{0pt}{\ch@pterform@t}

\titleformat{\section}[runin]%
   {\fontsize{\l@rgesize}{\baselineskip}\sffamily\bfseries}{}{0pt}{}%
   [\,\leaders\hrule height 0.16ex\hfill\null\\]
\titlespacing*{\section}{0pt}{\baselineskip}{0pt}

\titleformat{\subsection}[runin]{\sffamily\bfseries}{}{0pt}{}
\titlespacing*{\subsection}{0pt}{\baselineskip}{0.666em}

\titleformat{\subsubsection}[runin]{\rmfamily\scshape}{}{0pt}{}
\titlespacing*{\subsubsection}{0pt}{\baselineskip}{0.666em}

\titleformat{\paragraph}[runin]{\rmfamily\itshape}{}{0pt}{}
\titlespacing*{\paragraph}{\parindent}{0pt}{0.666em}

\raggedbottom
\makeatother

\renewcommand{\thefootnote}{\fnsymbol{footnote}}

%%%%%%%%%%%%%%%%%%%%%%%%%%%%%%%%%%%%%%%%%%%%%%%%%%%%%%%%%%%%%%%%%%%%%%%%
%%%%%%%%%%%%%%%%%%%%%%%%%%%%%%%%%%%%%%%%%%%%%%%%%%%%%%%%%%%%%%%%%%%%%%%%
%%%%%%%%%%%%%%%%%%%%%%%%%%%%%%%%%%%%%%%%%%%%%%%%%%%%%%%%%%%%%%%%%%%%%%%%

\begin{document}

\maketitle

\renewcommand{\cfttoctitlefont}{%
  \huge\sffamily\bfseries}
\renewcommand{\cftaftertoctitle}{%
  {\huge\,\leaders\hrule height 0.2ex\hfill\null\vspace*{-4ex}}}

\def\gobbtohfil#1{%
  \begingroup\if#1\hfil\else\aftergroup\gobbtohfil\fi\endgroup}

\renewcommand{\cftchappresnum}{\gobbtohfil}
\renewcommand{\cftchapnumwidth}{0pt}
\renewcommand{\cftchapfont}{\sffamily\bfseries}
\renewcommand{\cftchappagefont}{\sffamily\bfseries}

\renewcommand{\cftsecpresnum}{\gobbtohfil}
\renewcommand{\cftsecnumwidth}{0pt}

\renewcommand{\cftsubsecpresnum}{\gobbtohfil}
\renewcommand{\cftsubsecnumwidth}{0pt}

\tableofcontents

%%%%%%%%%%%%%%%%%%%%%%%%%%%%%%%%%%%%%%%%%%%%%%%%%%%%%%%%%%%%%%%%%%%%%%%%
%%%%%%%%%%%%%%%%%%%%%%%%%%%%%%%%%%%%%%%%%%%%%%%%%%%%%%%%%%%%%%%%%%%%%%%%
%%%%%%%%%%%%%%%%%%%%%%%%%%%%%%%%%%%%%%%%%%%%%%%%%%%%%%%%%%%%%%%%%%%%%%%%

\chapter{Quick start}

\noindent
Use \texttt{idsgrep} much as you would use \texttt{grep}:

\begin{quotation}
  \texttt{idsgrep}
    \textit{$[\langle$options$\rangle]$}
    \textit{$\langle$pattern$\rangle$}
    \textit{$[\langle$file$\rangle\ldots ]$}
\end{quotation}

Its general function is to search one or more files for items matching a
pattern, like \texttt{grep}~\cite{grep} but with a different pattern
syntax.  Although potentially usable for an unlimited range of tasks,
\texttt{idsgrep}'s motivating application is to searching databases of Han
script (Chinese, Japanese, etc.)\ character descriptions.  It provides a
much more powerful replacement for the ``radical search'' feature of
dictionaries like Kiten~\cite{Kiten} and WWWJDIC~\cite{WWWJDIC}.

The syntax for matching patterns, and the range of command-line options
available, are complicated.  Later sections of this document explain those
things in detail; for now, here are some examples.

\begin{description}
\item[\texttt{idsgrep 萌 dictionary}]~\\
  A literal character searches for the decomposition of that character,
  exact match only.
\item[\texttt{idsgrep -d 萌}]~\\
  The \texttt{-d} option with empty argument searches a default collection
  of dictionaries.
\item[\texttt{idsgrep -dtsuku 萌}]~\\
  The \texttt{-d} option can be given an argument to choose a specific
  default dictionary.  Note the argument must be given in the same
  \texttt{argv}-element with the \texttt{-d}; the syntax \texttt{-d tsuku}
  with a space would mean ``Use the default dictionaries and search for the
  (syntactically invalid) pattern `\texttt{tsuku}.'\,''
\item[\texttt{othersoft | idsgrep 萌}]~\\
  Standard input will be used if no other input source is specified.
\item[\texttt{idsgrep -d ...日}]~\\
  Three dots match their argument anywhere, so this will match \texttt{日},
  \texttt{早}, and \texttt{萌}.
\item[\texttt{idsgrep -d '?'}]~\\
  A question mark, which will probably require shell escaping, matches
  anything.  This is most useful as part of a more complex pattern.
\item[\texttt{idsgrep -d '⿱?心'}]~\\
  Unicode Ideographic Description Characters can be used to build up
  sequences that also
  incorporate the wildcards; this example matches characters
  consisting of something above \texttt{心}, such as \texttt{忽} and
  \texttt{恋} but not \texttt{応}.
\item[\texttt{idsgrep -d '[tb](anything)心'}]~\\
  There are ASCII aliases
  for operators that may be inconvenient to type; this query is
  functionally the same as the previous one.
\end{description}

%%%%%%%%%%%%%%%%%%%%%%%%%%%%%%%%%%%%%%%%%%%%%%%%%%%%%%%%%%%%%%%%%%%%%%%%
%%%%%%%%%%%%%%%%%%%%%%%%%%%%%%%%%%%%%%%%%%%%%%%%%%%%%%%%%%%%%%%%%%%%%%%%
%%%%%%%%%%%%%%%%%%%%%%%%%%%%%%%%%%%%%%%%%%%%%%%%%%%%%%%%%%%%%%%%%%%%%%%%

\chapter{Introduction}

\noindent
The Han character set is open-ended.  Although a few thousand characters
suffice to write the languages most commonly written in Han script languages
(namely Chinese and Japanese) most of the time, popular standards define
tens of thousands of less-popular characters, and there are at least
hundreds of thousands of rare characters known to occur in names, historical
contexts, and in languages like Korean and Vietnamese that may still use Han
script occasionally despite now being written primarily in other scripts.

Computer text processing systems that use fixed lists of characters will
inevitably find themselves unable to represent some text.  As a result,
there is a need to \emph{describe} characters in a standard way that may
have no standard code points of their own.  A similar need for descriptions
of characters arises when looking up characters in a dictionary; a user may
recognize some or all the visual features of a character (such as its parts
and the way they are laid out) without knowing how to enter the character as
a whole.

IDSgrep's main function is to query character description databases in a
flexible way.  This need was identified during development of the
Tsukurimashou font family~\cite{Tsukurimashou}; there, the visual appearance
of Han character glyphs corresponds directly to the MetaPost code
implementing them, and the desire for code re-use and consistency often
motivates a close examination of the existing work to answer questions like
``What other characters contain this shape, and how did we implement it last
time?'' Standard tools like \texttt{grep}~\cite{grep} can sometimes be
applied to answer such questions by searching for subroutine names in the
source code, but the related question of ``What other characters, not yet
implemented, could we build that would use this shape?''\ requires comparing
with some external database of the characters commonly used in the language. 
How can we run \texttt{grep} on the writing system itself?

Someone confronted with an unknown character and wanting to look it up in a
more ordinary dictionary to find the meaning may, similarly, want to search
for characters based on specific features while leaving others unspecified,
with questions like ``What characters exist that have the common \texttt{心}
shape at the bottom, with the upper part divided into two things side by
side?  The two things at the top are shapes I don't recognize, printed too
small for me to identify them more precisely.'' Existing dictionary-query
methods are not adequate for some reasonable queries of this nature.

For instance, the radical-and-stroke-count method of traditional character
dictionaries requires identifying the head radical and counting strokes,
both of which may be difficult; dictionary codes like SKIP and Four Corners
key on some layout attributes but not all; multi-radical search allows the
user to choose whichever radicals they recognize, but it ignores layout
entirely; and computer handwriting recognition generally works well if and
only if the user is sure of the writing of the first few strokes in the
character.  Furthermore, these search schemes often are implemented only in
heavy, non-portable, GUI software that cannot be automated and mixes poorly
with standard computing environments.  IDSgrep, even in its current alpha
version with most features unimplemented, can answer the example query
correctly with a single, simple command line (\texttt{idsgrep -d
'[tb][lr]??心'}).  This software is intended to bring the user-friendliness
of \texttt{grep} to Han character dictionaries.

%%%%%%%%%%%%%%%%%%%%%%%%%%%%%%%%%%%%%%%%%%%%%%%%%%%%%%%%%%%%%%%%%%%%%%%%

\section{Download, build, and install}

IDSgrep is distributed under the umbrella of the Tsukurimashou project on
Sourceforge.JP~\cite{Tsukurimashou},
\url{http://en.sourceforge.jp/projects/tsukurimashou/}.  Releases of IDSgrep
will appear on the project download page; development versions are available
by SVN checkout from the \texttt{trunk/idsgrep} subdirectory of the
repository.

A minimal default build and install could run something like this:
\begin{verbatim}
tar -xzvf idsgrep-0.2.tar.gz
cd idsgrep-0.2
./configure
make
su -c 'make install'
\end{verbatim}

IDSgrep as such does not include a dictionary, but it can build dictionaries
from the Tsukurimashou font package, which is available through the same
Sourceforge.JP project as IDSgrep, or from the KanjiVG database available at
\url{http://kanjivg.tagaini.net/}~\cite{KanjiVG}.  For an ideal complete
installation of IDSgrep, one would download both these packages, build
Tsukurimashou first, and make it and KanjiVG available to the IDSgrep
\texttt{configure} script.  The \texttt{configure} script will by default
make a reasonable effort to find KanjiVG and Tsukurimashou; in many common
cases it is not necessary to specify them on the command line.  Here is a
more complete installation process relying on \texttt{configure} to find
KanjiVG in the current directory and Tsukurimashou in a sibling directory:
\begin{verbatim} unzip tsukurimashou-0.6.zip cd tsukurimashou-0.6
./configure
make
# install of Tsukurimashou not needed by IDSgrep
cd ..
tar -xzvf idsgrep-0.2.tar.gz
cd idsgrep-0.2
ln -s /some/where/else/kanjivg-20111029.xml.gz .
./configure
make
su -c 'make install'
\end{verbatim}

If the default search fails, the filename of KanjiVG (\texttt{.xml} or
\texttt{.xml.gz}) and the top directory of Tsukurimashou can be specified on
the \texttt{configure} command line with the \texttt{--with-kanjivg} and
\texttt{--with-tsuku-build} options.  For other options, run
\texttt{configure --help}.  It's a reasonably standard GNU
Autotools~\cite{Autotools} configuration script, albeit with a lot of
options for inapplicable installation directories removed to simplify the
help message.

%%%%%%%%%%%%%%%%%%%%%%%%%%%%%%%%%%%%%%%%%%%%%%%%%%%%%%%%%%%%%%%%%%%%%%%%

\section{Interface to KanjiVG}

The KanjiVG project~\cite{KanjiVG} maintains a database of kanji (Han
characters as used by Japanese) in an extended SVG format, which implies
that it is XML.  The \texttt{kvg2eids} Perl script, included as part of
IDSgrep, is capable of reading this database and converting it to Extended
Ideographic Description Sequences (EIDSes).  As described above, if a
reasonably recent version of KanjiVG's compressed XML file is available to
\texttt{configure}, then IDSgrep's build will create such a dictionary and
\texttt{make install} will install it.

KanjiVG describes characters primarily in terms of strokes, not radicals,
and it attempts to follow the official stroke order and etymological
radical breakdown.  That approach results in some peculiarities from the
point of view of dictionary searching.  For instance, in the kanji
\texttt{園}, the official stroke order is to write two strokes of the
enclosing box, then the central glyph, then the bottom of the box. 
KanjiVG's XML file lists two ``elements'' identified with the kanji
\texttt{囗}, one for the first two strokes and one for the final stroke,
with additional attributes specifying that they are actually two parts of
the same element.  KanjiVG has changed its own standard for how to represent
this information in the recent past, and not all entries have been updated
to the latest standard yet.  The current version of \texttt{kvg2eids} does
not correctly process \texttt{園} nor some other characters with parts
written in nonsequential order.  On that particular one it generates a
special functor containing debugging information; for some others, it may
actually generate an EIDS with the same radical appearing multiple times,
following the structure described in KanjiVG whether it's what was intended
or not.  As a result, not all entries in the dictionary will be right. 
However, only a few are affected by this issue.

With the current versions of IDSgrep and KanjiVG, the KanjiVG-derived
dictionary contains 6660 entries covering all the popularly-used Japanese
kanji.  Note that the KanjiVG input file, and presumably the resulting
format-converted dictionary, are covered by a Creative Commons
Attribution--ShareAlike license, distinct from the GNU GPL applicable to
IDSgrep itself.

%%%%%%%%%%%%%%%%%%%%%%%%%%%%%%%%%%%%%%%%%%%%%%%%%%%%%%%%%%%%%%%%%%%%%%%%

\section{Interface to Tsukurimashou}

IDSgrep is closely connected with the Tsukuimashou font
family~\cite{Tsukurimashou}.  They have the same author; it was largely for
use in Tsukurimashou development that IDSgrep was developed at all; and
IDSgrep's source control system is a subdirectory within Tsukurimashou's. 
Building IDSgrep in conjunction with Tsukurimashou allows IDSgrep to extract
from the Tsukurimashou build system a dictionary of character decompositions
as they appear in Tsukurimashou.  The Tsukurimashou fonts are also necessary
to build this IDSgrep user manual. However, IDSgrep and Tsukurimashou
are distributed as separate packages, because they have very different
audiences and build prerequisites.  Many people who can use one will be
unable to use the other, so it seems inappropriate to force all users to
download both.

When IDSgrep's \texttt{configure} script runs, it looks for a valid
Tsukurimashou build directory.  Ideally, that would be one in which
Tsukurimashou has actually been fully built; but a directory where the
Tsukurimashou \texttt{configure} script has been executed is enough.  If a
valid Tsukurimashou build directory is found automatically or specified with
the \texttt{--with-tsuku-build} option to \texttt{configure}, then when
\texttt{make} is run on IDSgrep, it will recursively go call \texttt{make
eids} in the Tsukurimashou build.  That is a hook that causes
Tsukurimashou's build system to generate the EIDS decomposition dictionary,
which is then copied or linked back into IDSgrep's build directory and can
be installed with IDSgrep's \texttt{make install}.  IDSgrep's build will
also look in Tsukurimashou's build directory for the font ``Tsukurimashou
Mincho'' which is needed to build this user manual, and will make
recursive calls to \texttt{make} for Tsukurimashou to build that if
necessary.

Note that neither Tsukurimashou nor IDSgrep is a true ``sub-package'' of the
other in the sense of Autotools~\cite{Autotools}, as mediated by the
\texttt{SUBDIRS} Automake variable and so on, notwithstanding that a
checked-out SVN working copy of Tsukurimashou will contain a working copy of
IDSgrep in a subdirectory.  Running the Tsukurimashou build will not invoke
the IDSgrep build at all; and running the IDSgrep build is not a good way to
trigger a full Tsukurimashou build, because it won't use the preferred
\texttt{-j} option, track all dependencies in detail, nor generate anything
that doesn't happen to be a prerequisite for the files IDSgrep needs.  If
you want to build both systems, it's best to build Tsukurimashou first and
then build IDSgrep pointing at Tsukurimashou.  Also, these two packages do
not necessarily have the same portability considerations, and it's possible
that the link between them may fail even on systems where each package
builds correctly by itself (for instance, possibly on some systems where GNU
Make is installed but non-default).  The link between Tsukurimashou and
IDSgrep provides some convenience for my own frequent case of making changes
to both packages at once.

In order for IDSgrep to work together with Tsukurimashou, it is necessary
that the Tsukurimashou build be one that supports the \texttt{make eids}
target in the first place.  No released version contains such support yet,
but it is planned for Tsukurimashou~0.6.  Development versions of
Tsukurimashou in the SVN repository have included EIDS support since early
January 2012.

%%%%%%%%%%%%%%%%%%%%%%%%%%%%%%%%%%%%%%%%%%%%%%%%%%%%%%%%%%%%%%%%%%%%%%%%

\section{Unicode IDSes}

Although \texttt{idsgrep} uses a more elaborate syntax, it is well to know
about the Unicode Consortium's ``Ideographic Description Sequences''
(IDSes), which are a subset of \texttt{idsgrep}'s.  These are documented
more fully in the Unicode standard~\cite{Unicode:IDS}.

\begin{itemize}

\item A character from one of the Unified Han or CJK Radical ranges is a
complete IDS and simply represents itself.  For instance, ``\texttt{大}'' is
a complete IDS.

\item The Ideographic Description Characer (IDC) code points U+2FF0, U+2FF1,
and U+2FF4 through U+2FFB, whose graphic images look like
\texttt{⿰⿱⿴⿵⿶⿷⿸⿹⿺⿻}, are prefix binary operators.
One of these characters followed by two complete IDSes
forms another complete IDS, representing a character formed by joining the
two smaller characters in a way suggested by the name and graphical image
of the IDC.  For instance, ``\texttt{⿰日月}'' describes the
character \texttt{明}.  These structures may be nested; for instance,
``\texttt{⿰言⿱五口}'' describes the character \texttt{語}.

\item The IDC code points U+2FF2 and U+2FF3, which look like \texttt{⿲⿳},
are prefix ternary operators. (Unicode uses the less-standard word
``trinary'' to describe them.) One of them can be followed by three complete
IDSes to form an IDS that describes a character made of three parts, much in
the same manner as the binary operators.  For instance,
``\texttt{⿱⿲糸言糸夂}'' describes the character \texttt{變}.

\item An IDS may not be more than 16 code points long overall nor contain more
than six consecutive non-operator characters.
This rule appears to be intended to make things easier for systems that need
to be able to jump into the middle of text and quickly find the starts and
ends of IDSes.

\item IDSes non-bindingly ``should'' be as short as possible.

\end{itemize}

%%%%%%%%%%%%%%%%%%%%%%%%%%%%%%%%%%%%%%%%%%%%%%%%%%%%%%%%%%%%%%%%%%%%%%%%
%%%%%%%%%%%%%%%%%%%%%%%%%%%%%%%%%%%%%%%%%%%%%%%%%%%%%%%%%%%%%%%%%%%%%%%%
%%%%%%%%%%%%%%%%%%%%%%%%%%%%%%%%%%%%%%%%%%%%%%%%%%%%%%%%%%%%%%%%%%%%%%%%

\chapter{Invoking \texttt{idsgrep}}

The command-line \texttt{idsgrep} utility works much like most other
command-line programs, and like \texttt{grep}~\cite{grep} in particular.  It
takes options and other arguments.  The first non-option argument is an EIDS
representing the matching pattern, and any remaining non-option arguments
are taken as filenames to read.  If there are no filenames, \texttt{idsgrep}
will read from standard input.  Output always goes to standard output.

When there is more than one file being read (either by direct specification
or indirectly with the \texttt{-d} dictionary option), \texttt{idsgrep} will
preface each EIDS in its output with
``\texttt{:}$\langle$\textit{filename}$\rangle$\texttt{:}'' to indicate
in which file the EIDS was found.  Note that under the EIDS syntax rules,
that creates a unary node senior to the entire tree, so that the output
remains in valid EIDS format, except in the case of filenames containing
colons, which will be handled via backslash escapes in the future when those
are implemented.

%%%%%%%%%%%%%%%%%%%%%%%%%%%%%%%%%%%%%%%%%%%%%%%%%%%%%%%%%%%%%%%%%%%%%%%%

\section{Command-line options}

\begin{description}

\item[\texttt{-d}, \texttt{--dictionary}]
Read a dictionary
from the standard location.  There is a pathname for dictionaries hardcoded
into the \texttt{idsgrep} binary, generally
\{\emph{prefix}\}\texttt{/share/dict}, and if this option is given, its
argument (which may be empty) will be appended to the dictionary directory
path, followed by ``\texttt{*.eids},'' and then treated as a shell glob
pattern.  Any matching files are then searched in addition to those
otherwise specified on the command line.  A small added wrinkle is that when
more than one file is searched (resulting in
\texttt{:}\textit{filename}\texttt{:} tags on the output lines), any of them
that came from the \texttt{-d} option will be abbreviated by omitting the
hardcoded path name.  The purpose of this option is to cover the common case
of searching the installed dictionaries.  Just specifying ``\texttt{-d}''
will search all the installed dictionaries; specifying an abbreviation of
the dictionary name, as ``\texttt{-dt}'' or ``\texttt{-dk},'' will search
just the matching one; and it remains possible to specify a file exactly or
use standard input in the usual \texttt{grep}-like way.

\item[\texttt{-V}, \texttt{--version}] Display the version and license
information for IDSgrep.

\item[\texttt{-h}, \texttt{--help}] Display a short summary of these
options.

\end{description}

%%%%%%%%%%%%%%%%%%%%%%%%%%%%%%%%%%%%%%%%%%%%%%%%%%%%%%%%%%%%%%%%%%%%%%%%

\section{Environment variables}

The \texttt{idsgrep} utility recognizes just one environment variable,
\texttt{IDSGREP\_DICTDIR}, which if present specifies a directory for the
\texttt{-d} option to search instead of its hardcoded default.

Note that \texttt{idsgrep} does not pay attention to any other environment
variables, and in particular, not \texttt{LC\_ALL} and company.  The input
and output of this program are always UTF-8 encoded Unicode \emph{regardless
of locale settings}.  Since the basic function of this program is closely
tied to the Unicode-specific ``ideographic description characters,'' it
would be difficult if not impossible for it to work in any non-Unicode
locale.  Predictability is also important because of the likely usefulness
of this software in automated contexts; if it followed locale environment
variables, many users would have to carefully override those all the time to
be sure of portability.  Instead of creating that situation,
\texttt{idsgrep} by design has a consistent input and output format on all
systems and users are welcome to pipe things through
a conversion program if necessary.

%%%%%%%%%%%%%%%%%%%%%%%%%%%%%%%%%%%%%%%%%%%%%%%%%%%%%%%%%%%%%%%%%%%%%%%%
%%%%%%%%%%%%%%%%%%%%%%%%%%%%%%%%%%%%%%%%%%%%%%%%%%%%%%%%%%%%%%%%%%%%%%%%
%%%%%%%%%%%%%%%%%%%%%%%%%%%%%%%%%%%%%%%%%%%%%%%%%%%%%%%%%%%%%%%%%%%%%%%%

\chapter{Technical details}

\noindent

This section is intended to describe IDSgrep's syntax and matching procedure
in complete detail; and those things are, in turn, designed to be powerful
rather than easy.  As a result, the description may be confusing for some
users.  See the examples in the ``Quick start'' section for a more
accessible introduction to how to use the utility.

The system is best understood in terms of three interconnected major
concepts:
\begin{itemize}
  \item an abstract data structure;
  \item a syntax for expressing instances of the data structure as
    ``Extended Ideographic Description Sequences'' (EIDSes);
  \item a function for determining whether two instances of the data
    structure ``match.''
\end{itemize}

Then the basic function of \texttt{idsgrep} is to take one EIDS as a
matching pattern, scan a file containing many more, and write out the ones
that match the matching pattern.  The three major concepts are described,
one each, in the following sections.

%%%%%%%%%%%%%%%%%%%%%%%%%%%%%%%%%%%%%%%%%%%%%%%%%%%%%%%%%%%%%%%%%%%%%%%%

\section{The data structure}

An \emph{EIDS tree} consists of the following:

\begin{itemize}
  \item An optional \emph{head}, which if present consists of a nonempty
    string of Unicode characters.
  \item A required \emph{functor}, which is a nonempty string of Unicode
    characters.
  \item A required \emph{arity}, which is an integer from 0 to 3 inclusive.
  \item A sequence of \emph{children}, of length equal to the arity (no
    children if arity is zero).  Each child is, recursively, an EIDS tree.
\end{itemize}

Trees with arity zero, one, two, and three are called, respectively,
nullary, unary, binary, and ternary.

Note that these ``nonempty strings of Unicode characters'' will very often
tend to be of length one (single characters) but that is not a requirement. 
They cannot be empty (length zero); the case of a tree without a head is
properly described by ``there is no head,'' not by ``the head is the empty
string.'' \emph{At present} no Unicode canonicalization is performed, that
being left to the user, but this may change in the future.  Zero bytes
(U+0000) are in principle permitted to occur in EIDS trees, but at present
there is no way to enter them in matching patterns because Unix passes
command-line arguments as null-terminated C strings.

Typically, these trees are used to describe kanji characters.  The literal
Unicode character being described will be the head, if there is a code point
for it; the functor will be either an ideographic description character like
\texttt{⿱} if the character can be subdivided, or else nullary \texttt{;}
if not.  Then the children will correspond to the parts into which it can be
decomposed.  Some parts of the character may also be available as
characters with Unicode code points in their own right; in that case, they
will have heads of their own.

%%%%%%%%%%%%%%%%%%%%%%%%%%%%%%%%%%%%%%%%%%%%%%%%%%%%%%%%%%%%%%%%%%%%%%%%

\section{EIDS syntax}

Unicode's IDS syntax serves a similar purpose to IDSgrep's extended IDS
syntax, but it lacks sufficient expressive power to cover some of IDSgrep's
needs.  Nonetheless, EIDS syntax is noticeably derived from that of Unicode
IDSes.  Broadly speaking, EIDSes are IDSes extended to include heads (which
we need for partial-character lookup); bracketed strings as functors (which
we need for capturing arbitrary data); and with arbitrary limits on allowed
characters and length relaxed (needed for complex characters and so that
matching patterns can be expressed in the same syntax).

Here are some sample EIDSes:
\begin{verbatim}
大
⿱田⿰虫⿱土土
⿸厂⿱今止
【萌】⿱艹<明>⿰日月
【店】⿸广<占>⿱卜口 
⿱艹⿰日?
&...男...女
[tb]艹[or][lr]?日[lr]日?
\end{verbatim}

The first three of these examples are valid in the Unicode IDS syntax.  The
next two contain heads, and are typical of what might exist in a dictionary
designed to be searched by the \texttt{idsgrep} command-line utility.  The
last three might be matching patterns a user would enter.

EIDS trees are written in a simple prefix notation that could be called
``Polish notation'' inasmuch as it is the reverse of ``reverse Polish
notation.'' To write a tree, simply write the head if there is one, the
functor, and then if the tree is not nullary, write each of the children. 
Heads and the functors of trees of different arity are (unless otherwise
specified below) written enclosed in different kinds of brackets that
indicate the difference between heads and functors, and the arity of the
tree when writing a functor.

The basic ASCII brackets for heads and functors are as follows:

\hspace*{\fill}
\begin{tabular}{cccc}
  head & \texttt{<} & \texttt{>} & \texttt{<example>} \\
  nullary functor (0) & \texttt{(} & \texttt{)} & \texttt{(example)} \\
  unary functor (1) & \texttt{.} & \texttt{.} & \texttt{.example.} \\
  binary functor (2) & \texttt{[} & \texttt{]} & \texttt{[example]} \\
  ternary functor (3) & \texttt{\{} & \texttt{\}} & \texttt{\{example\}}
\end{tabular}
\hspace*{\fill}\par

Note that the opening and closing brackets for unary functors are both equal
to the ASCII period, U+002E.

Parsing of bracketed strings has a few features worth noting.  First, there
is no special treatment of nested brackets.  After the ``\texttt{<}'' that
begins a head, for instance, the next ``\texttt{>}'' will end the head,
regardless of how many other instances of ``\texttt{<}'' have been seen. 
However, because no head or functor can be less than one character long, a
closing bracket immediately after the opening bracket (which would otherwise
create an illegal empty string) is specially treated as the first character
of the string and \emph{not} as a closing bracket.  Thus, ``\texttt{())}''
is legal syntax for a functor equal to a closing parenthesis, in a nullary
tree; and ``\texttt{...}'' is a functor equal to a single ASCII period in a
unary tree, an important example because it is the commonly-used
match-anywhere operator.

Each pair of ASCII brackets also has two pairs of generally
non-ASCII synonyms, as
follows:

{\ttfamily\hspace*{\fill}
\begin{tabular}{cccccc}
  <&>&【&】&〖&〗\\
  (&)&(&)&⦅&⦆\\
  .&.&:&:&・&・\\\relax
  [&]&[&]&〚&〛\\
  \{&\}&〔&〕&〘&〙
\end{tabular}
\hspace*{\fill}\par}

The closing synonymous brackets for functors of unary trees are always
identical to the opening brackets.  A string may be opened by any of the
three opening bracket characters for its type of string; but then it must be
closed by the closing bracket character that goes with the opening bracket. 
Brackets from other pairs are taken literally and do not end the string.
For instance,
``\texttt{【<example>】}'' is a head whose value consists of
``\texttt{<example>}'' including the ASCII angle brackets.  There are
several reasons for the existence of the synonyms:

\begin{itemize}
  \item They look cool.
  \item There is an established tradition of using \texttt{【}lenticular
    brackets\texttt{】} for heads in printed dictionaries, which is exactly
    their meaning here.
  \item Allowing ASCII colons to bracket unary-node functors makes possible
    a more appealing and \texttt{grep}-like syntax for \texttt{idsgrep}'s
    output in the case of processing multiple input files.
  \item Allowing more than one way to bracket each kind of string makes it
    easier to express bracket characters that may occur literally in a string.
  \item The non-ASCII brackets may be easier to type without switching modes
    in some input methods.
  \item On the other hand, keeping an ASCII option for every bracket type
    allows matching patterns to be entered on ASCII-only terminals.
  \item Multiple bracket types allow for creating human-visible
    computer-invisible distinctions in dictionary files, for instance to
    flag pseudo-entries that contain metadata, without needing to create a
    special syntax for comments.
\end{itemize}

If a character other than an opening bracket occurs in an EIDS where an
opening bracket would be expected, it is treated in one of three ways.

\begin{itemize}
  \item ASCII whitespace and control characters, U+0000 to U+0020 inclusive,
    are ignored.  In the future, this treatment might be extended to
    non-ASCII Unicode whitespace characters, which are best avoided because
    of the uncertainty.
  \item Some special characters, such as ``\texttt{⿰},'' have ``sugary
    implicit brackets.''  If one of these characters appears outside of
    brackets, it will be interpreted as a functor whose value is a
    single-character string equal to the literal character, and a fixed
    arity that depends on which character it is.  For instance,
    ``\texttt{⿰}'' and ``\texttt{[⿰]}'' will be parsed identically.
    A list of characters getting this treatment is below.
  \item Any other non-bracket character has a ``syrupy implicit semicolon.''
    That means it will be interpreted as a complete nullary tree with
    a single-character head equal to the literal character, and a
    single semicolon as the functor.  For instance, ``\texttt{x}'' and
    ``\texttt{<x>(;)}'' will be parsed identically.  Because semicolon
    itself has sugary implicit nullary brackets, we could also write
    ``\texttt{<x>;}'' for the same effect.
\end{itemize}

Here are all the characters that have sugary implicit brackets, with the
brackets they imply:  {\ttfamily (;) (?) .!. ./. .=. .*. .@. [\&] [|] [⿰]
[⿱] [⿴] [⿵] [⿶] [⿷] [⿸] [⿹] [⿺] [⿻] \{⿲\} \{⿳\}}

Note that the sugary and syrupy implications of a character are only
relevant when the character occurs where an opening bracket of some
type would otherwise be required; inside a bracketed string,
characters behave normally.

It is planned that in the future, \texttt{idsgrep}'s parser will also
recognize some backslash escape sequences.  This is not yet
implemented.

It is a consequence of these rules that all syntactically valid Unicode
IDSes are syntactically valid EIDSes, but the converse is not true.

Although it is technically not a parsing issue but rather a
transformation applied to the tree after parsing, there is one more
issue to mention: some functors have aliases.  If a functor and arity
matches one of the aliases on the following list, it will be replaced
with the indicated single-character functor.  The idea is to provide
verbose ASCII names for single-character functors of special
importance to the matching algorithm.  Note that the single-character
versions are always the canonical ones, and although the brackets are
shown explicitly for clarity, they are nearly all characters from the
``sugary implicit'' list.

\texttt{\begin{tabular}{cccccc}
  (anything) & $\Rightarrow$ & (?) & .anywhere. & $\Rightarrow$ & ... \\
  .not. & $\Rightarrow$ & .!. & .regex. & $\Rightarrow$ & ./. \\
  .equal. & $\Rightarrow$ & .=. & .unord. & $\Rightarrow$ & .*. \\
  .assoc. & $\Rightarrow$ & .@. & [and] & $\Rightarrow$ & [\&] \\\relax
  [or] & $\Rightarrow$ & [|] & [lr] & $\Rightarrow$ & [⿰] \\\relax
  [tb] & $\Rightarrow$ & [⿱] & [enclose] & $\Rightarrow$ & [⿴] \\\relax
  [wrapu] & $\Rightarrow$ & [⿵] & [wrapd] & $\Rightarrow$ & [⿶] \\\relax
  [wrapl] & $\Rightarrow$ & [⿷] & [wrapul] & $\Rightarrow$ & [⿸] \\\relax
  [wrapur] & $\Rightarrow$ & [⿹] & [wrapll] & $\Rightarrow$ & [⿺] \\\relax
  [overlap] & $\Rightarrow$ & [⿻] & \{lcr\} & $\Rightarrow$ & \{⿲\} \\\relax
  \{tcb\} & $\Rightarrow$ & \{⿳\}
\end{tabular}}

%%%%%%%%%%%%%%%%%%%%%%%%%%%%%%%%%%%%%%%%%%%%%%%%%%%%%%%%%%%%%%%%%%%%%%%%

\section{Matching}

The basic function of the \texttt{idsgrep} command-line utility is to
evaluate each item in the database against a matching pattern.  The matching
patterns are similar in spirit to the ``regular expressions'' common
throughout the Unix world; however, for theoretical and practical reasons
standard regular expressions would be unsuitable for the applications
considered by IDSgrep.

The main theoretical issue is that IDSes, whether IDSgrep-style ``extended''
or Unicode-style traditional ones, belong to the class of
\emph{context-free} languages.  They describe tree-like structures nested to
arbitrary depth, similar in nature to programming-language expressions
containing balanced parentheses although balanced parantheses as such are
not actually part of EIDS syntax.  The natural way to parse these involves
an abstract machine with a stack-like memory that can assume an infinite
number of different states.  Regular expressions can only be used to
recognize the smaller, simpler class of \emph{regular} languages, parsable
by an abstract machine with a finite-state memory.  It is not possible to
write a correct regular expression that will match balanced parentheses. 
Some advanced software implementations of so-called ``regular expressions''
(for instance, Perl's) contain special features that make them more powerful
than the standard theoretical model, so that they are capable of recognizing
some languages that are non-regular, including balanced parentheses.  It is
also possible to fake a stack with a finite depth limit by writing a
complicated regular expression, and that may be good enough in some
practical cases.  Some users may also settle for just doing a substring
query with \texttt{grep} and calling the result close enough.  But IDSgrep
tries to do it in a way that is really right, and that is described
precisely in this section.

We will define a function $\textit{match}(x,y)$ which takes two EIDS trees
as input and returns a Boolean value of true or false.  We
call $x$ the \emph{pattern} or \emph{needle} and $y$ the \emph{subject} or
\emph{haystack}.  The \texttt{idsgrep} command-line utility generally takes
$x$ from its command line and repeatedly evaluates this function for each
EIDS it reads from its input; it then writes out all the values of $y$ for
which $\textit{match}(x,y)$ is true.

The $\textit{match}(x,y)$ function is defined as follows:
\begin{itemize}
  \item If $x$ and $y$ both have heads, then $\textit{match}(x,y)$ is true
    if and only if their heads are identical.  Nothing else is examined (in
    particular, not the children).  Then the two cases below do not apply.
  \item If $x$ and $y$ do not both have heads, then
    $\textit{match}(x,y)=\textit{match}'(x,y)$, whose value
    generally depends on the functor and arity of $x$.  The
    $\textit{match}'$ function has many special cases described in the
    subsections below, expressing different kinds of special matching
    operations.  These operations roughly correspond to the ASCII
    characters with sugary implicit brackets in EIDS syntax.  They are
    shown with brackets for clarity in the discussion below, but users
    would generally type them without the brackets and depend on the
    sugar in actual use.
  \item If none of the subsections below applies, then
    $\textit{match}'(x,y)$ is true if and only if $x$ and $y$ have identical
    functors, identical arities, and $\textit{match}(x_i,y_i)$ is true
    recursively for all their corresponding children $x_i,y_i$.  Note that
    $\textit{match}'$ recurses to $\textit{match}$, not itself, so
    there is a chance for head matching on the children even if it was
    not relevant to the parent nodes.
\end{itemize}

Very few of the features below actually exist in the alpha version 0.2.  The
others are documented to give readers some idea of planned future
development.

\subsection{Match anything}

The value of $\textit{match}'(\texttt{(?)},y)$ is always true.  Thus,
\texttt{?}\ can be used as a wildcard in \texttt{idsgrep} patterns to match
an entire subtree regardless of its structure.  Mnemonic:  question mark
is a shell wildcard for matching a single character.
The verbose ASCII name for ``\texttt{(?)}'' is ``\texttt{(anything)}.''

\subsection{Match anywhere}

The value of $\textit{match}'(\texttt{...}x,y)$ is true if and only if there
exists any subtree of $y$ (including the entirety of $y$) for which
$\textit{match}'(x,y)$ is true.  In other words, this will look for an
instance of $x$ anywhere inside $y$ regardless of nesting level.  Mnemonic:
three dots suggest omitting a variable-length sequence, in this case the
variable-length chain of ancestors above $x$.
The verbose ASCII name for ``\texttt{...}'' is ``\texttt{.anywhere.}.''

\subsection{Match children in any order}

The value of $\textit{match}'(\texttt{.*.}x,y)$ is true if and only if there
exists a permutation of the children of $y$ such that $\textit{match}(x,y')$
is true of the resulting modified $y'$.  For instance, \texttt{*[a]bc}
matches both \texttt{[a]bc} and \texttt{[a]cb}.  This is obviously a
no-operation (matches simply if $x$ matches $y$, as if the asterisk were not
applied) for trees of arity less than two.  Mnemonic: asterisk is a general
wildcard, and this is a general matching operation.
The verbose ASCII name for ``\texttt{.*.}'' is ``\texttt{.unord.}.''

\subsection{NOT}

The value of $\textit{match}'(\texttt{.!.}x,y)$ is true if and only if
$\textit{match}(x,y)$ is false.  It matches any tree \emph{not} matched by
$x$ alone.  Mnemonic:  prefix exclamation point is logical NOT in many
programming languages.
The verbose ASCII name for ``\texttt{.!.}'' is ``\texttt{.not.}.''

\subsection{AND}

The value of $\textit{match}'(\texttt{[\&]}xy,z)$ is true if and only if
$\textit{match}(x,z) \wedge \textit{match}(y,z)$.  In other words, it
matches all trees that are matched by both $x$ and $y$; the set of strings
matched by $\texttt{[\&]}xy$ is the intersection of the sets matched by $x$
and by $y$.  Mnemonic: ampersand is logical or bitwise AND in many
programming languages.
The verbose ASCII name for ``\texttt{[\&]}'' is ``\texttt{[and]}.''

\subsection{OR}

The value of $\textit{match}'(\texttt{[|]}xy,z)$ is true if and only if
$\textit{match}(x,z) \vee \textit{match}(y,z)$.  In other words, it matches
all trees that are matched by at least one of $x$ or $y$; the set of strings
matched by $\texttt{[|]}xy$ is the union of the sets matched by $x$ and by
$y$.  Mnemonic: ASCII vertical bar is logical or bitwise OR in many
programming languages.
The verbose ASCII name for ``\texttt{[|]}'' is ``\texttt{[or]}.''

\subsection{Literal tree matching}

If $x$ and $y$ both have heads, then the value of
$\textit{match}'(\texttt{.=.}x,y)$ is true if and only if those heads are
identical.  Otherwise, it is true if and only if $x$ and $y$ have identical
functors, identical arity, and $\textit{match}(x_i,y_i)$ is true for each of
their corresponding children.

The effect of this operation is to ignore any special $\textit{match}'()$
semantics of $x$'s functor; the trees are compared as if that functor were
just an ordinary string, regardless of whether it might normally be special. 
Note that the full $\textit{match}()$ is still done on the children with
only the root taken literally; to do a completely literal match of the
entire trees it is necessary to insert an additional copy of \texttt{.=.}
above every node in the matching pattern, or at least every node that would
otherwise have a special meaning for $\textit{match}'()$, and even then
heads will continue to have their usual effect of overriding
recursion.\footnote{It may be interesting to consider how one could write a
pattern to test absolute identity of trees, with each node matching if and
only if its head or lack thereof is identical to the desired target as well
as the functors and arities matching and the same being true of all
children.} Mnemonic: equals sign suggests the literal equality that is being
tested rather than the more complicated comparisons that might otherwise be
used.
The verbose ASCII name for ``\texttt{.=.}'' is ``\texttt{.equal.}.''

For instance, this feature could allow searching for a unary tree whose
functor actually is \texttt{!}, where just specifying such a tree directly
as the matching pattern would instead (under the rule for ``NOT'' above)
search for trees that do not match the only child of \texttt{!}.  In the
original application of searching kanji decomposition databases this
operation is unlikely to be used because the special functors do not occur
anyway, but it seems important for potential applications of IDSgrep to more
general tree-querying, because otherwise some reasonable things people might
want to look for could not be found at all.

\subsection{Associative matching}

The value of $\textit{match}'(\texttt{.@.}x,y)$ is calculated as follows. 
Create a new EIDS tree $x'$, initially equal to $x$, which has the property
that its root may be of unlimited arity.  Then for every child of $x'$ whose
functor and arity are identical to the functor and arity of $x$, replace
that child in $x'$ with its children, in order.  Repeat that operation until
no more children of $x'$ have functor and arity identical to the
functor and arity of $x$.  Compute $y'$ from $y$ by the same process.  Then
$\textit{match}'(\texttt{.@.}x,y)=\textit{match}(\texttt{.=.}x',y')$.

This matching operator is intended for the case of three or more
things combined using a binary operator that has, or can be said to sometimes
have, an associative law.  For instance, the kanji \texttt{怠} could be
described by ``\texttt{⿱⿱厶口心}'' (\texttt{⿱厶口} over \texttt{心}) or
by ``\texttt{⿱厶⿱口心}'' (\texttt{厶} over \texttt{⿱口心}).  Unicode
might encourage use of the ternary operator \texttt{⿳} for this particular
case instead, but that does not cover all reasonably-occurring cases, and
the default databases seldom if ever use the Unicode ternary operators.

The difference between the representations is sometimes useful information
that the database \emph{should} retain; for instance, in the case of
Tsukurimashou, ``\texttt{⿱⿱厶口心},'' ``\texttt{⿱厶⿱口心},'' and
``\texttt{⿳厶口心}'' would correspond to three very different
stanzas of MetaPost source code, and the user might want a query
that separates them.  On the other hand, the user might instead have a more
general query along the lines of ``find three things stacked vertically with
\texttt{心} at the bottom'' and intend that that should match both cases of
binary decomposition.  The at-sign matching operation is meant for queries
that don't care about the order of binary operators; without it, matching
will by default follow the tree structure strictly.

Note that even with \texttt{.@.}, IDSgrep will not consider binary operators
in any way interchangeable with ternary ones; users must still use
\texttt{.|.} to achieve such an effect if desired.  Although the at-sign is
fully defined for all arities, it is only intended for use with binary
trees.  Note also that \texttt{.@.} and \texttt{.*.} behave according to
their definitions.  Incautious attempts to use them together will often fail
to have the desired effects, because the definitions do not include special
exceptions that some users might intuitively expect for these two operators
happening to occur near each other.  In a pattern like
``\texttt{*@[a][a]bcd},'' \texttt{.*.} will recognize \texttt{.@.} as the
functor of a unary tree and expand the single permutation of its one child,
and so that pattern will match the same things as if the asterisk had not
been present, namely ``\texttt{[a][a]bcd}'' and ``\texttt{[a]b[a]cd]}'' but
not, for instance, ``\texttt{[a][a]dcb}.'' In a pattern like
``\texttt{@[a]b*[a]cd},'' \texttt{.@.} will recognize \texttt{.*.} as a
different arity and functor from \texttt{[a]} and choose not to expand it in
$x'$, with the result that that pattern matches the same things as if the
at-sign had not been present, namely ``\texttt{[a]b[a]cd}'' and
``\texttt{[a]b[a]dc}'' but not ``\texttt{[a][a]bcd}'' nor
``\texttt{[a][a]bdc}.''

When considered as an operation on trees, what \texttt{.@.} does is
fundamentally the same thing as the algebraic operation that considers
$(a+b)+c$ equivalent to $a+(b+c)$, and for that reason it is called
``associative'' matching.  The mnemonic for at-sign is that it is a fancy
``a'' for ``associative.''
The verbose ASCII name for ``\texttt{.@.}'' is ``\texttt{.assoc.}.''

This feature is not yet implemented in version 0.1.

\subsection{Regular expression matching}

It is planned that some future version (likely version 0.3) will support
special behaviour for $\textit{match}'(\texttt{./.}x,y)$ to call a regular
expression library and do string matching within heads or functors, but
the detailed semantics of how that will work are not yet decided.  The
mnemonic is that slash is Perl's regular-expression match operator; the
motivating application is to further development of IDSgrep's own
dictionary-generating programs, which tend to create long nullary functors
full of debugging information when they encounter constructs they don't
understand in the other-format dictionaries they read.

%%%%%%%%%%%%%%%%%%%%%%%%%%%%%%%%%%%%%%%%%%%%%%%%%%%%%%%%%%%%%%%%%%%%%%%%
%%%%%%%%%%%%%%%%%%%%%%%%%%%%%%%%%%%%%%%%%%%%%%%%%%%%%%%%%%%%%%%%%%%%%%%%
%%%%%%%%%%%%%%%%%%%%%%%%%%%%%%%%%%%%%%%%%%%%%%%%%%%%%%%%%%%%%%%%%%%%%%%%

\clearpage
\addcontentsline{toc}{chapter}{Bibliography}
\bibliographystyle{plain}
\bibliography{idsgrep}

%%%%%%%%%%%%%%%%%%%%%%%%%%%%%%%%%%%%%%%%%%%%%%%%%%%%%%%%%%%%%%%%%%%%%%%%
%%%%%%%%%%%%%%%%%%%%%%%%%%%%%%%%%%%%%%%%%%%%%%%%%%%%%%%%%%%%%%%%%%%%%%%%
%%%%%%%%%%%%%%%%%%%%%%%%%%%%%%%%%%%%%%%%%%%%%%%%%%%%%%%%%%%%%%%%%%%%%%%%

\end{document}
